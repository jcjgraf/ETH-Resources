%! TEX root = ./main.tex

\section{Axiomatic Semantics}
\begin{itemize}
    \item Expresses specific properties of the effect of executing a program
    \item Some aspects of the computation may be ignored
    \item Useful for program verification
    \ides{Partial Correctness:} Expresses that if a program terminates then there will be a certain relationship between the initial and the final state
    \ides{Total Correctness:} Expresses that a program will terminate and there will be a certain relationship between the initial and the final state
        \begin{itemize}
            \item Total Correctness $=$ Partial Correctness $+$ Termination
        \end{itemize}
    \item Proofs are too detailed when using operational semantics
    \ides{Hoare Triples:} The system we use
        \begin{itemize}
            \item $\{ P \} s \{ Q \}$
                \begin{itemize}
                    \ides{P:} Precondition (Assertion)
                    \ides{Q:} Postcondition (Assertion)
                    \ides{s:} Statement
                \end{itemize}
            \item If $P$ evaluates to true in an initial state $\sigma$, and if the execution of $s$ from $\sigma$ terminates in an state $\sigma'$ then $Q$ will evaluate to true in $\sigma'$
                \begin{itemize}
                    \item Describes parietal correctness
                \end{itemize}
            \ides{Local Variables}
                \begin{itemize}
                    \item Can be used to save a value in the inital state so that it can be referenced later
                    \item Occur only in assertions
                    \item Are never assigned to and are not used by the program
                \end{itemize}
            \ides{Assertions}
                \begin{itemize}
                    \item Consists of boolean expression with local variables (optional)
                        \begin{itemize}
                            \item Can be extended with other expressions like quantifiers, new operators etc.
                        \end{itemize}
                    \item Pre- and postcondition are assertions
                    \item We use some convenience notations like $\wedge$ for $\mathtt{and}$ etc.
                \end{itemize}
            \ides{Derivation System}
                \begin{itemize}
                    \ides{Rules}
                        \begin{itemize}
                            \ides{Skip}
                                \begin{itemize}
                                    \item $\Inf[\text{SKIP}_\text{Ax}]{\{\ub{P}\} \mathtt{skip} \{\ub{P}\}}$
                                \end{itemize}
                            \ides{Assignment}
                                \begin{itemize}
                                    \item $\Inf[\text{ASS}_\text{Ax}]{\{\ub{P}[\ub{x} \mapsto \ub{e}\} \ub{x} := \ub{e} \{\ub{P}\}}$
                                \end{itemize}
                            \ides{Sequential Composition}
                                \begin{itemize}
                                    \item $\Inf[\text{SEQ}_\text{AX}]{\{\ub{P}\} \ub{s} \{\ub{Q}\}}{\{\ub{Q}\} \ub{s}' \{\ub{R}\}}{\{\ub{P}\} \ub{s}; \ub{s}' \{\ub{R}\}}$
                                \end{itemize}
                            \ides{Conditional Statement}
                                \begin{itemize}
                                    \item $\Inf[\text{IF}_\text{AX}]{\{\ub{b} \wedge \ub{P}\} \ub{s} \{\ub{Q}\}}{\{\neg \ub{b} \wedge \ub{P}\} \ub{s}' \{\ub{Q}\}}{\{\ub{P}\} \texttt{if} \ \ub{b} \ \texttt{then} \ \ub{s} \ \texttt{else} \ \ub{s}' \ \texttt{end} \{\ub{Q}\}}$
                                \end{itemize}
                            \ides{Loop}
                                \begin{itemize}
                                    \item $\Inf[\text{WH}_\text{Ax}]{\{\ub{b} \wedge \ub{P}\} \ub{s} \{\ub{P}\}}{\{\ub{P}\} \texttt{while} \ \ub{b} \ \texttt{do} \ \ub{s} \ \texttt{end} \{\neg \ub{b} \wedge \ub{P}\}}$
                                    \item The assertion $P$ is the loop invariant
                                \end{itemize}
                            \ides{Consequence}
                                \begin{itemize}
                                    \item $\Inf[\text{CONS}_\text{Ax}][\mathtt{if} \ \ub{P} \models \ub{P}' \ \texttt{and} \ \ub{Q}' \models \ub{Q}]{\{\ub{P}'\} \ub{s} \{\ub{Q}'\}}{\{\ub{P}\} \ub{s} \{\ub{Q}\}}$
                                    \ides{Semantic Entailment $\mathbf{\models}$:}  $P \models Q \iff \forall \sigma, \mc{B}[[P]] \sigma = \text{tt} \implies \mc{B}[[Q]]\sigma = \text{tt}$
                                \item Strengen precondition
                                \item Weaken postcondition
                                \end{itemize}
                        \end{itemize}
                    \ides{Derivation Tree}
                        \begin{itemize}
                            \item As we are used to
                            \item $\vdash \{P\} s \{Q\} \iff \exists T. \text{root}(T) \equiv \{P\} s \{Q\}$
                        \end{itemize}
                    \ides{Proof}
                        \begin{itemize}
                            \item Derivation trees tend to get very larege
                            \item We write assertions before and after each statement
                            \item We write rule instances vertically
                            \item Proof outlines are often best developed bottom-up
                            \item Loop-invariant is determined by looking how the value changes in consecutive iterations
                        \end{itemize}
                \end{itemize}
        \end{itemize}
    \ides{Properties}
        \begin{itemize}
            \item Properties are typically proven by induction on the shape of derivation tree
                \begin{itemize}
                    \item Structural induction does often not work due to the rule of consequence
                \end{itemize}
            \ides{Semantic Equivalence}
                \begin{itemize}
                    \ides{Semantically equivalent} are two statements $s_1, s_2$ if $\forall P, Q, \vdash \{P\} s_1 \{Q\} \iff \vdash \{P\} s_2 \{Q\}$
                \end{itemize}
        \end{itemize}
    \ides{Total Correctness (Termination)}
        \begin{itemize}
            \ides{Total Correctness:} If $P$ evaluates to true in the initial state $\sigma$ then the execution of $s$ from $\sigma$ terminates and $Q$ will evaluate to true in the final statement
            \ides{Notation:} $\{P\} s \{\Downarrow Q\}$
            \ides{Loop Variant:}
                \begin{itemize}
                    \item Expression that evaluates to a value in a well-founded set before each iteration
                    \item Each loop iteration must decrease the value of the invariant
                    \item Loop has to terminate once the minimal value of the well-founded set is reached
                    \item Used to prove termination
                \end{itemize}
            \item This is a separate axiomatic semantic and is not mixed with the previous one
            \ides{Rules}
                \begin{itemize}
                    \ides{Loop}
                        \begin{itemize}
                            \item $\Inf[\text{WHTOT}_\text{Ax}][\mathtt{if} \ \ub{b} \wedge \ub{P} \models 0 \le \ub{e} \text{ and } Z \notin \ub{P}]{\{\ub{b} \wedge \ub{P} \wedge \ub{e} = Z\} \ub{s} \{\Downarrow \ub{P} \wedge \ub{e} < Z\}}{\{\ub{P}\} \texttt{while} \ \ub{b} \ \texttt{do} \ \ub{s} \ \texttt{end} \{\Downarrow \neg \ub{b} \wedge \ub{P}\}}$
                        \end{itemize}
                \end{itemize}

        \end{itemize}
\end{itemize}

\subsection{Soundness and Completeness}
\begin{itemize}
    \ides{Soundness:} If a property can be prove then it does indeed hold
        \begin{itemize}
            \item $\vdash \{P\} s \{Q\}  \implies \models \{P\} s \{Q\}$
        \end{itemize}
    \ides{Completeness:} If a property does hold then it can be proved
        \begin{itemize}
            \item $\models \{P\} s \{Q\} \implies \vdash \{P\} s \{Q\}$
        \end{itemize}
    \item Hard to create an axiomatic semantic which is sound and complete
    \item Soundness and completeness can be proved with respect to an operational semantics
        \begin{itemize}
            \item $\{P\} s \{Q\}$ is valid, written as $\models \{P\} s \{Q\}$ iff:\\ $\forall \sigma, \sigma'. \mc{B}[[P]]\sigma = \text{tt} \wedge \vdash \langle s, \sigma \rangle \to \sigma' \implies \mc{B}[[Q]] \sigma' = \text{tt}$
            \item I.e. $\models \{P\} s \{Q\}$ is ture if, whenever we start execution of $s$ from a state where $P$ holds, if the execution terminates, then $Q$ will hold in the final state
        \end{itemize}
    \ides{Theorem:} For all partial correctness triplets $\{P\} s \{Q\}$ of IMP we have $\vdash \{P\} s \{Q\} \iff \models \{P\} s \{Q\}$
        \begin{itemize}
            \ides{Proof Idea:}
                \begin{itemize}
                    \ides{$\mathbf{\Rightarrow}$:} Induction on the shape of the derivation tree for $\{P\} s \{Q\}$
                    \ides{$\mathbf{\Leftarrow}$:} Induction but using some weakest precondition stuff
                \end{itemize}
        \end{itemize}
\end{itemize}
