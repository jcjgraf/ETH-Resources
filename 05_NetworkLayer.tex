%! TEX root = ./main.tex

\section{Network Layer}
\subsection{General}
\begin{itemize}
    \item Builds on link layer and provides service to transport layer
    \item Transports packets
    \item Challenges
        \begin{itemize}
            \item Scale to a global internet
            \item Heterogeneity
            \item Bandwidth control
            \item Economics
        \end{itemize}
    \ides{Routing:} Figure out which path to take
    \ides{Forwarding:} Send packet on its way
    \ides{Network Service Models}
        \begin{itemize}
            \item Describes what service does it provide to the transport layer
            \ides{Store-And-Forward Packet Switching:}
                \begin{itemize}
                    \item Both are implemented with store-and-forward packet switching
                    \item Routers receive packages, and store it (if necessary) in a FIFO quere before forwarding it
                        \begin{itemize}
                            \item Have a buffer per port (slightly simplified view)
                        \end{itemize}
                \end{itemize}
            \item Provide two different service models
            \ides{Datagrams (/Connection Less Service)}
                \begin{itemize}
                    \item Packets contain a destination address
                    \item Destination address is used to forward each packet from router to router
                        \begin{itemize}
                            \item Possible over different paths
                        \end{itemize}
                    \item Like postal letters
                    \item Prime type used today
                    \item Used for IP protocol
                    \ides{Forwarding Table:}
                        \begin{itemize}
                            \item Used by each router
                            \item Keyed by address
                            \item Gives next hop for each destination address
                            \item Is dynamic
                                \begin{itemize}
                                    \item Changes when new links are created and old are removed
                                \end{itemize}
                        \end{itemize}
                \end{itemize}
            \ides{Virtual Circuits (Connection-Oriented Service)}
                \begin{itemize}
                    \item Uses circuit switching (in a virtual sense)
                        \begin{itemize}
                            \item There is no bandwidth reservation
                            \item But statistical sharing of link
                        \end{itemize}
                    \item Like a telephone call
                    \item Three phases
                        \begin{itemize}
                            \item Connection establishment
                                \begin{itemize}
                                    \item Circuit is set up
                                    \item Path is chosen
                                \end{itemize}
                            \item Data transfer
                                \begin{itemize}
                                    \item Circuit is used
                                    \item Packets are forwarded
                                \end{itemize}
                            \item Connection teardown
                                \begin{itemize}
                                    \item Circuit is deleted
                                    \item Circuit information is removed from all routers
                                \end{itemize}
                        \end{itemize}
                    \item Each packet has a shorter label
                        \begin{itemize}
                            \item Used to identify the circuit and not destination
                        \end{itemize}
                    \ides{Forwarding Table}
                        \begin{itemize}
                            \item Used by each router
                            \item Keyed by the circuit identifier
                                \begin{itemize}
                                    \item Last router name, and label
                                \end{itemize}
                            \item Gives next router name and new packet label
                        \end{itemize}
                    \ides{Multi-Protocol Label Switching (MPLS)}
                        \begin{itemize}
                            \item Virtual-Circuit like
                            \item Used by many ISPs in their backbone
                            \item Adds MPLS fields upon entering their network
                            \item Removed MPLS field upon leaving their network
                            \item Takes up some digits of the IP
                            \ipro Allows forwarding on routes not possible using standard IP protocol
                            \ipro Potential increased switching speed
                                \begin{itemize}
                                    \item No longest-prefix matching required
                                \end{itemize}
                            \icon Dated
                            \icon Unflexible
                            \icon Hard to setup
                        \end{itemize}
                \end{itemize}
            \item Datagrams vs Virtual Circuits
                \begin{itemize}
                    \item
                    \begin{tabular}{| l | l | l |}
                        Setup phase & \textbf{Not needed} & Required\\
                        Router State & \textbf{Per destination} & Per connection\\
                        Addresses & Packet carries full address & \textbf{Packet carries short label}\\
                        Routing & Per packet & \textbf{Per circuit}\\
                        Failures & \textbf{Easier to mask} & Difficult to mask\\
                        Quality of Service & Difficult to add & \textbf{Easier to add}\\
                    \end{tabular}
                \end{itemize}
        \end{itemize}
    \ides{Internetworking}
        \begin{itemize}
            \item Connection different networks together
            \item Networks are different in:
                \begin{itemize}
                    \item Service model
                    \item Addressing
                    \item QoS
                    \item Packet size
                    \item Security
                \end{itemize}
            \item Hides the differences with a common protocol
            \ides{Internet Protocol (IP):}
                \begin{itemize}
                    \item Lowest common denominator
                    \item Asks for little
                    \item Provides little
                \end{itemize}
        \end{itemize}
\end{itemize}

\subsection{IPv4}
\begin{itemize}
    \ides{IP Datagram}
        \begin{itemize}
            \item Each row is $32$ bits
            \ides{Version:} $4$ or $6$
            \ides{IHL:}
            \ides{Type of Service:} distinguish different types of datagrams
                \begin{itemize}
                    \item real-time, non-real time etc.
                    \item For QoS
                \end{itemize}
            \ides{Datagram length (bytes):} total length of datagram in bytes
                \begin{itemize}
                    \item At most $2^{16}$
                \end{itemize}
                \ides{Identification:} \todo{What is identification for?}
            \ides{Flags:} Used for fragmentation
            \ides{Fragment Offset:} Used for fragmentation
            \ides{TTL:} Maximal desired number of hops
                \begin{itemize}
                    \item Decreased by one at every hop
                \end{itemize}
            \ides{Protocol:} TCP or UDP
                \begin{itemize}
                    \item Only considered at final destination
                \end{itemize}
            \ides{Header Checksum:} Checksum over header
            \ides{Source Address:} $32$ bit source
            \ides{Destination Address:} $32$ bit destination
            \ides{Options:} Optional fields to set
        \end{itemize}
    \ides{IP Addresses and Prefixes}
        \begin{itemize}
            \item $32$ bits long
            \ides{Dotted Nottation:} Split into four parts of $8$ bits length
                \begin{itemize}
                    \item $a.b.c.d$
                \end{itemize}
            \ides{Prefixes:} Addresses are allocated in blocks
                \begin{itemize}
                    \item Blocks are called prefixes
                    \item All addresses in the same $L$-bit prefix have the same top $L$ bits
                    \item There are $2^{32 - L}$ addresses in an $L$-bit block
                    \item The remaining represents the hosts on the network
                    \item Written as \verb+IP Address / Prefix Length+
                        \begin{itemize}
                            \item IP address is the first address in the block
                        \end{itemize}
                    \ides{More specific:} longer prefix
                    \ides{Less Specific:} shorter prefix
                    \item Fist and last address in of prefix cannot be used
                        \begin{itemize}
                            \ides{Network Identifier:} first address in prefix
                                \begin{itemize}
                                    \item Hosts bits are all zero
                                \end{itemize}
                            \ides{Broadcast Address:} last address in prefix
                                \begin{itemize}
                                    \item Host bits are all one
                                \end{itemize}
                        \end{itemize}
                    \ides{Network Mask:} Together with the address give the prefix
                        \begin{itemize}
                            \item Bitwise and of mask and address give prefix/network identifier
                                \begin{itemize}
                                    \item Network bits are one
                                    \item Host bits are zero
                                \end{itemize}
                        \end{itemize}
                \end{itemize}
                \ides{Public Addresses}
                    \begin{itemize}
                        \item Assigned by IANA to different RIRs (regional internet registries)
                        \item RIRs issue ranges to ISP
                        \item ISP issue to customers
                        \item Buy IP4 is difficult
                            \begin{itemize}
                                \item Rund out of addresses
                                \item Expensive
                                \item May have been be misused before
                                \item Geolocation
                            \end{itemize}
                    \end{itemize}
                \ides{Private Addresses}
                    \begin{itemize}
                        \item Only valid within a private network
                        \item Three prefixes are available
                            \begin{itemize}
                                \item $10.0.0.0/8$
                                \item $172.16.0.0/12$
                                \item $192.168.0.0/16$
                            \end{itemize}
                        \item Need public IP and NAT to connect to global internet
                    \end{itemize}
        \end{itemize}
    \ides{IP Forwarding}
        \begin{itemize}
            \item All IP addresses on one network have the same prefix
            \item Next hop is determined using forwarding table
                \begin{itemize}
                    \item Indexed using IP prefix
                    \item Prefixes may overlap
                \end{itemize}
            \ides{Longest Matching Prefix Forwarding Rule:}
                \begin{itemize}
                    \item Algorithm
                        \begin{itemize}
                            \item For each packet find all matching prefixes
                            \item Find the longest prefix from that set
                            \item Forward the packet to the next hop router for that prefix
                        \end{itemize}
                    \ides{Compress Forwarding Table:} Remove redundant forwarding rules
                        \begin{itemize}
                            \item I.e. rules which are overwritten by more specific ones
                        \end{itemize}
                    \item Can provide default behaviour
                        \begin{itemize}
                            \item With less specific prefixes
                        \end{itemize}
                    \item Can provide special behaviour
                        \begin{itemize}
                            \item With more specific prefixes
                        \end{itemize}
                    \item Hierarchical addresses allow compact tables
                    \item Find longest matching prefix is more complex than table lookup
                \end{itemize}
            \icon Size of forwarding tables is keep growing
            \ides{Host Forwarding}
                \begin{itemize}
                    \item Host has a lookup table with two entries:
                        \begin{itemize}
                            \ides{Network Prefix:} Send all traffic directly to the host
                            \ides{Default Route:} $0.0.0.0/0$
                                \begin{itemize}
                                    \item Matches all addresses
                                    \item Any more specific address is captured first
                                \end{itemize}
                        \end{itemize}
                \end{itemize}
            \item At each forward, the router must
                \begin{itemize}
                    \item Decrement TTL
                    \item Check and recalculate header checksum
                    \item Fragment larger packets if link network MTU is limited
                    \item Send congestion signals
                    \item Generate error messages
                    \item Handle options specified in the IP header
                \end{itemize}
        \end{itemize}
\end{itemize}

\subsection{IP Helper Protocols}
\begin{itemize}
    \item IP \textbf{requires} help from other protocols
    \item There area many of then
    \item Often involve broadcast
    \ides{Dynamic Host Configuration Protocol (DHCP)}
        \begin{itemize}
            \item How does a host get
                \begin{itemize}
                    \item Its IP address
                    \item Routers IP address
                    \item Network prefix
                    \ides{Default gateway}: Local router address
                    \item DNS Server
                    \item etc.
                \end{itemize}
            \item On connection, only the link MAC address is known
            \item Introduces in 1993
            \item Is a client - server application
            \item Part of the network (router)
            \item Uses UDP on port 67, 68
            \item Steps
                \begin{itemize}
                    \ides{Broadcast Message:} Client sends \verb+DISCOVER+ message to broadcast address
                        \begin{itemize}
                            \item Message is received by all hosts on network
                            \item \verb+255.255.255.255+ for IP and \verb+ff:ff:ff:ff:ff:ff+ for MAC \todo{Why needed?}
                        \end{itemize}
                     \item DHCP server send \verb+OFFER+ and offers certain options
                     \item Client send \verb+REQUEST+ to accept an offer
                     \item Server sends \verb+ACK+ an leases IP to client
                \end{itemize}
            \item Renewing lease only requires \verb+REQUEST+ followed by \verb+ACK+
            \icon Does not provide any security
                \begin{itemize}
                    \item Malicious host can act as a DHCP server
                \end{itemize}
        \end{itemize}
    \ides{Address Resolution Protocol (ARP)}
        \begin{itemize}
            \item Node needs link layer address to send frame but only has destination IP address
            \item Sits on the link layer
            \item Only works if node and target are on the same link
                \todo{What does that mean?}
            \item Steps
                \begin{itemize}
                    \item Nodes sends a \verb+REQUEST+ broadcast message with an IP address
                    \item Target sees broadcast and notices its IP address
                    \item Target sends \verb+REPLY+ containing its MAC address
                    \item Note caches the MAC address in its ARP cache
                \end{itemize}
            \item Asks nodes addressed with their IP address to identify themselves
        \end{itemize}
\end{itemize}

\subsection{Packet Size}
\begin{itemize}
    \ides{Maximum Transmission Unit (MTU)}
        \begin{itemize}
            \item Maximal allowed packet size
        \end{itemize}
    \item Different networks have different MTUs
    \item Goal: send packets as large as possible
        \begin{itemize}
            \item Less header overhead
            \item Fever packets required
        \end{itemize}
    \icon Hard to figure out appropriate size to send packets
    \item Two methods to solve this problem
        \begin{itemize}
            \ides{Fragmentation}
                \begin{itemize}
                    \item If a packet is too large for e given link, the router splits the packet into multiple smaller ones
                        \begin{itemize}
                            \item Break data into pieces
                            \item Copy main IP header into header of pieces
                            \item Adjust the \verb+length+ in the header
                            \item Set \verb+fragment offset+ in the header to indicate the position in the packet
                            \item Set \verb+MF+ (more fragments) \verb+flags+ in IP header on all pieces except the last
                                \begin{itemize}
                                    \item If set it means a fragment is following the current one
                                \end{itemize}
                        \end{itemize}
                    \item Receiver reassembles the fragmented packets
                        \begin{itemize}
                            \item \verb+Identification+ links pieces together
                            \item \verb+MF+ tells than all pieces are together
                        \end{itemize}
                    \icon Repeated fragmentation is possible $\implies$ messy
                    \icon Fragmentation is undesirable
                        \begin{itemize}
                            \icon More work for routers
                            \icon Fragment loss $\implies$ packet loss $\implies$ retransmission of whole packet
                            \icon Security vulnerability
                        \end{itemize}
                    \item Not used today anymore
                \end{itemize}
            \ides{Path MTU Discovery}
                \begin{itemize}
                    \item Host sends largest packets
                        \begin{itemize}
                            \item \verb+DF+ flag (do not fragment) set
                            \item Hosts use heuristisch on which MTUs to try
                                \begin{itemize}
                                    \item There are protocols in practice but they are not used
                                \end{itemize}
                        \end{itemize}
                    \item Routers send ICMP and tell their max largest size in case the original packet was too large
                    \item Hosts adjusts packet size to fit the MTU of the path
                    \icon Path may change and break the transmission
                    \ipro Avoids fragmentation
                    \item Used today
                \end{itemize}
        \end{itemize}
\end{itemize}

\subsection{Internet Control Message Protocol (ICMP)}
\begin{itemize}
    \item Sits on top of IP
    \item Is carried in an IP packet
    \item Used for:
        \begin{itemize}
            \ides{Error:} Router sends ICMP when it encounters an error during forwarding
            \ides{Testing:} \todo{Not sure what can be tested}
        \end{itemize} 
    \ides{Format}
        \begin{itemize}
            \item Header contains type, code and checksum
            \item Payload is often the problematic IP packet
                \begin{itemize}
                    \item For the host to identify the problem
                \end{itemize}
            \item Types:
                \begin{itemize}
                    \item 
\begin{tabular}{| l | l | l}
    Dest. Unreachable & 3/0 or 1 & Lack of connectivity\\
    Dest. Unreachable & 3/4 & Path MTU Discovery\\
    Time Exceeded (Transit) & 11/0 & Traceroute\\
    Echo Request or Reply & 8 or 0/0 & Ping
\end{tabular}
                \end{itemize}
        \end{itemize}
    \ides{Traceroute:}
        \begin{itemize}
            \item Allows host to find every router hop
            \item Sends probe packets with increasing TTL
                \begin{itemize}
                    \item Start with $1$ and increase to $N$
                \end{itemize}
            \item When TTL is $0$ for a router, it reports that using ICMP
                \begin{itemize}
                    \item We can identify the router
                \end{itemize}
            \icon Routers can identify probe and behave differently
        \end{itemize}
    \icon Has not security measurements
\end{itemize}

\subsection{Network Address Translation (NAT)}
\begin{itemize}
    \item Routers are supposed at only IP headero
    \ides{Middlebox:} Device which should not but looks at transmission protocol too
        \begin{itemize}
            \item Allows for new functionality
                \begin{itemize}
                    \item NAT Box
                    \item Firewall
                    \item Intrusion detection
                \end{itemize}
                \ipro Possible rapid deployment path when there is not other option \todo{What does it mean?}
            \ipro Control over many hosts
            \icon Break end-to-end connectivity
                \begin{itemize}
                    \item Strange side-effects
                \end{itemize}
            \icon Poor vantage point for many tasks
            \icon Cause internet ossification
                \begin{itemize}
                    \item Almost impossible to deploy new transport protocols
                \end{itemize}
        \end{itemize}
    \ides{NAT Box}
        \begin{itemize}
            \item Done by a middlebox
            \item Connects an internal network to an external network
                \begin{itemize}
                    \item Does multiplexing
                        \begin{itemize}
                            \item Many local hosts - single external IP
                        \end{itemize}
                \end{itemize}
            \item Translates addresses and ports
        \end{itemize}
    \ides{How NAT works}
        \begin{itemize}
            \item Keeps a internal/external table
                \begin{itemize}
                    \item Contains internal to external IP and Port mapping
                        \begin{itemize}
                            \item Port mapping is required that incoming packet get to the right host (in case there are multiple hosts with the same port)
                        \end{itemize}
                \end{itemize}
            \ides{Internal $\mathbf{\to}$ external:} When an internal hosts establishes a TCP connection to the outside
                \begin{itemize}
                    \item Lookup in table
                    \item Create new entry if none exists
                    \item Rewrite source IP/Port of IP header
                \end{itemize}
            \ides{External $\mathbf{\to}$ internal:}
                \begin{itemize}
                    \item Lookup in table
                    \item Rewrite address and port
                    \item If no entry in table the connection is blocked
                        \begin{itemize}
                            \item Need to configure entry manually
                        \end{itemize}
                \end{itemize}
        \end{itemize}
    \icon Connectivity broken
        \begin{itemize}
            \item Only works for initial outgoing traffix
        \end{itemize}
    \icon Problematic with UDP
        \begin{itemize}
            \item Since there is no connection \todo{How is this solved?}
        \end{itemize}
    \ipro Multiplexing
    \ipro Flexible in address space
    \ipro Useful functionality
        \begin{itemize}
            \item Firewall
            \item Privacy (hides network structure)
        \end{itemize}
\end{itemize}

\subsection{IPv6}
\begin{itemize}
    \item Replacement for IPv4
    \item IPv4 address space is exhausted
    \item Proposed in 1994
    \item $128$ bits addresses
    \item Denoted in $8$ groups of $4$ hex digits ($16$ bits)
    \item Omit leading zeros within each group
    \item Omit (at most one) group of all zeros
        \begin{itemize}
            \item Else unknown how many omitted (or something)
        \end{itemize}
    \icon Not backwards compatible
        \begin{itemize}
            \item Specific routers required
        \end{itemize}
    \item Header
        \begin{itemize}
            \item Simpler
            \item $32$ bits per line
                \ides{Version:} 6
                \ides{Traffic Class:} Former type of service
                \ides{Flow label:} Allow router to identify a flow
                \begin{itemize}
                    \item Packets of same flow stay together (same path)
                \end{itemize}
                \ides{Payload Length:} Former total length
                \ides{Next hdr (header):} Former protocol
                \ides{Hop Limit:} Old TTL
                \ides{Source Address:} $128$ bits
                \ides{Destination Address:} $128$ bits
            \ides{Flow:} Allows router to identify a flow
            \ides{Payload Length:} Former total length
            \item Removed
                \begin{itemize}
                    \ides{IHL:} 
                        \begin{itemize}
                            \ides{Streamlined Header Pressing:} No option fields
                                \begin{itemize}
                                    \item Header has fixed $40$ bytes size
                                    \item Allows for faster processing
                                \end{itemize}
                        \end{itemize}
                    \item Fragmentation related fields were removed
                        \begin{itemize}
                            \item Host must ensure that MTU is not exceeded
                        \end{itemize}
                    \item Header checksum is removed
                        \begin{itemize}
                            \item Rely on transport layer checksum
                        \end{itemize}
                \end{itemize}
        \end{itemize}
\end{itemize}
