% ! TEX root = ./main.tex

\section{Grundideen}
\begin{itemize}
    \item Man fasst die Daten $x_1, \dots, x_n$ auf als Realisierungen $X_1(\omega), \dots, X_n(\omega)$ von ZV $X_1, \dots, X_n$, und sucht dann Aussagen über die Verteilung von $X_1, \dots, X_n$.
        \begin{itemize}
            \item $x_1, \dots, x_n$ sind Daten (in der Regel Zahlen)
            \item $X_1, \dots, X_n$ sind der generierende Mechanismus (ZV)
            \ides{Strichprobe:} $X_1, \dots, X_n$ nennt man oft auch Strichprobe, $n$ heisst \textbf{Stichprobenumfang}
        \end{itemize}
\end{itemize}
\todo{Definiere parameter theta (generelle Idee)}
