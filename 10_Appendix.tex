%! TEX root = ./main.tex

\section{Appendix}
\begin{itemize}
    \ides{This section is not exam relevant}
\end{itemize}

\subsection{Basic Tools}
\begin{itemize}
    \ides{ping}
        \begin{itemize}
            \item \verb+ping some_address+
            \item Check connectivity and measure RTT
            \item \verb+-c some_num+ limits the number of pings
            \item Sends ARP packets
        \end{itemize}
    \ides{dig}
        \begin{itemize}
            \item \verb+dig some_domain+
            \item Queries DNS
            \item The anders is in the `ANSWER SECTION` on the right
            \item The `SERVER` at the bottom is who answered (DNS server address I guess)
        \end{itemize}
    \ides{ip}
        \begin{itemize}
            \item \verb+ip addr+
            \item Gives Local IP information
        \end{itemize}
    \ides{iftop}
        \begin{itemize}
            \item \verb+iftop -i some_interface+
            \item Displays all traffic over the given interface
            \item \verb+Tx:+ stands for send
            \item \verb+Rx:+ stands for receive
            \item Columns on the right display average over $2, 10, 40$ seconds
        \end{itemize}
    \ides{Wireshark}
        \begin{itemize}
            \item Monitor network
            \item Has UI
        \end{itemize}
    \ides{tcpdump}
        \begin{itemize}
            \item Similar to wireshark
            \item Cli application
            \ides{-w:} write output to file
            \ides{-c N:} stop after N packages
            \ides{-i some\_interface:} listen on specified interface
            \ides{-n:} do not convert names to ip
        \end{itemize}
    \ides{netcat}
        \begin{itemize}
            \item Allows for quick communication between two hosts
            \item Send text
                \begin{itemize}
                    \item Server: \verb+netcat -l 4444+
                    \item Client: \verb+netcat localhost 4444+
                \end{itemize}
            \item Send file
                \begin{itemize}
                    \item Server: \verb+netcat -l 4444 >  received.txt+
                    \item Client: \verb+netcat -l 4444 <  send.txt+
                \end{itemize}
        \end{itemize}
    \ides{ss}
        \begin{itemize}
            \item Displays sockets
            \item \verb+ss -at+: show TCP services
            \item \verb+ss -au+: show UDP services
            \item \verb+-n+: show port instead of service name
        \end{itemize}
\end{itemize}

\subsection{DNS Stuff}
\begin{itemize}
    \item Figure out where data comes from
        \begin{itemize}
            \item I.e. if from content delivery or directly
            \item \verb+curl -Ls -o /dev/null -w %{url_effective} some_resource+
        \end{itemize}
    \item Get IP we are actually talking to
        \begin{itemize}
            \item \verb+sudo hping3 -c 1 -S --tcp-mss 1460 --tcp-timestamp -L 0 -w 1024 -s 39826 -p 443 microsoft.com+
        \end{itemize}
    \item Reverse DNS Query
        \begin{itemize}
            \item \verb|dig +noall +answer -x some_ip|
        \end{itemize}
    \item Reverse DNS Query
        \begin{itemize}
            \item \verb+whois some_ip+
        \end{itemize}
\end{itemize}

\subsection{Congestion Control}
\begin{itemize}
    \item \verb+/lib/modules/$(uname -r)/kernel/net/ip4+ contains all available congestion control algorithms
    \item \verb+/proc/sys/net/ipv4/tcp_congestion_control+ contains the current algorithm
    \item \verb+tcp_bbr+ is a good algorithm
    \item \verb+iperf+ or \verb+iperf3+ can be used to speed measurements
\end{itemize}
