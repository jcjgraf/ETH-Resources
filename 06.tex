% ! TEX root = ./main.tex

\section{Schätzer}
\begin{itemize}
    \item Seien $X_1, \dots, X_n$ eine Stichprobe, für die wir ein Modell suchen. Wir haben also einen Parameterraum $\Theta$ und für jedes $\theta \in \Theta$ einen Wahrscheinlichkeitsraum $(\Omega, \mc{F}, P_\theta)$. Meistens ist $\Omega \in \R^m$, und wir suchen dann für die Parameter $\theta_1, \dots, \theta_n$ Schätzer $T_1, \dots, T_n$ aufgrund unsere Stichprobe. Solche Schätzer sind Zufallsvariablen der Form $T_j = t_j(X_1, \dots, X_n)$, wobei wir die Schätzfunkton $t_j: \R^n \to \R$ noch geeignet wählen/finden müssen. Einsetzen von Daten $x_i = X_i(\omega), i = 1, \dots, n$ liefert dann Schätzwert $T_j(\omega) = t_j(x_1, \dots, x_n)$ für $\omega_j, j = 1, \dots, m$. Der Kürze halber schreiben wir oft auch $T = (T_1, \dots, T_m)$ und $\theta = (\theta_1, \dots, \theta_m)$.
    \item Schätzer: ZV, Schätzwert: Zahl
\end{itemize}

\subsection{Grundbegriffe}
\begin{itemize}
    \ides{Erwartungstreu:} Schätzer $T$ ist erwartungstreu für $\theta$, falls gilt $E_\theta[T] = \theta$
        \begin{itemize}
            \ides{Bias:} $E_\theta[T] - \theta$
        \end{itemize}
    \ides{Mittlere Quadratische Schätzfehler (MSE):} Definiert als $\text{MSE}_\theta[T] := E_\theta[(T - \theta)^2]$
        \begin{itemize}
            \item $\text{MSE}_\theta[T] = E_\theta[(T - \theta)^2] = \text{Var}_\theta[T] + (E_\theta[T] - \theta)^2$
        \end{itemize}
    \ides{Konsistent:} Folge von Schätzern $T^{(n)}, n \in \R,$ heisst konsistent für $\theta$, falls $T^{(n)}$ für $n \to \infty$ gegen $\theta$ konvergiert. D.h. $\forall \theta \in Theta$ gilt: $\lim_{n \to \infty} P_\theta[|T^{(n)} - \theta| > \epsilon] = 0$
\end{itemize}

\subsection{Maximum-Likelihood-Methode (ML-Methode)}
\begin{itemize}
    \ides{Likelihood-Funktion:} $L(x_1, \dots, x_n; \theta) :=
\begin{cases}
    p(x_1, \dots, x_n; \theta &\text{im diskreten Fall}\\
    f(x_1, \dots, x_n; \theta &\text{im stetigen Fall}
\end{cases}$
        \begin{itemize}
            \ides{Log-Likelihood:} Often brauchen wir $\log L(x_1, \dots, x_n;\theta)$ damit, wenn ZV i.i.d. sind, rechnen wir mit Summen und nicht Produkten.
        \end{itemize}
        \ides{Maximum-Likelihood-Schätzer (ML-Schätzer) $T_{ML}$} für $\theta$ maximiert $\theta \mapsto L(X_1, \dots, X_n;\theta)$ als Funktion von $\theta$.
        \begin{itemize}
            \item Statt zu maximieren sucht man meistens nur Nullstellen der Ableitung nach $\theta$.
            \item Falls Funktion nicht differenzierbar muss man anders vorgehen
        \end{itemize}
\end{itemize}

\subsection{Momentenschätzer (MM)}
\begin{itemize}
    \item Seinen $X_1, \dots, X_n$ i.i.d. Für eine Funktion $h: \Theta \to \R^d$ wollen wir die $d$ Grössen $h_j(\theta), j = 1, \dots, d$ schätzen. Um MM $h(\theta)$ zu bestimmen:
        \begin{itemize}
            \item[1)] Für $j = 1, \dots, d$ berechnen wir in jedem Modell $P_\theta$ das $j$-te Moment $m_j(\theta) = E_\theta[X^j]$ als Funktion von $\theta$.
            \item[2)] Für $j = 1, \dots, d$ definieren wir das $j$-Te \textit{empirische Mittel} als $\overset{\sim}{m_j}(x_1, \dots, x_n) := \frac{1}{n} \sum_{i=1}^{n} x_i^j$
            \item[3)] Betrachten System von $d$ Gleichungen $\overset{\sim}{m_j}(x_1, \dots, x_n) = m_j(\theta), j = 1, \dots, d$ und lösen Gleichungssystem für $d$ Unbekannten $h_1(\theta), \dots h_d(\theta)$.
            \item Falls eindeutige Lösung existiert, nennen wir diese $t_{MM}(x_1, \dots, x_n)$, $t_{MM}: \R^n \to \R^d$.
        \end{itemize}
\end{itemize}

\subsection{Verteilungsaussagen}
\begin{itemize}
    \item Verteilungsaussagen über Schätzer sind nicht einfach.
    \item Falls Schätzer von der Form $\sum_{i=1}^{n} Y_i$, wobei $Y_i$ i.i.d., kann man ZGS benutzen. Einfaches Bsp: Falls $T = \frac{1}{n} \sum_{i=1}^{n} Y_i$ folgt für grosse $n$, dass $\sum_{i=1}^{n} Y_i \overset{\text{approx}}{\sim} \mc{N}(n E_\theta[Y_i], n \text{Var}_\theta[Y_i])$
    \item Seien $X_1, \dots, X_n$ i.i.d. $\sim \mc{N}(\mu, \sigma^2)$ Dann gilt:
        \begin{itemize}
            \item[1)] $\overline{X}_n \sim \mc{N}(\mu, \frac{1}{n} \sigma^2)$, und $\frac{\overline{X}_n - \mu}{\frac{\sigma}{\sqrt{n}}} \sim \mc{N}(0, 1)$
            \item[2)] $\frac{n - 1}{\sigma^2}S^2 = \frac{1}{\sigma^2}\sum_{i=1}^{n} (X_i - \overline{X}_n)^2 \sim \mc{X}^2$ mit $n - 1$ Freiheitsgraden
            \item[3)] $\overline{X}_n$ und $S^2$ sind unabhängig
            \item[4)] Der Quotient $\frac{\overline{X}_n - \mu}{\frac{S}{\sqrt{n}}} = \frac{\frac{\overline{X}_n - \mu}{\frac{\sigma}{\sqrt{n}}}}{\frac{S}{\sigma}} = \frac{\frac{\overline{X}_n - \mu}{\frac{\sigma}{\sqrt{n}}}}{\sqrt{\frac{1}{n - 1}\frac{n - 1}{\sigma^2}S^2}}$ ist $t$-verteilt mit $n - 1$ Freiheitsgraden
        \end{itemize}
\end{itemize}

\subsubsection{$\mc{X}^2$-Verteilung mit $n$ Freiheitsgraden}
\begin{itemize}
    \item Diese gehört zu einer stetigen ZV $Y$ mit DF $f_Y(y) = \frac{1}{2^{\frac{n}{2}} \Gamma(\frac{n}{2})}y^{\frac{n}{2} - 1}e^{-\frac{1}{2}y}, \ y \ge 0$.
    \item Entsteht wie folgt: Sind die ZV $X_1, \dots, X_n$ i.i.d. und $\sim \mc{N}(0, 1)$, so ist die Summe $Y := \sum_{i=1}^{n} X_i^2 \sim \mc{X}_n^2$
    \item Für $n = 2$ erhält man $\sim Exp(\frac{1}{2})$
\end{itemize}

\subsubsection{$t$-Verteilung mit $n$ Freiheitsgraden}
\begin{itemize}
    \item Diese gehört zu einer stetigen ZV $Z$ mit DF $f_Z(z) = \frac{\Gamma(\frac{n + 1}{2})}{\sqrt{n \pi} \Gamma(\frac{n}{2})} \left( 1 + \frac{z^2}{n}\right)^{- \frac{n + 1}{2}}, \ z \in \R$
    \item Entsteht wie folgt: Sind $X$ und $Y$ unabhängig mit $X \sim \mc{N}(0, 1)$ und $Y \sim \mc{X}_n^2$ so ist $Z := \frac{X}{\sqrt{\frac{1}{n} Y}} \ t$-verteilt mit $n$ Freiheitsgraden
    \item Für $n = 1$ ist das eine Cauchy-Verteilung
    \item Für $n \to \infty$ erhält man $\sim \mc{N}(0, 1)$
\end{itemize}
