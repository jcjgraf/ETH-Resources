%! TEX root = ./main.tex

\section{Graphical Modelling}
\begin{itemize}
    \item Models an application
    \item Graphical way to represent entities and their relation
    \item Consists of three steps
        \begin{itemize}
            \ides{Conceptual Modeling:} Capture of domains to be represented
                \begin{itemize}
                    \item Create diagram from ``real world''
                    \item We consider Entity Relation (ER) Model in this course
                    \item Specified all DB instances that are valid/allowed in our application
                \end{itemize}
            \ides{Logical Modeling:} Map concepts to a concrete logical representation
                \begin{itemize}
                    \item Convert diagram to table schema
                \end{itemize}
            \ides{Physical Modeling:} Implementation in Hardware
                \begin{itemize}
                    \item Convert table to bits
                \end{itemize}
        \end{itemize}
\end{itemize}

\subsection{Conceptual Modelling (ER-Diagram)}
\begin{itemize}
    \ides{Formal Semantics}
        \begin{itemize}
            \item Diagram defines valid DB instances
            \item All values can take $\mc{D} = \mc{B} \cup \Delta$
                \begin{itemize}
                    \ides{$\mathbf{\mc{B}}$:} Concrete values
                        \begin{itemize}
                            \item Int, String, Float, etc
                        \end{itemize}
                    \ides{$\mathbf{\Delta}$:} Abstract values
                        \begin{itemize}
                            \item Correspond to an entity
                        \end{itemize}
                \end{itemize}
            \ides{Entity Set $E$:} $1$-ary Predicate $E(x)$
                \begin{itemize}
                    \item $E(x) = \text{True}$ if $x$ is of Entity Type $E$
                    \item $E^\mc{J} \subseteq \Delta$
                \end{itemize}
            \ides{Attribute $A$:} Binary Predicate $A(x, y)$
                \begin{itemize}
                    \item $A(e, a) = \text{True}$ if $e$ has attribute $a$
                    \item $A^\mc{J} \subseteq \Delta \times \mc{B}$
                \end{itemize}
            \ides{$n$-ary Relation $R$:} $n$-ary Predicate $R(x_1, \dots, x_n)$
                \begin{itemize}
                    \item $R(x_1, \dots, x_n) = \text{True}$ if $(x_1, \dots, x_n)$ participate in $R$
                    \item $R^\mc{J} \subseteq \Delta^n$
                \end{itemize}
            \item Each subgraph introduces a first-order logic sentence
            \item Entity $E_1$ and $E_2$ linked by relation $R$
                \begin{itemize}
                    \item $\forall x_1, x_2 \in \Delta. R(x_1, x_2) \implies E_1(x_1) \wedge E_2(x_2)$
                \end{itemize}
            \item Entity $E$ with attribute $A$
                \begin{itemize}
                    \item $\forall x, E(x) \implies \underbrace{E^{=1}}_{\text{uniquely exists}} y. A(x, y) \wedge y \in \mc{B}$
                \end{itemize}
        \end{itemize}
    \ides{Building Blocks}
        \begin{description}
            \ides{Entity:} Instance of an entity set which is distinguishable from other instances of the same set
            \ides{Entity Set:} Set of entities of the same "type"
                \begin{itemize}
                    \item Rectangular box
                \end{itemize}
            \ides{Attributes:} Properties of a certain entity set
                \begin{itemize}
                    \item Round box
                \end{itemize}
            \ides{Relationships:} Connection among $\ge 2$ entity sets
                \begin{itemize}
                    \item Rhombus box
                    \ides{Roles}
                        \begin{itemize}
                            \item Each entity set can have a role in a relation
                            \item Label lines by the role the entity set is
                        \end{itemize}
                \end{itemize}
            \ides{Key:} Minimal set of attributes which uniquely identify an entity in the entity set
                \begin{description}
                    \ides{Candidate Key:} All possible sets of keys
                    \ides{Primary Key:} One selected key
                        \begin{itemize}
                            \item Every entity set must have one
                            \item Underlined
                        \end{itemize}
                \end{description}
        \end{description}
    \ides{Cardinality}
        \begin{itemize}
            \item Two main notations
            \ides{N/M-Notation}
            \begin{description}
                \ides{One to One (1/1):}
                    \begin{itemize}
                        \item $A$ is in a one to one relationship with $B$ if:
                            \begin{itemize}
                                \item $1 A$ entity can only have one relation with a $B$ entity and
                                \item $1 B$ entity can only have one relation with an $A$ entity
                            \end{itemize}
                    \end{itemize}
                \ides{One to Many (1/N):}
                    \begin{itemize}
                        \item $A$ is in a one to may relation with $B$ if:
                            \begin{itemize}
                                \item $1 A$ can have relationships with multiple $B$ entities and
                                \item $1 B$ can only have one relation with an $A$ entity
                            \end{itemize}
                    \end{itemize}
                \ides{Many to One (N/1):}
                \ides{Many to Many (N/M):}
                    \begin{itemize}
                        \item $A$ is in a many to many relation with $B$ if:
                            \begin{itemize}
                                \item $1 A$ entity can have relationships with multiple $B$ entities and
                                \item $1 B$ entity can have relationships with multiple $A$ entities
                            \end{itemize}
                    \end{itemize}
            \end{description}
            \ides{(min, max)-Notation}
                \begin{itemize}
                    \item For a relation we give the min and the max value of relations one entity can have
                    \item Stronger than N/M-notation
                    \item \verb+*+ means infinity
                    \item (min, max) is written in opposite was to N/M
                \end{itemize}
        \end{itemize}
    \ides{Weak Entity}
        \begin{itemize}
            \item Some entity relation depends on other entity
                \begin{itemize}
                    \item I.e. it is not unique by itself
                    \item Can only be uniquely identified with the main entity
                \end{itemize}
            \item \textit{Weak} entity is the one which depends on another
            \item Indicated by dotted underline
            \item Is a $1/M$ relationship
        \end{itemize}
    \ides{Generalisation}
        \begin{description}
            \item Represent that a entity set is is an instance of another entity set
            \item Entity $A$ \textbf{is\_a} entity of $B$
                \begin{itemize}
                    \item Draw an arrow from $A$ to $B$
                    \item $A$ shares $B$s attributes and primary key
                \end{itemize}
        \end{description}
    \item There are many other flavours of ER
    \ides{Design Principles}
        \begin{itemize}
            \item Model should reflect the application we want to build
            \item Avoid redundancy
            \item Keep it as simple as possible; less entities is better
            \item Entity if the concept has more than one relationship
            \item Attribute if the concept has only one 1:1 relationship
            \item Models are large, partition it
        \end{itemize}
\end{itemize}

\subsection{Logical Modelling}
\begin{itemize}
    \item Take ER-model and convert to relational model
    \item Some constraints get lost
    \ides{Steps}
        \begin{enumerate}
            \ides{Entity Sets:} Become relations
            \ides{Attributes:} Become attributes of the relation
            \ides{Relationship:}
                \begin{description}
                    \ides{Without Cardinality Constrait (or N:M):}
                        \begin{itemize}
                            \item Become relation containing the attributes of all participating relations
                            \item The primary key of the relation are all the primary keys together
                        \end{itemize}
                    \ides{With Cardinality Constraints:}
                        \begin{itemize}
                            \item Very tricky
                            \item Become relation containing the attributes of all participating relations
                            \item The primary key of the relation are the keys of the entities on the \textit{many} side. Or the relation gets merged into the table on the \textit{many} side.
                        \end{itemize}
                    \ides{Role:} Can be used to distinguish columns with the same entity type.
                        \begin{itemize}
                            \item Done by renaming the two columns appropriately
                        \end{itemize}
                \end{description}
            \ides{Weak Entity:} 
                \begin{itemize}
                    \item Can be omitted
                    \item The week entity is modeled as a relation on its own with the primary key of the main relation and its own key
                \end{itemize}
            \ides{Generalisation:}
                \begin{itemize}
                    \item Two ways to represent this
                    \item Better way depends on application
                    \item[1)] \textit{Child} has its own relation and the \textit{Parent} relation
                    \item[2)] Each \textit{Child} is a full blown relation containing all keys of the \textit{parent} relation
                        \begin{itemize}
                            \icon Lot of redundant data if entity is multiple child and parent at the same time
                            \icon Cannot constraint that entity is only on of them
                        \end{itemize}
                \end{itemize}
            \ides{Rezept}
                \begin{itemize}
                    \item \todo{Rezept} 
                        %Convert:
                        %
                        %Model Entity:
                        %Model Relation:
                        %Merge: if relation has shared primary key (aka. Denormalize)
                        %
                        %with extra key, indicate other possibilities for primary keys
                        %
                        %specialisatuib: specialisation get key from paret
                \end{itemize}
        \end{enumerate}
    \item Can be done (Semi-) automatically
\end{itemize}
