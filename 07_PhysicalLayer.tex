%! TEX root = ./main.tex

\section{Physical Layer}
\begin{itemize}
    \item How to send digital signals in an analog way
    \item Convert to digital bits
\end{itemize}

\subsection{Link Model}
\begin{itemize}
    \item Abstraction
    \item Physical channel is characterized by:
        \begin{itemize}
            \ides{Rate R} in bps
                \begin{itemize}
                    \item Also called \textbf{bandwidth}, \textbf{capacity} or \textbf{speed}
                \end{itemize}
            \ides{Delay D} in seconds
                \begin{itemize}
                    \item Depends on message length and media length
                    \item Also called \textbf{latency}
                \end{itemize}
            \item Other properties like \textit{is channel broadcast} and its error rate
        \end{itemize}
    \ides{Latency L:} Delay to send a message over a link
        \begin{itemize}
            \item Composed of two parts
                \begin{itemize}
                    \ides{Transmission Delay T:} Time to put $M$-bit message on the wire \begin{itemize}
                            \item $T = \frac{M \text{[bits]}}{R \text{[bits/sec]}} = \frac{M}{R} \text{[sec]}$
                            \item Dominant today
                        \end{itemize}
                    \ides{Propagation Delay P:} Time for bits to propagate across the wire
                        \begin{itemize}
                            \item $P = \frac{\text{Length}}{\text{speed of signal}} = \frac{\text{Length}}{\frac{2}{3}c} = D \text{[sec]}$
                        \end{itemize}
                \end{itemize}
            \item $L = \frac{M}{R} + D$
            \item High of slow rate or long link
        \end{itemize}
    \ides{Bandwidth-Delay (BD) Product:} Amount of data in flight
        \begin{itemize}
            \item $\text{BD} = R \cdot D$
            \item Measured in bits or messages
        \end{itemize}
\end{itemize}

\subsection{Types of Media}
\begin{itemize}
    \ides{Wires - Twisted Pair}
        \begin{itemize}
            \item Very common
            \item Twists can reduce radiated signal or reduce effect of external interference
                \begin{itemize}
                    \item Subtracting the signal of both wires cancels out interference \todo{Or something like that}
                \end{itemize}
            \ipro Cheap
        \end{itemize}
    \ides{Wire - Coaxial Cable}
        \begin{itemize}
            \item Also common
            \item Copper core shielded with multiple layers of insulation and protection
            \ipro Better performance due to better shielding
        \end{itemize}
    \ides{Wire - Fiber}
        \begin{itemize}
            \item Strands of glass inside multiple layer of protection
            \item Transmit light captured inside tue to total reflection
            \ipro Enormous bandwidth
            \ipro Usable for long distances
            \item Two variants
                \begin{itemize}
                    \ides{Multi-Mode:} Send multiple wavelength of light simultaneously
                        \begin{itemize}
                            \item For shorter links
                            \item Cheaper
                        \end{itemize}
                    \ides{Single-Mode:} Send a single wavelength
                        \begin{itemize}
                            \item Up to $\sim 100$km
                        \end{itemize}
                \end{itemize}
        \end{itemize}
    \ides{Wireless}
        \begin{itemize}
            \item Sender radiates signal over a region
            \item Send to potentially many receivers
            \item Nearby signals may interfere at a receiver
                \begin{itemize}
                    \item Coordination required
                \end{itemize}
            \ides{Industry Science Medicine (ISM):} Bandwidth for everyone but only locally
                \begin{itemize}
                    \item Used by WiFi, Bluetooth etc.
                \end{itemize}
        \end{itemize}
\end{itemize}

\subsection{Signals}
\begin{itemize}
    \item Analog signals encode digital bits
    \item A signal over time can be represented by its frequency components (Fourier analysis)
    \item More spectrum available the more data one can send
    \item Different frequencies have different properties (e.g. influence by rain etc.)
     \item Fewer frequencies $\implies$ less bandwidth $\implies$ degraded signal
        \begin{itemize}
            \item High frequencies carry a great deal of the data
        \end{itemize}
    \ides{Bandwith:} information carrying capacity
        \begin{itemize}
            \item Measured in bits/sec
            \item In electrical engineering it is measured in Hz
        \end{itemize}
    \ides{Signal on Wire}
        \begin{itemize}
            \item It is delayed since it propagates at $\frac{2}{3}c$
            \item It is attenuated
                \begin{itemize}
                    \item Especially signals above a cutoff are highly attenuated
                \end{itemize}
            \item Noise is added to the signal
        \end{itemize}
    \ides{Signal on Fiber}
        \begin{itemize}
            \item Light propagates with very low loss in three very wide frequency bands
                \begin{itemize}
                    \item Attenuation is low at these frequencies
                \end{itemize}
        \end{itemize}
    \ides{Signal over Wireless}
        \begin{itemize}
            \item Travel at speed of light
            \item Attenuation increases quadratically over distance from the sender
            \item Interference at the receiver if multiple senders use the same frequencies
                \begin{itemize}
                    \ides{Spatial Reuse:} Interference leads to notion of spatial reuse
                \end{itemize}
            \ides{Multipath:} Signals bounce of objects and interferes with itself
        \end{itemize}
\end{itemize}

\subsection{Modulation}
\begin{itemize}
    \item How to represents bits as signals?
    \item Multiple methods
    \ides{Baseband Modulation}
        \begin{itemize}
            \item Signal is sent directly onto the wire
                \begin{itemize}
                    \item Not working for fiber and wireless
                \end{itemize}
            \ides{Non-Return to Zero (NRZ)}
                \begin{itemize}
                    \item High voltage ($+1$ V) represents $1$, low voltage ($-V$) represents $0$
                    \item Average voltage is $0$ and therefore no current is flowing
                    \icon Clock recovery is hard
                \end{itemize}
            \ides{Clock Recovery}
                \begin{itemize}
                    \item Hard to count how many consecutive $1$s (or $0$s) were transmitted if the signal does not change for a longer time
                \end{itemize}
            \ides{Non-Return to Zero Inverse (NRZI)}
                \begin{itemize}
                    \item Invert signal level on a $1$
                    \ipro Breaks long runs of $1$
                    \icon Does not break long runs of $0$
                \end{itemize}
            \ides{Manchester coding:}
                \begin{itemize}
                    \item To send a $1$ switch from $0$ to $1$
                    \item To send a $0$ switch from $1$ to $0$
                    \item Value between clock holds the actual value
                    \ipro No clock recovery
                    \item There is also inverse manchester encoding
                \end{itemize}
            \ides{4B5B}
                \begin{itemize}
                    \item Translate $4$ bits pattern into specific $5$ bits pattern
                        \begin{itemize}
                            \item Done according to some translation dictionary
                        \end{itemize}
                    \item Has at most $3$ consecutive $0$s
                    \item Has at most $8$ consecutive $1$s
                \end{itemize}
            \ides{More Signal Levels}
                \begin{itemize}
                    \item All introduced methods use $2$ levels
                    \item By increasing it to e.g. $4$ we could use $2$ bits per symbol \todo{What does it mean?}\todo{What are the cons?}
                \end{itemize}
        \end{itemize}
    \ides{Passband Modulation}
        \begin{itemize}
            \item Take a baseband modulation and apply passband modulation on top of it \todo{Is this correct?}
            \item Carries signal by modulating a carrier
                \begin{itemize}
                    \ides{Carrier:} Some base signal oscillating at a desired frequency
                    \item Modify this carrier to indicate $1$s or $0$s
                \end{itemize}
            \item Can module different things
                \begin{itemize}
                    \ides{Amplitude Shift Keying:} Shift amplitude when sending $0$
                        \begin{itemize}
                            \item I.e. stop oscillation of carrier while sending $0$
                        \end{itemize}
                    \ides{Frequency Shift Keying:} Increase/decrease frequency depending if sending $0$ or $1$
                    \ides{Phase Shift Keying:}  Uncontinuous at change of single
                \end{itemize}
            \item Can also combine multiple modulations
        \end{itemize}
\end{itemize}

\subsection{Limits}
\begin{itemize}
    \item Channel is characterized by bandwidth $B$, signal strength $S$ and noise $N$
        \begin{itemize}
            \item $B$ limits the rate of transmissions
            \item $S$ and $N$ limit how many signals levels we can distinguish
        \end{itemize}
    \ides{Nyquist Limit:} Maximum symbol rate is $2B$
        \begin{itemize}
            \item If there are $V$ signal levels the maximum bit rate is $R = 2B \log_2 V \ \text{[bits/s]}$
            \item Ignores noise
        \end{itemize}
    \ides{Shannon Capacity}
        \begin{itemize}
            \item Considers noise
                \ides{Signal-to-Noise Ration (SNR):} Limits the number of levels we can distinguish
                    \begin{itemize}
                        \item Measure in decibels
                    \end{itemize}
            \item SNR given as: $\text{SNR}_{dB} = 10 \log_{10} (S/N) \ \text{[dB]}$
            \ides{Capacity C:} is the maximum information carrying rate of the channel
                \begin{itemize}
                    \item $C = B \log_2 (1 + \frac{S}{N}) \ \text{[bits/sec]}$
                \end{itemize}
        \end{itemize}
    \ides{Conclusion Wired/Wireless}
        \begin{itemize}
            \ides{Wires and Fiber:} For fixed R, engineers link to have required SNR and B
            \ides{Wireless:} For fixed B, SNR varies greatly
                \begin{itemize}
                    \item Designe R for worst SNR is wasteful
                    \item Must adopt R dynamically
                \end{itemize}
        \end{itemize}
    \ides{Digital Subscriber Line (DSL)}
        \begin{itemize}
            \item Used passband modulation
                \begin{itemize}
                    \item Modulate amplitude and phase
                \end{itemize}
            \item Uses separate bands for up- and downstream
            \item On high SNR, use up to $15$ bits/symbol
            \item On low SNR, use $1$ bit/symbol
        \end{itemize}
\end{itemize}
