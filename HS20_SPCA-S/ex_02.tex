%! TEX root=./main.tex

\lecture{Ex 2}{Week: 2}{}

\subsubsection{Header Files and Modules}

\paragraph{Definition vs. Declaration}
There is a difference between declaration and definition:
\begin{itemize}
    \item Declaration gives the signature of the function / variable.
    \item Definition gives the code /storage space for variables.
\end{itemize}

\paragraph{Header Files}
In order to share a function among different source file, we have to place the function declaration into the header file.

Using \code{\#include "mymodule.h"} the function can be included into other source files. Note the difference between the prior include and e.g. \code{\#include <stdio.h>}. The first one looks for header files in the same directory while the second one looks for system header files.

Header files are considered globally accessible. So one should only put there what is actually necessary like e.g. public interfaces.

Header files are actually copy-pasted into the source file during compilation. To prevent multiple imports (e.g. for nested modules), a guard structure should be used.

\paragraph{Compilation}
To compile programs which consists of multiple source files with their associated header files, we have to list all required source (not header) files for gcc.

\subsubsection{Makefile}
Makefiles are used to automate compilation. It it like defining functions for or complex compile commands. The general structure is

\begin{verbatim}
targets: prerequisites
    command
    command
\end{verbatim}

where \verb+target+ is the name of the function, in \verb+prerequisits+ we list all required source and header files and beneath we place the command(s) required for compilation.

We can also define rules for different tasks than only compiling like for build cleanup.

\subsubsection{gcc Flags}
Useful flags:

\begin{description}
    \item[Werror:] Make all warnings into errors.
    \item[Wpedantic:] Issue all the warnings demanded by strict ISO C and reject all programs that use forbidden extensions.
    \item[Wall:] Enable different warnings about questionable code.
    \item[fsanitize=address:] Instrument code to detect heap management errors.
    \item[fsanitize=undefined:] Instrument code to detect undefined behaviour at runtime.
    \item[fsanitize-address-use-after-scope:] Instrument code to detect addresses used when already out of scope.
    \item[fstack-protector-all:] Instrument code to detect buffer overflow on the stack.
\end{description}

In practice we should use something like:

\verb+gcc <FILES> -Wall -Wpedantic -Wextra -Werror -std=c99 -Wmissing-prototypes+
