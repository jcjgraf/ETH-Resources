%! TEX root = ./main.tex

\section{Case Study}
\subsection{Overview Interpreter}
\begin{itemize}
    \item Has three basic steps:
    \ides{Read}
        \begin{itemize}
            \ides{Input:} Text
            \ides{Phases:}
                \begin{itemize}
                    \ides{Lexical Analysis}
                        \begin{itemize}
                            \item Convert source code to tokens
                                \begin{itemize}
                                    \item I.e. tell for each groups of symbols what they are
                                \end{itemize}
                            \ides{Tokens:} Identifier (variables), arithmetic symbols, assignment symbol, numbers, etc.
                            \item White-spaces and comments are removed
                        \end{itemize}
                    \ides{Parsing}
                        \begin{itemize}
                            \item Build abstract syntax tree
                            \item Syntax is specified by a given grammar
                                \begin{itemize}
                                    \item I.e. a data type in Haskell
                                \end{itemize}
                        \end{itemize}
                    \ides{Outer Phases}
                        \begin{itemize}
                            \item Depending on the applications, further phases may come now
                            \item Things like type conversion, type checking, dependency analysis, etc.
                        \end{itemize}
                \end{itemize}
            \ides{Output:} Abstract Syntax Tree
            \item Lexical analysis and parsing is required for all systems
        \end{itemize}
    \ides{Evaluate}
        \begin{itemize}
            \ides{Input:} Abstract Syntax Tree
            \ides{Semantic Interpretation}
            \ides{Output:} Abstract Syntax Tree
        \end{itemize}
    \ides{Print}
        \begin{itemize}
            \ides{Input:} Abstract Syntax Tree
            \ides{Pretty Print Output}
            \ides{Output:} Text
        \end{itemize}
\end{itemize}

\subsection{Overview Parser}
\begin{itemize}
    \item Parser is a function
    \ides{Input:} String
    \ides{Output:} Element of type \verb+a+
        \begin{itemize}
            \item Typically \verb+a+ is some data type
        \end{itemize}
    \item A parser may not necessarily parse the whole input
        \begin{itemize}
            \ides{Combinatory Parsing}
            \item There is a remainder
            \item \verb+res, rem+
            \item Remainder may be parsed by a different parser
        \end{itemize}
    \item A parser may try to produce different results for the same input
        \begin{itemize}
            \item Store \verb+(res_i, rem_i)+ in a list
            \item If \verb+rem_i = ""+ the parse is complete
            \item \verb+data Parser a = Prs (String -> [(a, String)]+
        \end{itemize}
    \ides{Application}
        \begin{itemize}
            \item
\begin{verbatim}
parse :: Parser a -> String -> [(a, String)]
parse (Prs p) inp = p inp
\end{verbatim}
        \end{itemize}
    \ides{Result of (first) Complete Parse}
        \begin{itemize}
            \item
\begin{verbatim}
completeParse :: Parser a -> String -> a
completeParse p inp
| result == [] = error "Parse unsuccessful"
| otherwise    = head results
where results  = [res | (res, "") <- parse p inp]
\end{verbatim}
        \end{itemize}
    \ides{Primitive Parsers}
        \begin{itemize}
            \item Server as a basic building block
            \ides{Failure:}
                \begin{itemize}
                    \item Fails trivially
                    \item \verb+[]+ signifies a unsuccessful parse
                    \item
\begin{verbatim}
failure :: Parser a
failure = Prs (\inp -> [])
\end{verbatim}
                    \item Ex.
\begin{verbatim}
$ parse failure "3+5"
[] :: [(a, String)]
\end{verbatim}
                \end{itemize}
            \ides{Return:}
                \begin{itemize}
                    \item Succeeds trivially
                    \item Without progress
                    \item
\begin{verbatim}
return :: a -> Parser a
return x = Prs (\inp -> [(x, inp)]
\end{verbatim}
                    \item Ex.
\begin{verbatim}
$ parse (return "foo") "3+5"
[("foo", "3+5")] :: [([Char], String)]
\end{verbatim}
                \end{itemize}
            \ides{Item:}
                \begin{itemize}
                    \item Succeeds trivially
                    \item With progress
                    \item
\begin{verbatim}
item :: Parser Char
item = Prs (\inpt -> case inp of
                    "" -> []
                    (x:xs) -> [(x, xs)])
\end{verbatim}
                    \item Ex.
\begin{verbatim}
$ parse item "3+5"
[('3', "+5")] :: [(Char, String)]
\end{verbatim}
                \end{itemize}
            \ides{Sat:}
                \begin{itemize}
                    \item Parse single char with property $p$
                    \item
\begin{verbatim}
sat :: (Char -> Bool) -> Parser Char
sat p = Prs (\inp -> case inp of
                    "" -> []
                    (x:xs) -> if p x then [(x,xs)] else [])
\end{verbatim}
                    \item Alternatively
\begin{verbatim}
sat :: (Char -> Bool) -> Parser Char
sat p = item >>= \x -> if p x then return x else failure
\end{verbatim}
                    \item Ex. isDigit
\begin{verbatim}
$ parse (sat (\x -> '0' <= x && x <= '9')) "3+5"
[('3', "+5")] :: [(Char, String)]
\end{verbatim}
                    \item Ex. isArithOp
\begin{verbatim}
$ parse (sat (\x -> x == '+' || x == '-')) "3+5"
[] :: [(Char, String)]
\end{verbatim}
                \end{itemize}
            \ides{Char:}
                \begin{itemize}
                    \item
\begin{verbatim}
char :: Char -> Parser Char
char x = sat (==x)
\end{verbatim}
                \end{itemize}
            \ides{String:}
                \begin{itemize}
                    \item
\begin{verbatim}
string :: String -> Parser Strin
string "" = return ""
string (x:xs) = char x >> string xs >> return (x:xs)
\end{verbatim}
                \end{itemize}
            \ides{Many:}
                \begin{itemize}
                    \item $0$ or more repetitions of $p$
                    \item
\begin{verbatim}
many :: Parser a -> Parser [a]
many p = many1 p ||| return []
\end{verbatim}
                \end{itemize}
            \ides{Many1:}
                \begin{itemize}
                    \item $1$ or more repetitions of $p$
                    \item
\begin{verbatim}
many1 :: Parser a -> Parser [a]
many1 p = p >>= \t -> many p >>= \ts -> return (t:ts)
\end{verbatim}
                \end{itemize}
            \ides{numPos:}
                \begin{itemize}
                    \item
\begin{verbatim}
numPos :: Parser Int
numPos = do ts <- many1 (sat isDigit)
    return (read ts)
\end{verbatim}
                \end{itemize}
            \ides{numNeg}
                \begin{itemize}
                    \item
\begin{verbatim}
numNeg :: Parser Int
numNeg = do char '-'
    t <- numPos
    return (-t)
\end{verbatim}
                \end{itemize}
            \ides{num}
                \begin{itemize}
                    \item
\begin{verbatim}
num :: Parser Int
num = numPos ||| numNeg
\end{verbatim}
                    \item
\begin{verbatim}
$ parse num "123"
[(123, ""), (12, "3"), (1, "23")]
$ parse num "-123"
[(-123, ""), (-12, "3"), (-1, "23")]
\end{verbatim}
                \end{itemize}
        \end{itemize}
    \ides{Combining Parsers}
        \begin{itemize}
            \ides{Mutual Selection:} Apply both parser and concatenate result
                \begin{itemize}
                    \item
\begin{verbatim}
(|||) :: Parser a -> Parser a -> Parser a
p ||| q = Prs (\s -> Parser p s ++ parser q s)
\end{verbatim}
                    \item Ex
\begin{verbatim}
$ parse (return '!' ||| sat isDigit) "3+5"
[('!', "3+5"), ('3', "+5")]
\end{verbatim}
                \end{itemize}
            \ides{Alternative Selection:} Apply second parser only of first fail
                \begin{itemize}
                    \item
\begin{verbatim}
(+++) :: Parser a -> Parser a -> Parser a
p +++ q = Prs (\s -> case parser p s of
                [] -> parser q s
                res -> res)
\end{verbatim}
                    \item Ex
\begin{verbatim}
$ parse (return '!' +++ sat isDigit) "3+5"
[('!', "3+5")]
\end{verbatim}
                \end{itemize}
            \ides{Sequencing:} Apply second parser on remainder of first parser. Return result and remainder of second parser
                \begin{itemize}
                    \item Result of first parser is lost
                    \item
\begin{verbatim}
(>>) :: Parser a -> Parser b -> Parser b
p >> q = Prs (\s -> [(u, s'') | (t, s') <- parse p s,
                        (u, s'') <- parse q s'])
\end{verbatim}
                    \item Ex
\begin{verbatim}
$ parse (sat isDigit >> sat (== '+')) "3+5"
[('+', "5")]
\end{verbatim}
                \end{itemize}
                \ides{Sequencing 2:} Apply second parser on result of first parser. Return combined result and remainder of second parser
                    \begin{itemize}
                        \item Second parser is a \textbf{parser generator}
                        \item
\begin{verbatim}
(>>=) :: Parser a -> (a -> Parser b) -> Parser b
p >>= g = Prs (\s -> [(u, s'') | (t, s') <- parser p s
                        (u, s'') <- parser (g t) s'])
\end{verbatim}
                        \item Can improve readability by using syntactic sugar
                            \begin{itemize}
                                \item
\begin{verbatim}
do t1 <- p1
t2 <- p2
...
tn <- pn
return (f t1 t2 ... tn)
\end{verbatim} for
\begin{verbatim}
p1 >>= \t1 ->
p2 >>= \t2 ->
...
pn >>= \tn ->
return (f t1 t2 ... tn)
\end{verbatim}
                                \item \verb+Parser+ must be an instance of \verb+Monad+
                            \end{itemize}
                        \item Ex
\begin{verbatim}
$ parse (sat isDigit >>=
        \t -> sat isDigit >>=
        \u -> return (t:u:[])) "31+5"
[("31", "+5")]
$ parse (sat isDigit >>=
        \t -> sat isDigit >>=
        \u -> return (t:u:[])) "3+5"
[]
\end{verbatim}
                    \end{itemize}
        \end{itemize}
\end{itemize}

\subsection{Arithmetic Interpretation}
\begin{itemize}
    \ides{Read}
        \begin{itemize}
            \ides{Grammar:} \verb/Expr ::= Int | Expr '+' Expr | Expr '-' Expr/
                \begin{itemize}
                    \item Resp. \verb+data Expr n Lit Int | Add Expr Expr | Sub Expr Expr+
                \end{itemize}
            \ides{Lexical Analysis:} Recognize integers, \verb/'+'/, \verb/'-'/, parentheses and white space
            \ides{Parsing:} Convert to abstract syntax tree
        \end{itemize}
    \ides{Evaluation}
        \begin{itemize}
            \item
\begin{verbatim}
eval :: Expr -> Int
eval (Lit n) = n
eval (Add e1 e2) = (eval e1) + (eval e2)
eval (Sub e1 e2) = (eval e1) - (eval e2)
\end{verbatim}
        \end{itemize}
    \ides{Print}
        \begin{itemize}
            \item Instance of type class show
            \item
\begin{verbatim}
instance Show Expr where
    show (Lit n) = show n
    show (Add e1 e2) = "(" ++ show e1 ++ "+" ++ show e2 ++ ")"
    show (Sub e1 e2) = "(" ++ show e1 ++ "-" ++ show e2 ++ ")"
\end{verbatim}
        \end{itemize}
    \ides{Parser}
        \begin{itemize}
            \item Given grammar is ambiguous
                \begin{itemize}
                    \item Provide user a way to get rid of ambiguity
                    \item \verb/Expr ::= Int | Expr '+' Expr | Expr '-' Expr | '(' Expr ')'/
                \end{itemize}
            \item Given grammar is left-recursive
                \begin{itemize}
                    \item Parsing \verb+Expr+ requires to first parse \verb+Expr+
                    \item We can get an infinitely non-terminating recursion
                    \item
\begin{verbatim}
Atom ::= Int | '(' Expr ')'
Expr ::= Atom | Atom '+' Expr | Atom '-' Expr/
\end{verbatim}
                \end{itemize}
            \item Parser
                \begin{itemize}
                    \item
\begin{verbatim}
data Expr = Lit Int | Add Expr Expr | Sub Expr Expr
    deriving (Show, Eq)

atom lit ||| pexpr
expr = atom ||| add ||| sub

lit = do x <- num
        return (Lit x)

pexpr = do string "("
            e <- expr
            string ")"
            return e

add = do a <- atom
        string "+"
        e <- expr
        return (Add a e)

sub = do a <- atom
        string "-"
        e <- expr
        return (Sub a e)
\end{verbatim}
                \end{itemize}
            \item Evaluator
                \begin{itemize}
                    \item
\begin{verbatim}
str2expr :: String -> Expr
str2expr s = completeParse expr s

eval :: Expr -> Int
eval (Lit n) = n
eval (Add x y) = eval x + eval y
eval (Sub x y) = eval x - eval y

calculate :: String -> Int
calculate = eval . str2expr
\end{verbatim}
                \end{itemize}
        \end{itemize}
\end{itemize}

\todo{Example 2}
