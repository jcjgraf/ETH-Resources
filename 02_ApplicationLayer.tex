%! TEX root = ./main.tex

\section{Application Layer}
\subsection{Domain Name Service (DNS)}
\begin{itemize}
    \item Provides core functionality for the internet
    \item Human identify hosts using hostnames, while internet uses IPs
    \item DNS provides hostnames to IP mapping
    \item Not one-to-one
        \begin{itemize}
            \ides{Load Balancing:} One hostname maps to multiple IP
            \ides{Reuse hardware:} Multiple hostnames map to same IP
                \begin{itemize}
                    \item E.g. host multiple websites on same machine
                \end{itemize}
        \end{itemize}
    \ides{History}
        \begin{itemize}
            \item One list of all mappings
            \item Manually shared and updated
            \item Placed in \verb+/etc/hosts+
            \icon Not scalable
            \icon Hard to manage
            \icon Inconsistent
            \icon List no always available for download
        \end{itemize}
    \ipro Scalability
    \ipro Availability
        \begin{itemize}
            \item Domains are independent of each other
        \end{itemize}
    \ipro Extensible
        \begin{itemize}
            \item Can extend any part without interference of other parts
        \end{itemize}
\end{itemize}

\subsubsection{Structure}
\begin{itemize}
    \item DNS uses three intertwined hierarchies
        \begin{itemize}
            \ides{Naming Structure}
                \begin{itemize}
                    \ides{Root:} After the TLD
                        \begin{itemize}
                            \item Imaginary dot
                        \end{itemize}
                    \ides{Top Level Domain (TLD):} Sit at the top
                    \ides{Domain:} Subtree of the TLD
                    \ides{Subdomain:} Subtree of domain
                        \begin{itemize}
                            \item Can be arbitrarily nested
                        \end{itemize}
                    \item A name is a leaf-to-root path
                \end{itemize}
            \ides{Management}
                \begin{itemize}
                    \item Root is managed by the IANA
                        \begin{itemize}
                            \item Association of 13 companies
                        \end{itemize}
                    \item TLDs are managed by private or federal organisations
                    \item Domains are managed by private organisations
                    \item Name collision is trivially avoided
                \end{itemize}
            \ides{Infrastructure}
                \begin{itemize}
                    \item $13$ root servers
                        \begin{itemize}
                            \item Named \textit{a} through \textit{m}
                            \item Each has numerous mirrors
                            \item Distributed all around the world
                            \item Two \textit{k} root server are located in CH
                        \end{itemize}
                    \item TLD are managed professionally
                    \item Domains are managed by ISPs or locally
                \end{itemize}
        \end{itemize}
    \ides{BGP Anycast}
        \begin{itemize}
            \item Allows sharing of same IP for multiple servers
            \item Determines ``fastest'' path
                \begin{itemize}
                    \item Not necessarily the shortest
                \end{itemize}
            \item Used by the roots servers for load balancing
        \end{itemize}
    \item For the system to work, we need
        \begin{itemize}
            \item Each DNS server knows the address of the root server
            \item Each root servers knows the address of all TLD servers
            \item Each DNS server knows the address of all its children
        \end{itemize}
    \ides{DNS Server}
        \begin{itemize}
            \item Each domain must have at least two DNS server to ensure availability
            \item Stores \textit{Resource Records} as tuples \verb+(name, value, type, TTL)+
                \begin{itemize}
                    \ides{Name:} The variable which we want to map
                    \ides{Value:} Desired value
                    \ides{Type:} One of:

                        \begin{tabular}{l l l}
                            Record Type & Name & Value\\
                            \hline
                            A & hostname & IP address\\
                            NS & domain & DNS server name\\
                            MX & domain & Mail server name\\
                            CNAME & alias & canonical name\\
                            PTR & IP address & corresponding hostname
                        \end{tabular}
                    \ides{Time-to-Life (TTL):} How long the mapping is value
                        \begin{itemize}
                            \item Used for caching
                        \end{itemize}
                \end{itemize}
        \end{itemize}
\end{itemize}

\subsubsection{DNS Query}
\begin{itemize}
    \item Used UDP on port 53
        \begin{itemize}
            \item Unreliable; may not get an answer
        \end{itemize}
    \ides{Main Steps}
        \begin{itemize}
            \item Browser wants to access a domain name
            \item OS triggers resolution process
                \begin{itemize}
                    \item OS sends request to local DNS server
                \end{itemize}
            \item DNS systems translates domain
        \end{itemize}
    \ides{Local DNS Server}
        \begin{itemize}
            \item Normally closed to the endhost
            \item Can be either:
                \begin{itemize}
                    \ides{Locally:} Running on host
                    \ides{Enterprise:} Somewhere in the network
                    \ides{ISP:} Running by the ISP
                    \ides{External:} Run by a third-party company
                        \begin{itemize}
                            \icon Wrong geographical local
                            \icon PGB Anycast does not give fastest server
                            \icon Insecure
                        \end{itemize}
                \end{itemize}
            \item Endhosts can choose a DNS server by:
                \begin{itemize}
                    \item Hardcoding it to \verb+/etc/resolv.conf+
                    \item Rely on dynamic allocation by \verb+DHCP+
                \end{itemize}
        \end{itemize}
    \ides{Query Type}
        \begin{itemize}
            \ides{Recursive Query:}
                \begin{itemize}
                    \item Client offloads task of resolving to the next server
                    \item Not done in practice
                        \begin{itemize}
                            \item Except for external resolvers apparently
                        \end{itemize}
                    \item Steps:
                        \begin{itemize}
                            \item DNS client asks DNS server for domain IP
                            \item DNS server queries root server
                            \item Root server queries TLD server
                            \item TLD server queries domain server
                            \item Domain server returns domain IP to TLD server
                            \item TLD server returns domain IP to root server
                            \item Root server returns domain IP to DNS server
                            \item DNS server return domain IP to DNS client
                        \end{itemize}
                \end{itemize}
            \ides{Iterative Query:}
                \begin{itemize}
                    \item Each server only returns address to next server
                    \item Steps
                        \begin{itemize}
                            \item DNS client asks DNS server for IP
                            \item DNS server queries root server
                            \item Root server returns TLD server IP to DNS server
                            \item DNS server queries TLD server at given IP
                            \item TLD server returns domain server IP to DNS server
                            \item DNS server queries domain server at given IP
                            \item Domain server returns domain IP to DNS server
                            \item DNS server return IP to DNS client
                        \end{itemize}
                    \item Somehow we sometimes say that CNS client is recursively (makes only one request) while the DNS server is iteratively (makes many requests)
                \end{itemize}
        \end{itemize}
\end{itemize}

\subsubsection{Caching}
\begin{itemize}
    \item Most DNS queries requests the same site
    \item Cache result of former queries
    \item DNS records is cached for TTL seconds
    \item Hit rate of about $75\%$
\end{itemize}

\subsubsection{Problems}
\begin{itemize}
    \item DNS is vital for the internet
        \begin{itemize}
            \item Server not accessible when DNS server not available
        \end{itemize}
    \item DNS servers can see the sites one visits
        \begin{itemize}
            \item Sell data
        \end{itemize}
    \item Vulnerable to different security breaches
\end{itemize}

\subsection{Web}
\subsection{Internet Video}
