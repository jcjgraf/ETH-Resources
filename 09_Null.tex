%! TEX root = ./main.tex

\section{NULL}
\begin{itemize}
    \item \verb+NULL+ is a state and not a value
        \begin{itemize}
            \item Check: \verb+some_value IS NULL+
                \begin{itemize}
                    \item \verb+(NULL IS NULL) -> TRUE+
                \end{itemize}
            \item And not: \verb+some_value = NULL+
                \begin{itemize}
                    \item \verb+(NULL = NULL) -> UNKNOWN+
                \end{itemize}
            \item We cannot compare \verb+NULL+, but we can check if it is \verb+NULL+
        \end{itemize}
    \ides{Arithmetic:} Always gives \verb+NULL+ if an operand is \verb+NULL+
    \ides{Comparison:} Always gives \verb+UNKNOWN+ if one of the comparators is \verb+NULL+
    \ides{Logical operator:} Treats \verb+NULL+ as \verb+UNKNOWN+
        \begin{itemize}
            \item Returns \verb+UNKNOWN+ when the result depends on the concrete value of the variable assigned to \verb+NULL+
        \end{itemize}
    \ides{Aggregation}
        \begin{itemize}
            \item If \verb+group by+ a columns containing \verb+NULL+s, \verb+NULL+ will be in one group
            \item Most aggregation functions ignore the \verb+NULL+
            \item \verb+Count(*)+ ignores \verb+NULL+ not, \verb+Count(column)+ ignores it
        \end{itemize}
    \item Some operators may introduce \verb+NULL+
        \begin{itemize}
            \item E.g. (Left/Right) outer join
        \end{itemize}
\end{itemize}
