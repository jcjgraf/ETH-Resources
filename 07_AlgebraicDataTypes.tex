%! TEX root = ./main.tex

\section{Algebraic Data Types}
\begin{itemize}
    \item Declare new data types suitable for the object being modeled
    \ipro Less error prone
\end{itemize}

\subsection{Different Types}
\subsubsection{Enumeration Types}
\begin{itemize}
    \item Set of possible types
        \begin{itemize}
            \item Each element is a \textit{type constructor}
        \end{itemize}
    \item Initiated by keyword \verb+data+
    \item Constructors must have unique names
    \item First letter of each constructor must be upper-case
    \item \verb+data TypeName = Const1 | Const2 | Const3+
    \item Function can use this type for pattern matching
        \begin{itemize}
            \item \verb+func :: TypeName -> SomeOtherType+
        \end{itemize}
\end{itemize}

\subsubsection{Product Type}
\begin{itemize}
    \item Consists of a type name and a set of ``attributes''
        \begin{itemize}
            \item Attribute must be a certain type
                \begin{itemize}
                    \item Giving an alias using \verb+type+ adds a layer of abstraction
                \end{itemize}
        \end{itemize}
    \item \verb+data TypeName = Name Attr1 Attr2+
        \begin{itemize}
            \item \verb+type Attr1 = Sometype+
        \end{itemize}
    \item Constructor is a function \verb+Attr1 -> Attr2 -> TypeName+
    \item Functions can use this type for pattern matching
        \begin{itemize}
            \item
\begin{verbatim}
func :: Type1 -> SomeOtherType
func (Name Attr1 Attr2) = ...
\end{verbatim}
        \end{itemize}
    \item Could use tuples instead of product types
        \begin{itemize}
            \item \verb+data TypeName = (Attr1, Attr2)+
            \icon Makes arguments ambiguous
            \ipro Allows application of polymorphic functions like \verb+fst, zip...+
            \ipro Shorter definition
        \end{itemize}
\end{itemize}

\subsubsection{Enumeration and Product Types}
\begin{itemize}
    \item Enumeration and product types can be combined
    \item \verb+data TypeName = Name1 Attr1 | Name2 Attr1 Attr2+
    \item Functions can use this type for pattern matching
        \begin{itemize}
            \item
\begin{verbatim}
func :: TypeName -> SomeOtherType
func (Name1 Attr1)       = ...
func (Name2 Attr1 Attr2) = ...
\end{verbatim}
        \end{itemize}
\end{itemize}

\subsection{Integration with Classes}
\begin{itemize}
    \item Default function are not defined for algebraic data types
    \item Have to be explicitly created
        \begin{itemize}
            \item
\begin{verbatim}
data TypeName = Name1 Attr1 | Name2 Attr1 Attr2
instance TypeClass TypeName where
 ...
\end{verbatim}
        \end{itemize}
    \item In some cases class instances can be automatically derived
        \begin{itemize}
            \item 
\begin{verbatim}
data TypeName = Name1 Attr1 | Name2 Attr1 Attr2
                deriving(TypeClass1, TypeClass2, TypeClass3)
\end{verbatim}
        \end{itemize}
\end{itemize}

\subsection{Recursive Types}
\begin{itemize}
    \item Defined using recursive data types
        \begin{itemize}
            \item \verb+data Expr = Lit Int | Add Expr Expr+
        \end{itemize}
    \item Are evaluated recursively
        \begin{itemize}
            \item
\begin{verbatim}
eval :: Expr -> Int
eval (Lit n) = n
eval (Add a b) = (eval a) + (eval b)
\end{verbatim}
        \end{itemize}
\end{itemize}

\subsubsection{Trees}
\begin{itemize}
    \item Are a prime example
    \item Can describe many data structures
    \item
\begin{verbatim}
data Tree t = Leaf | Node t (Tree t) (Tree t)
            deriving (Eq, Ord, Show)
\end{verbatim}
    \item Standard types are algebraic data types defined in the prelude
\end{itemize}

\subsection{Algebraic Types and Type Classes}
\begin{itemize}
    \item Algebraic types are \textit{fist class} citizens
        \begin{itemize}
            \item Fully compatible with polymorphism and type classes
        \end{itemize}
    \ipro Make program simpler to read and understand
    \ipro Allow reusability
\end{itemize}

\subsection{Correctness}
\begin{itemize}
    \ides{Natural Number}
        \begin{itemize}
            \item \verb+data Nat = Zero | Succ Nat deriving (Eq, Ord, Show)+
            \item Isomorphic to \verb+{Zero, Succ Zero, Succ (Succ Zero), ...}+
            \item Build step by step
            \item Allows structural induction proofs
        \end{itemize}
    \ides{Lists}
        \begin{itemize}
            \item \verb+data L t = Nil | Cons t (L t)+
            \item Elements in $L \ t$ are build in steps
                \begin{itemize}
                    \item $\{\text{Nil}\}$
                    \item $\{\text{Cons } a \text{ Nil} \in L \ t \mid a \in t\}$
                    \item $\{\text{Cons } b (\text{Cons } a \text{ Nil}) \in L \ t \mid a,b \in t\}$
                    \item $\vdots$
                \end{itemize}
            \item $l \in L \ t$ iff $l$ appears in some of the construction step
            \item \todo{Add deduction rule}
        \end{itemize}
    \ides{Trees}
        \begin{itemize}
            \item \verb+data Tree t = Leaf | Node t (Tree t) (Tree t)+
            \item Elements in $Tree \ t$ are build in steps
                \begin{itemize}
                    \item $\{\text{Leaf}\}$
                    \item $\{\text{Node } a \text{ Leaf Leaf} \in Tree \ t \mid a \in t\}$
                    \item $\vdots$
                    \item Trees in step $i$ are of form $\text{Node }a \ l \ r$ where $a \in t$, and $l$ and $r$ were constructed in the previous step
                \end{itemize}
            \item $s \in Tree \ t$ iff $s$ appears in some of the construction step
            \item \todo{Add deduction rule}
        \end{itemize}
    \ides{General Idea}
        \begin{itemize}
            \item Adopt induction to the structure of the algebraic data type
            \item Proof non-recursively step $0$
            \item Proof recursively how to get from step $i - 1$ to $i$
        \end{itemize}
\end{itemize}
