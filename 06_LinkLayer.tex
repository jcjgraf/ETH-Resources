%! TEX root = ./main.tex

\section{Link Layer}
\begin{itemize}
    \item Transfers messages over one or more established links
    \item Messages are frames
    \item Frames have a limited size
    \item Adds header and sometimes trailer to network packet
    \item Link layer is split in two piece
        \begin{itemize}
            \item OS driver
            \item Network interface card (NIC)
                \begin{itemize}
                    \item Interface to physical layer
                \end{itemize}
        \end{itemize}
\end{itemize}

\subsection{Framing}
\begin{itemize}
    \item Physical layer provides stream of bits
    \item Need to figure out where a frame starts and ends
    \item Different methods
    \ides{Byte Count}
        \begin{itemize}
            \ides{Idea:} Add length field to header
            \ipro Simple
            \icon Hard to re-synchronize after error to length field
        \end{itemize}
    \ides{Byte Stuffing}
        \begin{itemize}
            \ides{Idea:} Byte length flag \verb+Flag+ which indicates start and end of frame
            \icon Flag may be part of the payload
                \begin{itemize}
                    \item Must be escaped: \verb+ESC FLAG+
                    \item Escape may be part of the payload too
                        \begin{itemize}
                            \item Must be escaped itself: \verb+ESC ESC+
                        \end{itemize}
                \end{itemize}
            \icon Problematic on bitflip of flags/escapes
            \ides{Worst case}: Frame size gets double
            \ides{Average case}: $\approx 0.6$ or $1/28$ or something, not sure \todo{Find average case}
        \end{itemize}
    \ides{Bit Stuffing}
        \begin{itemize}
            \item Same principle as byte stuffing
            \item $6$ consecutive $1$s are a \verb+FLAG+
            \ides{On transmit:} Insert a $0$ after $5$ consecutive $1$s
            \ides{On receive:} delete each $0$ which occurs after $5 \ 1$s
            \ides{Worst case:} Size gets longer by $20\%$
            \ides{Average case:} \todo{Find average case}
        \end{itemize}
\end{itemize}


\subsection{Error Detection and Correction}
\begin{itemize}
    \item Get bit-errors due to signal attenuation and noise
        \begin{itemize}
            \item We assume the bit flips are random
        \end{itemize}
    \ides{Goal:} Detect and possibly recover from such errors
    \ides{Error Codes}
        \begin{itemize}
            \item Code word consists of $D + R$ bits
                \begin{itemize}
                    \ides{D:} Data bits which if our message
                    \ides{R:} Check bits which are computed as $R = \text{fn}(D)$
                \end{itemize}
            \ides{Sender:}
                \begin{itemize}
                    \item Computes $R = \text{fn}(D)$
                    \item Sets $R$ and sends frame of
                \end{itemize}
            \ides{Receiver:}
                \begin{itemize}
                    \item Takes $D'$ and computes $R'= \text{fn}(D')$
                        \begin{itemize}
                            \ides{$\mathbf{D'}$:} is $D$ with possible, unknown errors
                        \end{itemize}
                    \item Compare $R = R'$
                \end{itemize}
            \item Set of correct codewords should be a tiny fraction of the set of all codewords
                \begin{itemize}
                    \item Reduces probability of a getting a correct codeword by randomly selection
                \end{itemize}
        \end{itemize}
    \ides{Hamming Distance}
        \begin{itemize}
            \ides{Distance:} Number of bit flips required to turn $D + R_1$ into $D + R_2$
            \ides{Hamming Distance} for a code is the minimal distance between any pair of codewords
            \ides{Error Detection:} Code of hamming distance $d + 1$ can detect up to $d$ errors
            \ides{Error Correction:} Code of hamming distance $2d + 1$ can correct up to $d$ errors
        \end{itemize}
    \ides{Error Detection}
        \begin{itemize}
            \item Add check bits to the message
            \item Different methods
            \ides{Duplicate Send of Message}
                \begin{itemize}
                    \ides{Idea:} Send each message twice
                    \item Error if both are different
                    \item Can only detect a single error
                    \item Fails if same bit is flipped in both copies
                \end{itemize}
            \ides{Parity Bit}
                \begin{itemize}
                    \item $R$ has length $1$
                    \item For $D$ date bits, $R$ is the sum (modulo or XOR) of the all bits of $D$
                        \begin{itemize}
                            \item I.e. $R \equiv_2 \sum_{i=1}^{D} d_i$
                        \end{itemize}
                    \ides{Distance:} $2$
                        \begin{itemize}
                            \item Can detect $1$
                            \item Can correct $0$
                        \end{itemize}
                    \item Detect errors with $50\%$ chance
                    \icon Fails for systematic errors where bits of the $D$ are swapped
                \end{itemize}
            \ides{Checksum}
                \begin{itemize}
                    \item $R$ has length $N$
                    \ides{Internet Checksum:} Production example
                        \begin{itemize}
                            \item Used in TCP, UDP etc.
                            \ides{Sender side:}
                                \begin{itemize}
                                    \item Arrange data in $16$-bit words
                                    \item Put zero in checksum position
                                    \item Add numbers
                                    \item Add any carry-over back
                                    \item Calculate $ffff -$ res
                                \end{itemize}
                            \ides{Receiver side:}
                                \begin{itemize}
                                    \item Arrange data in $16$-bit words
                                    \item Add numbers (including checksum)
                                    \item Add any carry-over back
                                    \item Checksum will be non-zero, add
                                    \item Calculate $ffff -$ res
                                    \item If $0$ ok, else error
                                \end{itemize}
                            \item Distance is $2$
                                \begin{itemize}
                                    \item Can detect $1$
                                    \item Can correct $0$
                                \end{itemize}
                            \item Often errors occur in consecutive bits
                                \begin{itemize}
                                    \ipro Errors are always detected if first and last error bit are within $16$ bit range
                                \end{itemize}
                            \icon Vulnerable to systematic errors (switching of blocks)
                        \end{itemize}
                    \ipro Easy and simple
                    \ipro Stronger than parity
                \end{itemize}
            \ides{Cyclic Redundancy Check (CRC)}
                \begin{itemize}
                    \item Widely used: Ethernet, WIFI, ADSL
                    \ides{Generator:} bit pattern of length $k + 1$
                        \begin{itemize}
                            \item Both sides agree on a specific one
                            \item Can be expressed as a polynomial over $GF(2)$
                        \end{itemize}
                    \ides{Sender side:}
                        \begin{itemize}
                            \item Extend the $len(D) = d$ date bits with $k$ zeros
                            \item Divide by the generator $C$
                                \begin{itemize}
                                    \item If $1$ in leading position, take XOR
                                    \item If $0$ in leading position, take new value down and drop leading $0$
                                \end{itemize}
                            \item Add insert remainder as check bits
                        \end{itemize}
                    \ides{Receiver side:}
                        \begin{itemize}
                            \item Divide received message by $C$
                            \item If $0$ no error, else error
                        \end{itemize}
                    \item Protection depends on the generator $C$
                    \item Distance $4$
                        \begin{itemize}
                            \item For standard CRC-32 generator
                            \item Can detect $3$
                            \item Can correct $1$
                        \end{itemize}
                    \item Detect odd number of errors
                    \item Not vulnerable to systemanis errors
                    \item Detect consecutive bit errors of length $\le k$
                \end{itemize}
        \end{itemize}
    \ides{Error Correction}
        \begin{itemize}
            \item Add more check bits to recover from error
            \item Hard because check bits can contain errors too
            \ides{Hamming Code}
                \begin{itemize}
                    \item Has hamming distance $3$
                    \ides{Sender side:}
                        \begin{itemize}
                            \item Calculate minimal $k$ such that the equation is fulfilled $n \le 2^k - k - 1$
                                \begin{itemize}
                                    \ides{n:} Message length
                                    \ides{k:} Number of check bits
                                \end{itemize}
                            \item Insert check bits into original message at positions $2^i, i \in \N_0$
                            \item Check bit $i$ is a parity for values with a $2^i$ in their value
                                \begin{itemize}
                                    \item $2^i = 1$ iff sum of all values which contain $2^i$ $\equiv_2 1$, else $0$
                                \end{itemize}
                        \end{itemize}
                    \ides{Receiver side:}
                        \begin{itemize}
                            \item Recompute check bit sums
                            \ides{Syndrome:} Result of sums arranged as binary number
                            \item No error if syndrome is $0$
                            \item Bit flip at position $i =$ ``syndrome'' if syndrome $\neq 0$
                                \begin{itemize}
                                    \item Undo bit flip and get correct data
                                \end{itemize}
                        \end{itemize}
                \end{itemize}
            \item Practically used error correcting codes are much more involved
        \end{itemize}
    \ides{Detection vs Correction}
        \begin{itemize}
            \item Which is better depends on the pattern of errors
            \ides{Error Detection:} more efficient if no errors are expected and they are large when they occur
                \begin{itemize}
                    \item Heavily used in link layer and application layer
                \end{itemize}
            \ides{Error Correction:} needed when we expect errors (small number only) or if there is not time to retransmit
                \begin{itemize}
                    \item Heavily used in physical layer
                \end{itemize}
        \end{itemize}
\end{itemize}

\subsection{Retransmission} \todo{Merge into error detection and recover?}
\begin{itemize}
    \item How to deal with:
        \begin{itemize}
            \item lost messages
            \item messages with unrecoverable errors
        \end{itemize}
    \ides{Automatic Repeat reQuest (ARQ)}
        \begin{itemize}
            \item Used when errors are common or must be corrected
            \ides{Receiver side:}
                \begin{itemize}
                    \item Sends ACK iff receives message and it is correct
                \end{itemize}
            \ides{Sender side:}
                \begin{itemize}
                    \item Resends message if no ACK received within a set timeout
                \end{itemize}
            \ides{Stop-And-Wait:} implementation of ARQ
        \end{itemize}
\end{itemize}

\subsection{Multiplexing}
\begin{itemize}
    \ides{Multiplexing:} Share a limited resource
    \item One of the core goals in networking
    \item Two main principles
        \begin{itemize}
            \ides{Time-Division Multiplexing (TDM):} User sends at fixed schedule
                \begin{itemize}
                    \item Send high rate at a fraction of time
                \end{itemize}
            \ides{Frequency-Division Multiplexing (FDM):} Users send at different frequencies
                \begin{itemize}
                    \item Send low rate all the  time
                \end{itemize}
        \end{itemize}
    \ides{Multiple Access}
        \begin{itemize}
            \item Network traffic is bursty
            \item Infeasible to use TDM or FDM for every user
            \ides{Statistical Multiplexing:} Method of oversubscribing a limited resource
                \begin{itemize}
                    \item Assume that not everyone will use the resource at the same time
                \end{itemize}
            \item Need Multiple Access protocols to allocate users according to their demand
            \item Two main principles
                \begin{itemize}
                    \ides{Randomized Multiple Access}
                        \begin{itemize}
                            \item Nodes randomize their resource access attempts
                            \ipro Low load situations
                        \end{itemize}
                    \ides{Contention-Free Multiple Access}
                        \begin{itemize}
                            \item Nodes order their resource access attempts
                            \ipro High load or guaranteed QoS
                        \end{itemize}
                \end{itemize}
        \end{itemize}
\end{itemize}

\subsubsection{Randomized Multiple Access}
\begin{itemize}
    \item Assume we have a distributed system w/o leader
    \ides{ALOHA}
        \begin{itemize}
            \item Used in late 1960s
            \item Send when there is data
            \item On collision (no ACK received), drop packet
            \item Try to resend after a random timeout
            \ipro Works well when little traffic
        \end{itemize}
    \ides{Carrier Sense Multiple Access (CSMA)}
        \begin{itemize}
            \item Improves ALOHA
            \item Listens before sending to reduce collision probability
            \item Called \textbf{1-persistent} CSMA since once channel is free, it sends with probability $1$
            \icon Collisions can still happen:
                \begin{itemize}
                    \item Two nodes listen at the same time and here nothing and start sending at the same time
                    \item One node is sending, another is listening, but not hearing anything due to delay
                \end{itemize}
            \icon Upon collision, time is lost since some sending nodes may keep sending
        \end{itemize}
    \ides{CSMA with Collision Detection (CSMA/CD)}
        \begin{itemize}
            \item Upon collision, abort sending and jam the wire
                \begin{itemize}
                    \item Make everyone sending aware of the collision so that they abort sending \end{itemize}
            \ides{Jam:} Send for a short amount of time to make every sender aware of the collision
            \ides{Contention Slot / Slot time:} Time a node must wait before it can be sure that no other node's tarnmission has collided with its transmission
                \begin{itemize}
                    \item Is $2D$
                        \begin{itemize}
                            \ides{D:} Longest transmission delay possible in the network
                        \end{itemize}
                    \item Imposes a minimal frame size such that sending takes at least $2D$ seconds
                    \item Else a sender may not detect a collision
                \end{itemize}
            \end{itemize}
    \ides{CSMA ``Persistence''}
        \begin{itemize}
            \item Problem:
                \begin{itemize}
                    \item While one node is sending, other nodes which want to send will queue up
                    \item Once the wire is free, all try to send and will collide
                \end{itemize}
            \ides{Binary Exponential Backoff (BEB)}
                \begin{itemize}
                    \ides{Idea:} When $N$ nodes are queued, each will send with a probability of $\frac{1}{N}$
                    \item Working
                        \begin{itemize}
                            \item Initially, node directly sends
                            \item On first collision, node sends directly or $1$ frame later with each $50\%$ chance
                            \item On second collision, node waits either $0, 1, 2, 3$ frames (with same probability) before sending
                            \item On $i$-th collision, node waits either $0, \dots, 2^i-1$ frames
                        \end{itemize}
                    \ipro Highly efficient
                    \item Used by Ethernet
                \end{itemize}
        \end{itemize}
    \ides{Classical Ethernet}
        \begin{itemize}
            \item Introduced in 1973
            \item Hosts share a single coaxial cable
            \item Allowed for $10$ Mbps
            \item Uses CSMA/CD with BEB
            \item $2D$ is $64$ bytes
            \ides{Frame Format}
                \begin{itemize}
                    \item Contains source and destination MAC address
                    \item Puts the IP package into the DATA field
                        \begin{itemize}
                            \item Has size $0 - 1500$ bytes
                        \end{itemize}
                    \item Padding is added in PAD to mache frame reach minimal $64$ byte size
                    \item CRC-32 checksum for error detection
                \end{itemize}
        \end{itemize}
    \ides{Modern Ethernet}
        \begin{itemize}
            \item Based on switches
            \item No multiple access
            \item No problems of collisions anymore
        \end{itemize}
    \ides{Wireless Multiple Access}
        \begin{itemize}
            \item Main challenges
                \begin{itemize}
                    \item Nodes cannot hear while sending
                        \begin{itemize}
                            \item Cannot detect collisions
                            \item Send full frame and only realize after that there was a collision
                        \end{itemize}
                    \item Nodes may have different coverage areas
                        \begin{itemize}
                            \ides{Signal to Noise Ration (SNR):} Node can receive signal only if SNR is large enough
                            \ides{Hidden Terminal:} Two nodes are hidden terminals iff they cannot reach each other but can collide at an intermediate station
                            \ides{Exposed Terminal:} Two nodes are exposed terminals iff they can hear each other but do not collide
                                \begin{itemize}
                                    \item Both sender think there is a collision but there is actually none
                                    \item Want to send concurrently to increase performance
                                \end{itemize}
                        \end{itemize}
                \end{itemize}
            \ides{Multiple Access with Collision Avoidance (MACA)}
                \begin{itemize}
                    \item Alternative approach to CSMA
                    \ides{Idea:} Send short handshake to ask if allowed to send
                    \item Rules
                        \begin{itemize}
                            \ides{Request-to-Send (RTS):} Send by sender node to ask for allowance
                                \begin{itemize}
                                    \item Contains the frame length the sender wants to send
                                \end{itemize}
                            \ides{Clear-to-Send (CTS):} Send by receiver if sender is allowed to send frame
                                \begin{itemize}
                                    \item Contains the frame length the sender wants to send
                                \end{itemize}
                            \item Sender transmits frame
                        \end{itemize}
                    \item Hidden terminal does not receive senders RTS but receivers CTS
                        \begin{itemize}
                            \item Terminal knows that it should not send to receiver
                            \item Terminal knows from the frame length of the CTS how long the transmission will take
                        \end{itemize}
                    \item Exposed terminal does receive RTS but not CTS
                        \begin{itemize}
                            \item Terminal knows that it should not send to sender but is free to send to any receiver
                        \end{itemize}
                    \icon Collision still possible
                        \begin{itemize}
                            \item But less likely
                        \end{itemize}
                \end{itemize}
            \ides{WiFi}
                \begin{itemize}
                    \item Started in 1990s
                    \item Clients are wireless connected to wired AP
                    \ides{Physical Layer}
                        \begin{itemize}
                            \item Uses $20/40$ MHz channels on ISM (public available but only very local) band
                            \item Different frequencies depending on the WiFi Version
                        \end{itemize}
                    \ides{Link Layer}
                        \begin{itemize}
                            \item Multiple access using CSMA/CA
                            \item RTS/CTS is optional
                            \item Uses ARQ
                            \item Funky addressing due to AP \todo{What does it mean?}
                            \item Errors detection with a $32$-bit CRC
                            \item MTU is larger than for Ethernet
                            \item Can have many additional features
                        \end{itemize}
                    \ides{CSMA with Collision Avoidance (CSMA/CA):}
                        \begin{itemize}
                            \item After link gets free, queued sender select a random backoff (gap)
                            \item Node with smallest backoff goes first
                            \item Prevent starvation by always using the rest of the backoff and only select a new backoff after having send
                        \end{itemize}
                \end{itemize}
        \end{itemize}
\end{itemize}


\subsection{Contention-Free Multiple Access}
\begin{itemize}
    \ides{Turn-Taking Multiple Access Protocols}
        \begin{itemize}
            \ides{Idea:} Order nodes and each can send (or pass) if it is their turn
            \item Different possible orderings
            \ides{Token Ring}
                \begin{itemize}
                    \item Arrange nodes in a ring
                    \item ``Send permission'' token rotates to each node in turn
                    \ipro Fixed overhead
                    \ipro No collision
                    \ipro Regular change to send
                    \icon Complexity
                        \begin{itemize}
                            \item Who controls the rotation and what if controller does?
                            \item What if token is lost?
                        \end{itemize}
                    \icon Higher overhead at low load
                \end{itemize}
        \end{itemize}
    \item Random multiple access is dominant and hard to beat
\end{itemize}

\subsection{LAN Switches}
\begin{itemize}
    \item Used in modern Ethernet instead of a shared wire
    \item Hosts are wired to a switch ports using a twisted pair cables
    \item There are three types of devices:
        \begin{itemize}
            \ides{Hub or Repeater:} On physical layer
                \begin{itemize}
                    \item All ports are wired together
                    \item Signals are amplified among all ports
                    \item Similar to classic Ethernet but more convenient and reliable
                \end{itemize}
            \ides{Switch:} Up to link layer
            \ides{Router:} Up to network layer
        \end{itemize}
    \ides{Inside a Switch}
        \begin{itemize}
            \item Uses frame addresses to connect input port to the right output port
                \begin{itemize}
                    \item Done by a \textit{fabric}
                \end{itemize}
            \item Multiple frames may be switched in parallel
            \ides{Full-Duplex:} Traffic can flow in both direction
            \item Not multiple access control, hosts can just send
            \item Need buffers at input and output ports
        \end{itemize}
    \ides{Forwarding}
        \begin{itemize}
            \item Need to find the right output port based on the destination address in the Ethernet frame (MAC address)
            \ides{Backward Learning}
                \begin{itemize}
                    \item Possible solution to forwarding problem
                    \ides{Idea:} Use table to map MAC address to port
                    \item Works in two steps
                        \begin{itemize}
                            \ides{Receive frame:} ``Learn'' mapping my looking at source address and port on which the frame was received
                                \begin{itemize}
                                    \item Insert new mapping into table
                                \end{itemize}
                            \ides{Sending frame:} Index into table to get port, if not existing, broadcast frame on all ports
                        \end{itemize}
                    \ipro Also works if multiple addresses are behind one port (e.g. another switch)
                        \begin{itemize}
                            \item If there are no forwarding loops
                        \end{itemize}
                    \icon Table is of fixed size and can get overflow
                        \begin{itemize}
                            \item E.g. due to a hack
                            \item Switch acts equivalent to a repeater (broadcasts every message)
                        \end{itemize}
                \end{itemize}
            \ides{Forwarding Loops}
                \begin{itemize}
                    \item Loop in a topology
                    \item Used for reliability or by mistake
                    \item Cheap switches do not have a duplicate detection
                    \item Frame will loop infinitely
                \end{itemize}
            \ides{Spanning Tree}
                \begin{itemize}
                    \item Forwarding method fixing forwarding loops
                    \ides{Idea:}
                        \begin{itemize}
                            \item Switches collectively find a spanning free for the topology
                            \item Frames are forwarded along the tree and not all links
                            \item Broadcasts will go up to the root and down to all branches
                            \item Use backward learning on the tree
                        \end{itemize}
                    \ides{Spanning Tree Algorithm}
                        \begin{itemize}
                            \item Required
                                \begin{itemize}
                                    \item All switches run the same algorithm
                                    \item They start with no information
                                    \item Operate in parallel and send messages
                                    \item Always search for the best solution
                                \end{itemize}
                            \item Steps
                                \begin{itemize}
                                    \item Elect a root node of the tree
                                        \begin{itemize}
                                            \item Switch with the lowest address
                                            \item At the beginning, each nodes selects itself
                                        \end{itemize}
                                    \item Grow tree as shortest distance from the root
                                        \begin{itemize}
                                            \item On tie, select the one with the lower MAC address
                                        \end{itemize}
                                    \item Turn of ports for forwarding if they are not part of the tree
                                \end{itemize}
                            \ides{Root Port (RP):} Outgoing port towards the root node
                            \ides{Designated Port (DP):} Incoming port to a switch
                            \ides{Blocked Ports (BP):} Do not forward any traffic
                            \ides{Note:} When we have to run the algo on a network, we do not have to to it step-by-step but more directly
                                \begin{itemize}
                                    \item[1)] Select node with lowest address ($\implies$ root node)
                                    \item[2)] Find shortest-path tree from root
                                \end{itemize}
                            \item Nodes send periodic updates to neighbours
                                \begin{itemize}
                                    \item Contains own address, address of the root and distance in hops to the root
                                    \item Done to detect failure
                                \end{itemize}
                            \icon Final topology may not be optimal for some nodes
                        \end{itemize}
                    \icon Tree may break when a link fails
                \end{itemize}
        \end{itemize}
\end{itemize}

