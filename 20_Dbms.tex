%! TEX root = ./main.tex

\section{DMDB System}
\begin{itemize}
    \item Complex application
    \item Simplification we consider
        \begin{itemize}
            \item All data are stored on disk
            \item Disk is larger than memory
            \item Disk is slower than memory
            \item Disk flavors different access pattern
            \item Single CPU
            \item One relation is stored in a single file
        \end{itemize}
\end{itemize}

\subsection{Storage Hierarchy}
\begin{itemize}
    \item Storage is a hierarchy
    \item Challenge: keep CPU busy
    \item Rapidly changing
    \item Different hierarchies lead to different DB design
    \ides{Hard Drive}
        \begin{itemize}
            \item Plates spin
            \item Plate is split into sectors of fixed size
            \item Arm assembly is moved in or out to position a head on a desired track
            \item Tracks under head makes a cylinder (kind off)
            \item Only one head reads/writes at any time
            \item Block size is a multiple of sector size
            \item Performance
                \begin{itemize}
                    \ides{Seek time $t_s$:} Moving arm to position disk head on track
                        \begin{itemize}
                            \item $10 - 20 ms$
                        \end{itemize}
                    \ides{Rotate time $t_r$:} waiting for block to rotate under head
                        \begin{itemize}
                            \item $10 - 20 ms$
                        \end{itemize}
                    \ides{Transfer time $t_{tr}$:} actually moving data to/from disk surface
                        \begin{itemize}
                            \item $8KB / 0.1  ms$
                        \end{itemize}
                    \ides{Random access $\mathbf{D}$ times:} $D(t_s + t_r + t_{tr})$
                    \ides{Sequential access $\mathbf{D}$ time:} $t_s + t_r + Dt_{tr}$
                \end{itemize}
        \end{itemize}
    \item We consider abstraction HD $\to$ DRAM $\to$ CPU
    \item DRAM $\to$ CPU is much faster than HD $\to$ DRAM
        \begin{itemize}
            \item We only consider HD $\to$ DRAM optimisation
        \end{itemize}
\end{itemize}

\subsection{Disk Manager}
\begin{itemize}
    \item Lowest level
    \item Used by higher levels for
        \begin{itemize}
            \item Allocate/de-allocate pages
            \item Read/Write pages
        \end{itemize}
    \item Requests for sequential allocation must be satisfied
    \item Responsible for maintaining a database's files
    \item DB content is stored as one or multiple files
    \item Many DMDBs use the file system provided by the OS
        \begin{itemize}
            \item People used to build custom file systems for DMDBs
        \end{itemize}
    \ides{Files} contain a collection of pages
    \ides{Page} is a fixed-size block of data
        \begin{itemize}
            \item Contains a collection of tuples
            \ides{page id:} unique identifier of each page
        \end{itemize}
    \item A relation is stored as a collection of pages
\end{itemize}

\subsubsection{File Layout}
\begin{itemize}
    \item How does a file manage all its pages?
    \item Unordered collection of pages
    \item Support record level operations
    \item We must keep track of:
        \begin{itemize}
            \item the pages in the file
            \item the records on each page
            \item free space on each page
        \end{itemize}
    \ides{Heap file:} One/multiple header files
    \item Two ways to implement it
    \ides{Linked List}
        \begin{itemize}
            \ides{Header Page:} One single page
                \begin{itemize}
                    \item Kind of the root
                    \item Has to pointer to two linked lists
                        \begin{itemize}
                            \item Free pages list
                            \item Data pages list
                        \end{itemize}
                \end{itemize}
            \icon No global view on data
            \item Performance
                \begin{itemize}
                    \item Assume
                        \begin{itemize}
                            \item Directory fits in and is in memory
                            \item $\# Pages = D$
                            \item Pages are randomly allocated on disk
                                \begin{itemize}
                                    \item Models worst-case
                                \end{itemize}
                        \end{itemize}
                    \ides{Insert:} $t_{s + r}  + 2 t_{trans}$
                        \begin{itemize}
                            \item If page 1 has slot available
                        \end{itemize}
                    \ides{Find Record:} By non-RID value (RID is a pair page, id and slot id)
                        \begin{itemize}
                            \item $\frac{D}{2}(t_{r+s} + t_{trans})$
                        \end{itemize}
                    \ides{Scan:}
                        \begin{itemize}
                            \item $D(t_{s+r} + t_{trans})$
                        \end{itemize}
                \end{itemize}
        \end{itemize}
    \ides{Page Directory}
        \begin{itemize}
            \ides{Header Page:} Multiple pages
                \begin{itemize}
                    \item Each contains a list of pointer to data pages
                \end{itemize}
            \item Performance
                \begin{itemize}
                    \item Assume
                        \begin{itemize}
                            \item Directory fits in and is in memory
                            \item $\#Pages = D$
                            \item Pages are sequentially allocated on disk
                                \begin{itemize}
                                    \item Models best case
                                \end{itemize}
                        \end{itemize}
                    \ides{Insert:} $t_{s+r} + 2t_{trans}$
                    \ides{Find Record:} By non-RID value (RID is a pair page, id and slot id)
                        \begin{itemize}
                            \item $t_{s+r} + \frac{D}{2}t_{trans}$
                        \end{itemize}
                    \ides{Scan:}
                        \begin{itemize}
                            \item $t_{s+r} + Dt_{trans}$
                        \end{itemize}
                \end{itemize}
            \end{itemize}
\end{itemize}

\subsubsection{Page Layout}
\begin{itemize}
    \item Page consists of
        \begin{itemize}
            \ides{Header:} Contains metadata like
                \begin{itemize}
                    \item Page size
                    \item DBMS version
                    \item Compression information
                    \item Encryption information
                    \item Checksum
                \end{itemize}
            \ides{Data:} Actual tuples
        \end{itemize}
    \item Header is at top of page
    \item Data starts after header
    \item Multiple strategies
    \ides{Naive Strategy}
        \begin{itemize}
            \item Page is split into slots of fixed size
                \begin{itemize}
                    \item One tuple goes into one slot
                \end{itemize}
            \item Header keeps track of occupied/free slots
            \icon Lots of space wasted when tuples are of different length
        \end{itemize}
    \ides{Slotted Page}
        \begin{itemize}
            \item Record id = $<\text{page id}, \text{slot \#}>$
            \ides{Slot array:} Array of pointers and size of occupied slots
                \begin{itemize}
                    \item Comes right after the header
                \end{itemize}
            \ipro Can move tuples on page without changing record id
        \end{itemize}
\end{itemize}

\subsubsection{Tuple layout}
\begin{itemize}
    \item Data access methods:
        \ides{On-line transaction processing (OLTP):}
            \begin{itemize}
                \item Simple query
                \item Reads/writes a small amount of data related to a single entry
            \end{itemize}
        \ides{On-line analytical processing (OLAP):}
            \begin{itemize}
                \item Complex queries
                \item Read large portions of the DB spanning multiple entries
            \end{itemize}
        \item Multiple methods
    \ides{Row Storage}
        \begin{itemize}
            \item Store tuple together
            \item Divided into
                \begin{itemize}
                    \ides{Bitmap:} \todo{What is this?}
                    \ides{Fixed-Length:}
                        \begin{itemize}
                            \item Contains fixed-length fields
                            \item Arrangement and sizes are equal for all tuples
                            \item Can directly access the $i$-th field
                        \end{itemize}
                    \ides{Variable-Length:}
                        \begin{itemize}
                            \item Contains variable-length fields
                            \item Two implementation
                            \ides{Field Delimited:} Special characters mark the end/start of fields
                                \begin{itemize}
                                    \icon Access $i$-th fields requires scann of list
                                \end{itemize}
                            \ides{Field Offset Array:} Array where $A[i]$ contains the start of the $i$-th field
                                \begin{itemize}
                                    \item Stored at the beginning of the tuple
                                    \ipro Direct access to $i$-th field
                                \end{itemize}
                        \end{itemize}
                \end{itemize}
            \ipro All tuple information are together
            \ipro Good for OLTP
            \icon Bad for OLAP
                \begin{itemize}
                    \item Read lots of data we do not care
                \end{itemize}
        \end{itemize}
    \ides{Column Storage}
        \begin{itemize}
            \item Store a whole column together
            \ipro Good for OLAP
            \icon Slow for point queries (look for a single value), inserts, updates and deletes
            \icon Bad for OLTP
            \ipro Easier data compression
        \end{itemize}
\end{itemize}

\subsection{Buffer Pool Management}
\begin{itemize}
    \item Buffer manager acts like the intermediate layer between the system and disk manager
    \item On fetch, check if desired page is in RAM. If not, bring to RAM and returns. Else directly return it.
    \ides{Goal:} Provide illusion that all data in in RAM
    \item Page Replacement
        \begin{itemize}
            \item If RAM is full, we need to evict a page
            \item Eviction policy if of key importance
            \ides{Future access pattern known}
                \begin{itemize}
                    \ides{Idea:} Evict block whose next access is farthest in the future
                    \item Called Belady's MIN algorithm
                    \ipro Optimal under this assumption
                \end{itemize}
            \ides{Future access pattern unknown}
                \begin{itemize}
                    \item Rarely know about the future access pattern
                    \item Different strategies
                    \ides{Least Recently Used (LRU):} Evict the least recently used page
                        \begin{itemize}
                            \ipro Works well for repeated access to popular pages
                            \icon $100\%$ miss under sequential flooding
                                \begin{itemize}
                                    \ides{Sequential Flooding:} Access in a repeated pattern but such that not all pages fit into RAM
                                \end{itemize}
                            \item At most twice as bad compared to optimal when LRU has twice the memory \todo{What does this mean?}
                        \end{itemize}
                    \ides{Most Recently Used (MRU):} Evict the most recently used page
                        \begin{itemize}
                            \icon Frequently accessed page has to be fetched often
                                \begin{itemize}
                                    \item E.g. Index scan
                                \end{itemize}
                        \end{itemize}
                    \item Access patterns
                        \begin{itemize}
                            \ides{Sequential:} Table Scan: $1 \to 2 \to 3 \to 4 \to \dots$
                            \ides{Hierarchical:} Index Scan: $1 \to 4 \to 11 \to 1 \to 4 \to 12 \to 1 \to 3 \to 8 \dots$
                            \ides{Random:} Index Lookup: $12 \to 9 \to 4 \to 21 \to 55 \to 6 \to 42 \to \dots$
                            \ides{Cyclic:} Nested-Loop: $1 \to 2 \to 3 \to 4 \to 1 \to 2 \to 3 \to 4 \to \dots$
                        \end{itemize}
                    \item Depending on the pattern different strategies perform differently well
                    \item If we have some information about pattern, we can select an optimal policy
                        \begin{itemize}
                            \item This is the key different from the OS RAM manager
                        \end{itemize}
                \end{itemize}
        \end{itemize}
    \item OS vs DB
        \begin{itemize}
            \item Principles are similar, but in DB we know more what is going on
            \item Allows for better optimisation
            \item DB uses own buffer
            \item DB uses OS's filesystem
                \begin{itemize}
                    \item Has own buffer which can cause problems
                \end{itemize}
        \end{itemize}
\end{itemize}

\subsection{Access Methods}
\begin{itemize}
    \item Used by upper layer to access information in tables
    \item Provides different was of accessing data from a relation
    \item Goal: find a tuple whose attribute $X$ equals $Y$
    \ides{Sequential Scan:}
        \begin{itemize}
            \item Bring page 1 and scan, bring page 2 and scan etc.
            \item Cost:
                \begin{itemize}
                    \item When pages are sequentially allocated: $t_{s+r} + D(t_{tr})$
                    \item When pages are randomly allocated: $D(t_{s+r} + t_{tr}$
                    \item When write back on scan pay extra: $t_{tr}$
                \end{itemize}
        \end{itemize}
    \item Improve: build data structure $f$ over relation, which can easily answers our scan
        \begin{itemize}
            \item Evaluation of $f$ is usually cheaper than sequential scan
        \end{itemize}
    \ides{B-Tree (B+ Tree)}
        \begin{itemize}
            \item Self-balancing tree that keeps data sorted
            \item Properties
                \begin{itemize}
                    \item $M$-way search tree
                        \begin{itemize}
                            \item Like a binary search tree but each node has up to $M$ children
                        \end{itemize}
                    \item Perfectly balanced
                    \item Every inner node (except root) is at least half-full
                    \item Every inner node with $k$ keys has $k + 1$ non-null children
                    \item Each page is represented as a node
                    \item $\bigO(\log n)$ for search, insertion, deletion
                \end{itemize}
            \item Two flavours
                \begin{itemize}
                    \ides{Unclustered B+ Tree:} Leaf node contains RID (PageID, SlotID), pointing to the relation
                    \ides{Clustered B+ Tree:} Lead node contains the actual tuple
                        \begin{itemize}
                            \item Can have only one clustered B+ Tree per relation
                        \end{itemize}
                \end{itemize}
            \ides{Point query}
                \begin{itemize}
                    \item Find single key
                    \item Node size $= M$, $\#$ Tuples $= N$
                    \item Depth $\bigO(\log_M N)$
                    \item Num I/O
                        \begin{itemize}
                            \item Unclustered: $\underbrace{\log_M N}_{\text{search tree}} + \underbrace{1}_{\text{access actual tuple}}$
                            \item Clustered: $\underbrace{\log_M n}_{\text{search tree}}$
                        \end{itemize}
                    \item Much faster than sequential scan
                \end{itemize}
            \ides{Range query}
                \begin{itemize}
                    \item Find all tuples in range
                    \item Node size $= M$, $\#$ Tuples $= N$
                    \item Depth $\bigO(\log_M N)$
                    \item Num I/O
                        \begin{itemize}
                            \item Unclusered: $\underbrace{\log_M N}_{\text{search tree}} + \underbrace{\frac{\text{\#tuples}}{\text{tuples-per-page1}}}_{\#\text{ leafs to read to get all RIDs}} + \underbrace{\# \text{tuples}}_{\text{cost of reading}}$
                            \item Clustered: $\underbrace{\log_M N}_{\text{search tree}} + \underbrace{\frac{\text{\#tuples}}{\text{tuple-per-page2}}}_{\# \text{ leafs to read to bet all RIDs}}$
                            \item $\text{tuple-per-page2} < \text{tuple-per-page1}$ since clustered contains the actual tuples (which are larger than RID)
                        \end{itemize}
                \end{itemize}
            \ides{Insert}
                \begin{itemize}
                    \item Insert a tuple
                    \item Algorithm
                        \begin{itemize}
                            \item Find the right leaf node $L$
                            \item Put data in $L$
                                \begin{itemize}
                                    \item If $L$ has enough space we are done
                                    \item Otherwise, split $L$, insert key to the parent of $L$
                                \end{itemize}
                        \end{itemize}
                \end{itemize}
            \ides{Delete}
                \begin{itemize}
                    \item Delete a tuple
                    \item Algorithm
                        \begin{itemize}
                            \item Find the right leaf node $L$
                            \item Remove data in $L$
                                \begin{itemize}
                                    \item If $L$ is at least half full we are done
                                    \item Otherwise, merge two leaf nodes or borrow one tuple from neighbours, update parents
                                \end{itemize}
                        \end{itemize}
                \end{itemize}
            \item Heap file vs B+ Tree
                \begin{itemize}
                    \ides{Heap file:}
                        \begin{itemize}
                            \ipro Lot of sequential scans
                        \end{itemize}
                    \ides{B+ Tree:}
                        \begin{itemize}
                            \ipro Small number of random access
                        \end{itemize}
                    \item Tradeoff, between few expensive, or many cheap accesses
                \end{itemize}
            \item Use Btree: \verb+CREATE INDEX name ON table USING btree(column);+
            \item Bulk build
                \begin{itemize}
                    \item Different approaches
                    \item Insert tuple by tuple
                        \begin{itemize}
                            \icon Slow
                        \end{itemize}
                    \item Sort and insert tuple bottom-up
                \end{itemize}
            \ipro Query time is independent of data distribution
        \end{itemize}
    \ides{Hash Table}
        \begin{itemize}
            \item Goal: Do better than B+ Trees for point queries
            \ides{Hash Function $\mathbf{f(x)}$:} Maps each object into a entry in the table
                \begin{itemize}
                    \item E.g. or integer: $h(x) = (ax + b) \mod p$
                    \item E.g. or Strigs: $h(s_1, \dots, s_n) = (\sum_{i} s_ia^i) \mod p$
                    \item Hard to find ideal hash function
                \end{itemize}
            \item Ideally $a \neq b \implies f(a) \neq f(b)$
                \begin{itemize}
                    \item Else we get a collision
                \end{itemize}
            \item Different approaches to prevent collisions
            \ides{Closed Hashing:} We know how many elements are trying to index
                \begin{itemize}
                    \ides{Linear Probe}
                        \begin{itemize}
                            \item Working
                                \begin{itemize}
                                    \item Compute hash to find right slot
                                    \item Insert object into next empty slot
                                        \begin{itemize}
                                            \icon Deletion is non-trivial
                                                \begin{itemize}
                                                    \item We have to insert special marker into delete position to mark that this is a contiguous block
                                                \end{itemize}
                                        \end{itemize}
                                \end{itemize}
                            \item Search/Insert: $\bigO(\text{size of largest cluster})$
                                \begin{itemize}
                                    \ides{Cluster:} Largest consecutive sequence of occupied slots
                                \end{itemize}
                            \item For hash table:
                                \begin{itemize}
                                    \item Size $m$
                                    \item Contains $n = \lambda m$ keys
                                    \item Full to $\lambda \%$
                                \end{itemize}
                                on average:
                                \begin{itemize}
                                    \ides{Insert:} $\frac{1}{2} (1 + \frac{1}{(1 - \lambda)^2})$
                                \ides{(Successful) search:} $\frac{1}{2} (1 + \frac{1}{(1 - \lambda)})$
                                        \begin{itemize}
                                            \item Good if $\lambda \approx 50\%$ full
                                        \end{itemize}
                                \end{itemize}
                            \ipro Very cache-efficient
                            \icon Very sensitive to hash function
                                \begin{itemize}
                                    \item Hash function must be ``good''
                                \end{itemize}
                        \end{itemize}
                \end{itemize}
            \ides{Open Hashing:} We do not know how many elements are there
                \begin{itemize}
                    \ides{Chained Hashing}
                        \begin{itemize}
                            \item Working
                                \begin{itemize}
                                    \item Compute hash to find right slot
                                    \item Insert if no collision
                                    \item Else append to linked list
                                \end{itemize}
                            \item Expected length of chain for
                                \begin{itemize}
                                    \item $h(x)$ is uniformly random
                                    \item $m = \bigO(N)$
                                    \item $N$: Number of slots
                                \end{itemize}
                            is $\bigO(1)$
                        \end{itemize}
                \end{itemize}
        \end{itemize}
\end{itemize}

\subsection{Operator Execution}
\begin{itemize}
    \item Execute a relational algebra operator
    \item Used different access methods to implement these operators
    \ides{Select $\mathbf{\sigma_C(R)}$:}
        \begin{itemize}
            \item Input $R$, condition $C$
            \item Assume
                \begin{itemize}
                    \item $R$ has $|R|$ tuples and $B(R)$ pages
                    \item Buffer size: $M$ pages
                \end{itemize}
            \ides{Selectivity:} $\alpha(C, R)$
                \begin{itemize}
                    \item Is a constant
                    \item Number of tuples in $R$ that satisfy condition $C$ divided by $|R|$
                \end{itemize}
            \ides{Sequential Scan (Clustered) Heap File}
                \begin{itemize}
                    \item Bring pages to RAM one by one
                    \item Scan each tuple and check for the predicate $C$
                    \item If true, output tuple
                    \item Total cost: $\bigO(\underbrace{B(R)}_{\text{read}} + \underbrace{\alpha(C, R)B(R)}_{\text{write}})$
                    \item Cost is different for each query
                \end{itemize}
            \ides{Index Scan}
                \begin{itemize}
                    \item If $C \in \{<, >, =\}$
                    \ides{Unclustered B+ Tree}
                        \begin{itemize}
                            \item Steps
                                \begin{itemize}
                                    \item Find the right leaf node
                                    \item Scan the index
                                    \item Fetch and return corresponding tuple from heap file
                                \end{itemize}
                            \item Total Cost: $\bigO(\underbrace{\log |R|}_{\text{find leaf}} + \underbrace{\alpha(C, R) |R|}_{\text{fetch tuple}} + \underbrace{\alpha(C, R)B(R)}_{\text{write}})$
                        \end{itemize}
                    \ides{Clustered B+ Tree}
                        \begin{itemize}
                            \item Steps
                                \begin{itemize}
                                    \item Find the right leaf node
                                    \item Scan the index
                                    \item Return tuple
                                \end{itemize}
                            \item Total Cost: $\bigO(\underbrace{\log |R|}_{\text{find leaf}} + \underbrace{\alpha(C, R) B(R)}_{\text{fetch tuple}} + \underbrace{\alpha(C, R)B(R)}_{\text{write}})$
                        \end{itemize}
                \end{itemize}
        \end{itemize}
    \ides{Sort}
        \begin{itemize}
            \item \verb+Sort(R, Attribute A)+
            \ides{Clustered B+ Tree}
                \begin{itemize}
                    \item Sorted leaves and constant access time
                \end{itemize}
            \ides{Unclustered B+ Tree}
                \begin{itemize}
                    \item Sorted leafs but one random access per tuple
                \end{itemize}
            \item Different sorting algorithms
            \item Assume: Data $N$ is much larger than buffer $B$
            \ides{External Sort}
                \begin{itemize}
                    \item If $B = 3$ we can sort $2$ pages
                    \item Working
                        \begin{itemize}
                            \item If $N < B$
                                \begin{itemize}
                                    \item Run quicksort
                                \end{itemize}
                            \item Otherwise
                                \begin{itemize}
                                    \item Phase 1: Sorting
                                        \begin{itemize}
                                            \item Partition file into smaller chunks that fit into memory
                                            \item Sort smaller junks
                                        \end{itemize}
                                    \item Phase 2: Merging
                                        \begin{itemize}
                                            \item Combine smaller, sorted chunks into large file
                                        \end{itemize}
                                \end{itemize}
                        \end{itemize}
                    \item \todo{Add Cost}
                \end{itemize}
        \end{itemize}
    \ides{Join $\mathbf{S \bowtie_\theta S}$}
        \begin{itemize}
            \ides{Nested Loop Join}
                \begin{itemize}
                    \item
\begin{verbatim}
foreach tuple r in R:
    foreach tuple s in S:
        if Theta(r, s):
            output r, s
\end{verbatim}
                    \item $M = 3$: $\bigO(B(R) + |R| B(S) + \text{print IO})$
                    \item Order of tables matters
                        \begin{itemize}
                            \item Depends on table size, buffer size
                            \item Small $M$: Better if smaller relation is in outer loop
                            \item Large $M$ (assume smaller table is fully cached): Better if smaller relation is in the inner loop
                        \end{itemize}
                \end{itemize}
            \ides{Block Nested Loop Join}
                \begin{itemize}
                    \item
\begin{verbatim}
foreach block BR in R:
    foearch block BS in S:
        foreach tuple r in BR:
            foreach tuple s in BS:
                if Theta(r, s):
                    output r, s
\end{verbatim}
                    \item $M = 3$: $\bigO(B(R) + B(R) B(S))$
                    \item $M > 3$: $B(S) + B(R) \frac{B(S)}{M - 2})$
                    \item Partition $S$ into $a = \frac{B(S)}{M - 2}$ junks
                        \begin{itemize}
                            \item Fully cache junk
                            \item Pay $B(R)$ to join this junk with $R$: $(M - 2) + B(R)$
                            \item Repeat $\frac{B(S)}{M - 2}$ times \todo{Not sure what this means}
                        \end{itemize}
                \end{itemize}
            \ides{Sort Merge Join}
                \begin{itemize}
                    \item Assume both relations are sorted
                    \item Working
                        \begin{itemize}
                            \item Scan both relation
                            \item Compare to head
                        \end{itemize}
                    \item Cost $\bigO(B(R) + B(S) + \text{Sort}(R) + \text{Sort}(S))$
                \end{itemize}
            \ides{Hash Join}
                \begin{itemize}
                    \item \todo{Add algo}
                    \item Cost $\bigO(B(S) + B(R))$
                        \begin{itemize}
                            \item Assumed hash table fits into memory
                        \end{itemize}
                \end{itemize}
            \ides{Grace Hash Join}
                \begin{itemize}
                    \ipro Deals with the case where the hash table does not fit into memory
                    \item \todo{Add algo}
                    \item Cost $\bigO(3B(R) + 3B(S))$
                \end{itemize}
        \end{itemize}
\end{itemize}

\subsubsection{Query Optimizer}
\begin{itemize}
    \item Given a SQL query generate a good execution plan.
        \begin{itemize}
            \ides{Execution Plan:} Tree of relational algebra operators
        \end{itemize}
    \item Used query executor to actually execute the relational algebra operators
    \item Terminology
        \begin{itemize}
            \ides{Logical Plan:} What the user logically wants
            \ides{Physical Plan:} What the DMBS can understand and run
        \end{itemize}
    \item Steps
        \begin{itemize}
            \item User gives SQL query
            \item Parse SQL to logical plan
            \item Convert to physical plan
            \item Run each operator using the operator execution
        \end{itemize}
    \ides{Execution Model:} How different operators are gut together
        \begin{itemize}
            \item Different ways to put operators together
            \ides{Iterator Model}
                \begin{itemize}
                    \item Each operator is an iterator
                        \begin{itemize}
                            \ides{Input:} Set of streams of tuples
                            \ides{Output:} Stream of tuples
                            \item Calling \verb+next()+ on a iterator returns its result
                        \end{itemize}
                    \ides{Query Plan:} Is tree of iterators
                        \begin{itemize}
                            \item Result s returned by calling \verb+root.next()+ over and over
                        \end{itemize}
                    \ides{Volcano Model:} Data flow flow bottom to top
                    \ipro Generic interface for all operators
                    \ipro Easy to implement iterators
                    \ipro No overheads in terms of main memory
                    \ipro Supports pipelining
                    \ipro Supports parallelism and distribution
                    \icon High overhead of method calls
                    \icon Poor instruction cache locality
                \end{itemize}
            \ides{Materialization Model}
                \begin{itemize}
                    \item Each operator processes its inputs all at once and then emits its output all at once
                        \begin{itemize}
                            \ides{Input:} Full relation
                            \ides{Output:} Full relation
                        \end{itemize}
                    \item Good when the intermediate result is not too much larger than the final result
                        \begin{itemize}
                            \ipro Good for OLTP
                            \icon Bad for OLAP
                        \end{itemize}
                \end{itemize}
            \ides{Vectorization Model:}
                \begin{itemize}
                    \item Simian to iterator model
                    \item Each operator returns a batch of tuples
                        \begin{itemize}
                            \item Instead of a single tuple
                        \end{itemize}
                    \ipro Good for OLAP
                    \ipro Allows for vectored instructions to process batches of tuples
                \end{itemize}
        \end{itemize}
    \ides{Cost Model:} How to estimate the cost of each physical plan given our execution model
        \begin{itemize}
            \item Cannot be calculated but only estimate
            \item Estimated based on lower level (operator execution)
            \item Key variable to estimate is selectivity
                \begin{itemize}
                    \item Hard to estimate selectivity
                \end{itemize}
            \ides{Cardinality Estimation}
                \begin{itemize}
                    \item How many tuples does a query involve?
                    \ides{Histogram}
                        \begin{itemize}
                            \item Create histograms from the data
                            \item Lookup histograms to get an estimate of the number of tuples
                            \item Hard to combine multiple histograms into one
                                \begin{itemize}
                                    \item Due to missing correlation
                                    \item Assume they are independent
                                \end{itemize}
                            \item For continuous values use bins
                            \item We can also create multidimensional histograms
                        \end{itemize}
                    \item Can benefit form machine learning
                \end{itemize}
        \end{itemize}
    \ides{Space Space:} What are the logically equivalent sets of physical plans?
        \begin{itemize}
            \item Given an input logical plane, there are different was that one construct a physical plan
                \begin{itemize}
                    \item I.e. different orders of operators
                \end{itemize}
            \ides{Query rewriting Rules:} Set of transformations
            \ides{Input:} Relational algebra expression $E$
            \ides{Output:} Relational algebra expression $E'$
            \ides{Property:} $E$ is equivalent to $E'$
                \begin{itemize}
                    \item $\forall I \in \underbrace{\mathbf{I}}_{\text{Set of possible DB instances}}, E(I) = E'(I)$
                \end{itemize}
            \item Many different rules are possible
                \begin{itemize}
                    \item[1)] Conjunctive selection operations can be deconstructed into a sequence of individual selections
                        \begin{itemize}
                            \item $\sigma_{\theta_1 \wedge \theta_2}(E) = \sigma_{\theta_1}(\sigma_{\theta_2}(E))$
                        \end{itemize}
                    \item[2)] Selection operations are commutative
                        \begin{itemize}
                            \item $\sigma_{\theta_1}(\sigma_{\theta_2}(E)) = \sigma_{\theta_2}(\sigma_{\theta_1}(E))$
                        \end{itemize}
                    \item[3)] Only the last in a sequence of projections operations is needed
                        \begin{itemize}
                            \item $\Pi_{t_1}(\Pi_{t_2}(E)) = \Pi_{t_1}(E)$
                        \end{itemize}
                    \item[4)] Selections can be combined with Cartesian products and theta joins
                        \begin{itemize}
                            \item $\sigma_\theta(E_1 \times E_2) = E_1 \bowtie_\theta E_2$
                            \item $\sigma_{\theta_1}(E_1 \bowtie_{\theta_2} E_2) = E_1 \bowtie_{\theta_1 \wedge \theta_2} E_2$
                        \end{itemize}
                    \item[5)] Theta-join and natural join operations are commutative
                        \begin{itemize}
                            \item $E_1 \bowtie_\theta E_2 = E_2 \bowtie_\theta E_1$
                        \end{itemize}
                    \item[6)] Natural join operations are associative
                        \begin{itemize}
                            \item $(E_1 \bowtie E_2) \bowtie E_3 = E_1 \bowtie (E_2 \bowtie E_3)$
                        \end{itemize}
                    \item[6b)] Theta join operators are associative
                        \begin{itemize}
                            \item $(E_1 \bowtie_{\theta_1} E_2) \bowtie_{\theta_2 \wedge \theta_3} E_3 = E_1 \bowtie_{\theta_1 \wedge \theta_3}(E_2 \bowtie_{\Theta_2} E_3)$ when $\theta_2$ involves only attributes from $E_2$ and $E_3$
                        \end{itemize}
                    \item[7)] Pushdown Selection
                        \begin{itemize}
                            \item $\sigma_\theta(E_1 \bowtie E_2) = \sigma_\theta(E_1) \bowtie (E_2)$ if $\theta$ only involves attributes in $E_1$
                        \end{itemize}
                    \item[8)] The projections operation distributes over the theta join operation
                        \begin{itemize}
                            \item $\Pi_{L_1 \cup L_2}(E_1 \bowtie_\theta E_2) = (\Pi_{L_1}(E_1)) \bowtie_\theta(\Pi_{L_2}(E_2))$ if $\theta$ only involves attributes from $L_1 \cup L_2$
                        \end{itemize}
                    \item[9)] The set operations union and intersection are commutitive
                        \begin{itemize}
                            \item $E_1 \cup E_2 = E_2 \cup E_1$
                            \item $E_1 \cap E_2 = E_2 \cap E_1$
                        \end{itemize}
                    \item[10)] Set union and intersection are associative
                        \begin{itemize}
                            \item $(E_1 \cup E_2) \cup E_3 = E_1 \cup (E_2 \cup E_3)$
                            \item $(E_1 \cap E_2) \cap E_3 = E_1 \cap (E_2 \cap E_3)$
                        \end{itemize}
                    \item[11)] Selection operation distrubites over $\cup, \cap$ and $\setminus$
                        \begin{itemize}
                            \item $\sigma_\theta(E_1 \setminus E_2) = \sigma_\theta(E_1) \setminus \sigma_\theta(E_2)$
                            \item $\sigma_\theta(E_1 \cup E_2) = \sigma_\theta(E_1) \cup \sigma_\theta(E_2)$
                            \item $\sigma_\theta(E_1 \cap E_2) = \sigma_\theta(E_1) \cap \sigma_\theta(E_2)$
                            \item $\sigma_\theta(E_1 \setminus E_2) = \sigma_\theta(E_1) \setminus E_2$
                            \item $\sigma_\theta(E_1 \cap E_2) = \sigma_\theta(E_1) \cap E_2$
                            \item $\sigma_\theta(E_1 \cup E_2) \neq \sigma_\theta(E_1) \cup E_2$
                        \end{itemize}
                    \item[12)] Projection operation distributes over union
                        \begin{itemize}
                            \item $\Pi_L(E_1 \cup E_2) = (\Pi_L(E_1)) \cup (\Pi_L(E_2))$
                        \end{itemize}
                \end{itemize}
        \end{itemize}
    \ides{Search Algorithm:} How can we search the best physical plan, given cost model?
        \begin{itemize}
            \item Searching for the optimal query is hard
            \ides{Compromise 1:} Constraint search space
                \begin{itemize}
                    \item Only consider left-deep join trees
                    \item Allows fully pipelined plans
                        \begin{itemize}
                            \item Intermediate values do not need to be written to temporary files
                            \item Not all trees are fully pipelined (e.g. Sort Merge Join)
                        \end{itemize}
                    \item Process:
                        \begin{itemize}
                            \item Enumerate join orders
                                \begin{itemize}
                                    \item Only different left deep trees
                                \end{itemize}
                            \item Enumerate plans for each operator
                            \item Enumerate access method for each operator
                        \end{itemize}
                    \item Complexity $\bigO(n!), \ n$ is $\#$ relations
                    \item Can be done with DP
                \end{itemize}
            \ides{Compromise 2:} Heuristic-based Optimisation
                \begin{itemize}
                    \item Optimize query-tree by dappling a set of rules that typically improve execution performance
                        \begin{itemize}
                            \item Early selection
                            \item Early projection
                            \item Restrictive selections and joins before other similar operators
                        \end{itemize}
                    \item Search algorithm that enables pipelining
                \end{itemize}
        \end{itemize}
    \item \verb+EXPLAIN+ gives actual physical model
\end{itemize}

\subsection{Tuning}
\begin{itemize}
    \item DBMS have many internal parameters
    \item Improve performance
\end{itemize}
