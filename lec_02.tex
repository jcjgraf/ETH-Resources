%! TEX root = master.tex
\lecture{2}{Wed 16 Sep 2020 10:15}{}

Header files are external linked files. Ever program has to have \code{main} function which takes a several command line options 

\paragraph{Workflow}
Programmers write source code \code{.c} files, they are compiled into object files \code{.o}. Object files are linked with their external libraries \code{.a} and we get an executable file. There are also libraries (shared libraries) which are dynamically linked \code{.so} while running the code.\\
\code{gcc} is the compiler we use in all exercises in this lecure.

\paragraph{Compilation}

\begin{enumerate}
    \item C source \code{.c, .h}
    \item Macro substitution, include headers: cpp \code{.c}
    \item Compile each C file into assembly language: cc1 \code{.s}
    \item Assemble each file into object code: as \code{.o}
    \item Link object files into program binary: ld
    \item Executable
\end{enumerate}
We can stop the compilation at each stage.

\paragraph{Conditionals}
\begin{lstlisting}
if (Expression) Statement_when_true
else Statement_when_false
\end{lstlisting}

\begin{lstlisting}
switch (Expression) {
        case Constant_1: Statement; break;
        case Constant_2: Statement; break;
        ...
        default: Statement; break;
}
\end{lstlisting}
Switch is not equivalent to if, else if else if....

\paragraph{Loops}
\code{for (Initial; Test; Increment) Statement}
\code{while (Expression) Statement}
\code{do Statement while (Expression)}

\paragraph{Jumps}
\code{break} Break out of the current loop/switch
\code{continue} Stops current iteration of loop
\code{goto Label} Controversial, but occasionally very useful. Makes code hard to read.

\paragraph{Functions}
\code{type name(paraType paraName) {}}
\code{int main(int argc, char *argv[]) {return 0}}
\code{argc} is the number of parameter and \code{*argv} a list of string parameter.
\code{return (Expression)}
\code{*} indicates that a argument is a pointer.

\paragraph{I/O}
\code{man 3 printf} for docs. Forst argument is format string, remaining arguments are arbitrary but must match the format.

\paragraph{Declarations}
\code{type name;}
\code{type name = value;}

\begin{itemize}
    \item Inside block: Scope in block. When using \code{static} the value persists between function calls.
    \item Outside block: Scope of entire program. Wen using \code{static} the scope is limited to the file including its headers.
\end{itemize}

\paragraph{Types and Sizes}
\begin{table}[H]
    \centering
    \begin{tabular}{c | c}
        C data type & Intel x86-64\\
        \hline
        char & 1\\
        short & 2\\
        int & 4\\
        long & 8\\
        float & 4\\
        double & 8\\
        long double & 10/16\\
    \end{tabular}
\end{table}

Integers are signed by default. Use \code{signed} or \code{unsigned} to clarify.

\code{<stdint.h>} provides extended integer types:
\begin{itemize}
    \item Signed: \code{int8\_t},\code{int16\_t}, \code{int32\_t}, \code{int64\_t}
    \item Unsigned: \code{uint8\_t}, \code{uint16\_t}, \code{uint32\_t}, \code{uint64\_t}
\end{itemize}

\paragraph{Integers and Floats}
Conversion is complicated.
\begin{itemize}
    \item Implicit conversation between integer types.
    \item Implicit conversation between floating point types.
    \item Explicit conversation between anything.
\end{itemize}
Behaviour is determined by hardware.

\paragraph{Booleans}
Booleans are integers. Zero means false, anything else means true. However, C99 supports real boolean types via \code{<stdbool.h>}. Negating zero gives non-zero, and non-zero return zero.
Any C statement is also an expression (also counts for loops etc...).
\begin{lstlisting}
int rc;
if (rc = myFunc()) {
    # error
} else {
    // everything good
}
\end{lstlisting}

Sometimes we need to invert the return, e.g. of a file open.

\paragraph{void}
\code{void} is a type which has no value. Pointers have typically a type assigned, void is used for untyped pointers. We also use it for declaring functions without return value.


