%! TEX root = ./main.tex

\section{Integrity Constraints}
\begin{itemize}
    \item Additional constraint to the key and domain constraint
    \item Makes sure changes are consistent
    \item Control the content of the date and its consistency
    \item Are enforced by the schema
    \item Can be defined when:
        \begin{itemize}
            \item Creating the table (\verb+Create table+)
            \item Later (\verb+alter table+)
        \end{itemize}
\end{itemize}

\subsection{Types}
\begin{itemize}
    \item Checked for \verb+INSERT+ as well as \verb+UPDATE+
        \begin{itemize}
            \item For foreign key also on \verb+DELETE+
        \end{itemize}
    \item Check happen at tuple level and not at the semantic of the command
    \item Some check may fail or succeed depending on the order of the tuples
        \begin{itemize}
            \item We have no influence on this
        \end{itemize}
    \ides{NOT NULL:}
        \begin{itemize}
            \item Prevents attribute from being \verb+NULL+
        \end{itemize}
    \ides{PRIMARY KEY:}
        \begin{itemize}
            \item Mark attribute as primary key
            \item Must not be \verb+NULL+ and not empty
            \item If applied to a tuple, all field must not be \verb+NULL+
        \end{itemize}
    \ides{UNIQUE}
        \begin{itemize}
            \item In contrast to key, it can be \verb+NULL+ (also multiple \verb+NULL+s for one field)
            \item Multiple fields can be marked as unique
            \item Tuples of fields can be marked as unique
        \end{itemize}
    \ides{CHECK:}
        \begin{itemize}
            \item Boolean check based on values of a single tuple
            \item Reject if \verb+False+
            \item Accept if \verb+True+ or \verb+Unknown+
            \item Some engines treat check somewhat weirdly
        \end{itemize}
    \ides{FOREIGN KEY:}
        \begin{itemize}
            \item Used keyword \verb+references+
            \item Involve two relations
            \item Field must be \verb+NULL+ or a valid reference to another table
            \item The reference field is often a \verb+PRIMARY KEY+ or at least \verb+UNIQUE+
            \ides{Referencing Table:} Table which references a tuple form another table
            \ides{Referenced Table:} Table being referenced by another table
        \end{itemize}
    \item Using \verb+constraint <name>+ we can give a name to constraints
        \begin{itemize}
            \item Useful in practice since it allows easy modification later on
        \end{itemize}
\end{itemize}

\subsection{Maintenance}
\begin{itemize}
    \item For \verb+REFERENTIAL+
    \item Changes to the referenced table influences the referencing table
        \begin{itemize}
            \item And not the other way around!
            \item On \verb+UPDATE+ or \verb+DELETE+
        \end{itemize}
    \ides{Cascade:}
        \begin{itemize}
            \item Propagate modification or delete
        \end{itemize}
    \ides{Restrict:}
        \begin{itemize}
            \item Prevent deletion of primary key before trying to do the change
            \item Throw error immediately
        \end{itemize}
    \ides{No Action:}
        \begin{itemize}
            \item Prevent the modification after attempting the change
            \item Throw error at the end of the transaction
            \item Is the default of PostgreSQL
            \item Is equivalent to restrict in mySQL
        \end{itemize}
    \ides{Set Default/Set Null:}
        \begin{itemize}
            \item Set reference to default value or \verb+NULL+
        \end{itemize}
    \item Is specified with \verb+on update+ and \verb+on delete+
\end{itemize}
