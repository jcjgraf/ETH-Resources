%! TEX root = ./main.tex

\section{Introduction}
\subsection{Functional Programming}
\begin{itemize}
    \item Like mathematical expressions
    \item Consists of functions and values
    \item Functions are actually values themselves
    \item There is no state
    \ides{Referential Transparency:} Not state $\implies$ expression \textbf{always} evaluate to the same value
    \item No global variables
    \item Recursion instead of iteration
    \ipro Easy to parallelize
    \ipro Easy to analyze
    \ipro Flexible type system
\end{itemize}

\subsection{Haskell}
\begin{itemize}
    \ides{Lazy Evaluation:} expression evaluates always outermost and leftmost expression
        \begin{itemize}
            \item But pattern matching and some other functions force evaluation
        \end{itemize}
\end{itemize}

\subsubsection{Syntax}
\begin{itemize}
    \ides{Function}
        \begin{itemize}
            \item Function name and arguments start with lower-case
            \item Expression after the equal sign is the return value
            \ides{Pattern Matching}
                \begin{itemize}
                    \item Is used for:
                        \begin{itemize}
                            \item Check if argument has proper type
                            \item Bind values to variables
                        \end{itemize}
                    \ides{Pattern}
                        \begin{itemize}
                            \item Inductively defined
                            \item Pattern are
                                \begin{itemize}
                                    \item Constants
                                    \item Variables
                                    \item Wild Card (\verb+_+)
                                    \item Tuples $(p_1, p_2, ... , p_k)$ where $p_i$ is a pattern
                                    \item Non-Empty Lists $(p_1 : p_2)$ where $p_i$ is a pattern
                                \end{itemize}
                            \item Must be \textbf{linear}
                                \begin{itemize}
                                    \item I.e. each variable cannot occur more than once
                                    \item Does not count for wild card
                                \end{itemize}
                        \end{itemize}
                    \ides{Pattern Matching}
                        \begin{itemize}
                            \item Pattern matching is used to determine right definition
                            \item Pattern \verb+p+ matches term \verb+a+ by the following recursion on \verb+p+:
                                \begin{itemize}
                                    \ides{Constant:} \verb+p = c+ if \verb+c = a+
                                    \ides{Variable:} \verb+p = x+ always succeeds with binding \verb+x = a+
                                    \ides{Wild Card:} \verb+p = _+ always success but without binding
                                    \ides{Tuple:} $p = (p_1, \dots , p_k)$ succeeds if $a = (a_1, \dots , a_k)$ and $p_i$ matches $a_i \quad \forall i \in \{1, \dots , k\} $
                                    \ides{Non-Empty List:} $p = (p_1 : p_2)$ succeeds if $a = (a_1 : a_2)$ and $p_i$ matches $a_i \quad \forall i \in \{1, 2\}$
                                \end{itemize}
                            \item Forces evaluation (no longer lazy evaluation)
                            \item Can define the same function multiple times but with different patterns
                        \end{itemize}
                \end{itemize}
            \item May Contain several cases (\textit{guards})
                \begin{itemize}
                    \item Boolean expression
                    \item \verb+otherwise+ is the default case
                \end{itemize}
            \ides{Scope}
                \begin{itemize}
                    \item Functions have a global scope
                    \item Order of declaration does not matter
                    \item \verb+let <local func, var, const decl.> in <expr. using these defs>+
                        \begin{itemize}
                            \item More powerful than \verb+where+
                        \end{itemize}
                    \item \verb+where <local func, var, const decl.>+
                        \begin{itemize}
                            \item Follows guard or function return
                            \item Top-Down development (use and then declare)
                        \end{itemize}
                \end{itemize}
        \end{itemize}
    \item Constants can be defined outside of functions
    \item Program consists of multiple function definitions
    \ides{Indentations}
        \begin{itemize}
            \item Determines separation of definitions
            \item All function definitions start at the same indentation
            \item The body of a function definition needs to be indented
            \item If line is split into two, indent new line again
                \begin{itemize}
                    \item Can be done recursively
                \end{itemize}
            \item Spaces have to be used, not tabs
        \end{itemize}
\end{itemize}

\subsubsection{Types}
\begin{itemize}
    \item Strongly typed
    \item Can explicitly define function definition or let Haskell do that
    \ides{Integral}
        \begin{itemize}
            \ides{Int:} Bound
            \ides{Integer:} Arbitrary precision
        \end{itemize}
        \ides{Double}
    \ides{Char}
        \begin{itemize}
            \item Surrounded by \verb+'+
        \end{itemize}
    \ides{String}
        \begin{itemize}
            \item List of characters
            \item Surrounded by \verb+"+
            \item Concatenate using \verb|++|
        \end{itemize}
    \ides{Bool}
    \ides{Function/Operator}
        \begin{itemize}
            \ides{Operator:} Binary function which is used infix
            \item \verb+`func`+ makes function infix
            \item \verb+(op)+ makes operator prefix
        \end{itemize}
    \ides{Tuple}
        \begin{itemize}
            \item Compose multiple values of different type
            \item Composed by a \textit{Type Constructor}
            \item If $T_1, \dots, T_n$ are Types, then $(T_1, \dots , T_n)$ is a tuple type
            \item If $v_1 :: T_1, \dots, v_n :: T_n$ are values of matching type, then $(v_1, \dots , v_n) :: (T_1, \dots , T_n)$ is a valid tuple
            \item Can be nested
        \end{itemize}
\end{itemize}

\subsubsection{Input/Output}
\begin{itemize}
    \item I/O is not referential transparent (has side effects)
    \item Wrap by \verb+IO+ to capture side effects
    \item \verb+getLine :: IO String+ reads a string
    \item \verb+putStrLn :: String -> IO ()+ prints a string.
    \item \verb+do+ blocks sequences side effects
    \item \verb+<-+ extract values from IO
    \item \verb+return+ wraps values in IO
\end{itemize}
