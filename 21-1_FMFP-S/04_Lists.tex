%! TEX root = ./main.tex

\section{Lists}
\subsection{Introduction}
\begin{itemize}
    \ides{List Type:} If \verb+T+ is a type then \verb+[T]+ is a type
        \begin{itemize}
            \item Is a new type constructor
        \end{itemize}
    \ides{Empty List:} \verb+[] :: [T]+
    \ides{Non-Empty List:} \verb+(x : xs) :: [T]+ iff \verb+x :: T+ and \verb+xs :: [T]+
        \begin{itemize}
            \ides{Cons Operator:} \verb+:+ prepends an element to a list
                \begin{itemize}
                    \ides{Concatenate:} \verb|++| concatenates two lists
                \end{itemize}
            \item \verb+[a, b, c]+ is syntactic sugar for \verb+a : (b : (c : []))+
        \end{itemize}
    \item \verb+[n, p .. m]+ constructs a list from \verb+n+ to \verb+m+ with step \verb+p - n+
        \begin{itemize}
            \item \verb+p+ is optional; Default step is \verb+1+
            \item Can be seen as, ``First element is \verb+n+, second element is \verb+p+, continue like this till \verb+m+''
            \item \verb+m+ is not necessarily included
        \end{itemize}
    \item String is a list of chars
        \begin{itemize}
            \item \verb+['a', 'b'] == "ab"+
        \end{itemize}
\end{itemize}

\subsection{Sorting Algorithms}
\begin{itemize}
    \ides{Insertion Sort}
\begin{verbatim}
isort :: [Int] -> [Int]
isort [] = []
isort (x: xs) = ins x (isort xs)

ins :: Int -> [Int] -> [Int]
ins a [] = [a]
ins a (x: xs)
    | a <= x = a : (x: xs)
    | otherwise = x : ins a xs
\end{verbatim}
    \ides{Quick Sort (long form)}
\begin{verbatim}
qsort :: [Int] -> [Int]
qsort [] = []
qsort (x: xs) = qsort ( lesseq x xs) ++ [x] ++ qsort (greater x xs)
where
    lesseq _ [] = []
    lesseq x (y: ys)
        | y <= x = y : lesseq x ys
        | otherwise = lesseq x ys
    greater _ [] = []
    greater x (y: ys)
    | y >x = y : greater x ys
    | otherwise = greater x ys
\end{verbatim}
    \ides{Quick Sort (short form)}
\begin{verbatim}
q :: [Int] -> [Int]
q [] = []
q (p:xs) = q [x | x <- xs, x <= p] ++ [p] ++ q [x | x <- xs, x > p]
\end{verbatim}
\end{itemize}

\subsection{List Comprehension}
\begin{itemize}
    \item Notation for sequential processing of list elements
    \item Analogous to set comprehension in set theory
    \item General form: \verb+[func x | <gen_1>, ... , <gen_n>, <pred_1, ..., <pred_m>]+
    \item Generators can depend on each other
        \begin{itemize}
            \item E.g. \verb+[x | n <- [1..10], x <- [1..n]]+
        \end{itemize}
    \item Generators can depend on if then else
        \begin{itemize}
        \item E.g. \verb+[x | n <- [1..10], if even x then x <- [1,2] else x <- [1]]+
        \end{itemize}
    \item \todo{Add more handy dandy examples}
\end{itemize}

\subsection{Induction Over Lists}
\begin{itemize}
    \item Prove $P$ for all $xs$ in $[T]$
        \begin{itemize}
            \ides{Base Case:} prove $P[xs/[]]$
            \ides{Step Case:} prove $\forall y :: T, ys :: [T]. P[xs/ys] \to P[xs/y:ys]$
                \begin{itemize}
                    \ides{Fix} arbitrary but non-free $y :: T, ys :: [T]$
                    \ides{Induction Hypothesis:} Assume $P[xs/ys]$
                \end{itemize}
        \end{itemize}
    \item Sometimes hard to pick right induction variable
        \begin{itemize}
            \item Proof may fail depending on the variable
        \end{itemize}
    \ides{Generalisation}
        \begin{itemize}
            \item Proof a stronger statement as a subproof
            \item Required for some proofs
        \end{itemize}
\end{itemize}
