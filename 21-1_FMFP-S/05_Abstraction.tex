%! TEX root = ./main.tex

\section{Abstraction}
\begin{itemize}
    \ides{Polymorphic Type t:} A set of types
    \ides{Parametric Polymorphism:} Function works for type \verb+t+ iff it works for all types contained in \verb+t+
    \item A type \verb+w+ for function \verb+f+ is a \textbf{most general} (/principal) \textbf{type} iff for all types \verb+s+ for \verb+f+, \verb+s+ is an instance of \verb+w+.
    \item Given a function, Haskell always computes the most general type
        \begin{itemize}
            \item If we give a type, it must be an instance of the most general type
        \end{itemize}
    \item Type variables start with lower-case
\end{itemize}

\subsection{Higher-Order Functions}
\begin{itemize}
    \item Types
        \begin{itemize}
            \ides{First Order:} Arguments are base types or constructor types
                \begin{itemize}
                    \item \verb+Int -> [Int]+
                \end{itemize}
            \ides{Second Order:} Arguments are themselves functions
                \begin{itemize}
                    \item \verb+(Int -> Int) -> [Int]+
                \end{itemize}
            \ides{Third Order:} Arguments are functions, whose arguments are functions
                \begin{itemize}
                    \item \verb+((Int -> Int) -> Int) -> [Int]+
                \end{itemize}
            \ides{Higher-Order:} Functions of arbitrary order
        \end{itemize}
    \item Advantages
    \ipro Definition is easier to understand
    \ipro Parts are easier to modify
    \ipro Pars are easier to reuse
    \ipro Correctness is simpler to understand and show
\end{itemize}

\subsubsection{Examples}
\begin{itemize}
    \ides{Map}
        \begin{itemize}
            \item Apply function to each argument in a list
            \item
\begin{verbatim}
map :: (a -> b) -> [a] -> [b]
map f [] = []
map f (x:xs) = f x : map f xs
\end{verbatim}
        \end{itemize}
    \ides{Folding}
        \begin{itemize}
            \item Aggregate all elements of a list
            \ides{foldr}
                \begin{itemize}
                    \item Written as \verb+(f x_1 (f x_2 (f ... (f x_k e)))+ for list $x$, function $f$ and default value $e$
                    \item When seen as a tree, the con is replaced by $f$ and the empty list by $e$
                    \item Can operate on infinite list
\begin{verbatim}
foldr :: (a -> b -> b) -> b -> [a] -> b
foldr f e [] = e
foldr f e (x: xs) = f x (foldr f e xs)
\end{verbatim}
                    \ides{Recipe:} Implement some (suitable) function with folder
                        \begin{itemize}
                            \item Identify the following arguments:
                                \begin{itemize}
                                    \ides{Recursive Arguments:} The list which shrinks in each iteration
                                    \ides{Static Arguments:} The ones which do not change
                                    \ides{Dynamic Arguments:} The ones which change arbitrarily
                                \end{itemize}
                            \item Write a helper function \verb+aux+ with all recursive and then dynamic arguments
                            \item Move the dynamic arguments to the right of the equals
                                \begin{itemize}
                                    \item I.e. form a lambda function
                                    \item I.e. $\eta$-expansion
                                \end{itemize}
                            \item Rewrite the helper function using \verb+foldr+ and replace \verb+aux xs+ with local variable \verb+rec+
                            \item Inline the helper function
                        \end{itemize}
                \end{itemize}
            \ides{foldl}
                \begin{itemize}
                    \item Written as \verb+f(f(f(f e x_1) x_2) ...) x_k+ for list $x$, function $f$ and default value $e$
                    \item Runs infinitely on infinite lists
                    \item
\begin{verbatim}
foldl :: (b -> a -> b) -> b -> [a] -> b
foldl f e [] = e
foldl f e (x: xs) = foldl f ( f e x) xs
\end{verbatim}
                \end{itemize}
            \item foldr and foldl are equivalent for associative functions
        \end{itemize}
\end{itemize}

\subsection{$\lambda$-Expression}
\begin{itemize}
    \item Allows in-line function definitions
    \item Constructed as \verb+(\v_1 -> ... -> \v_k -> <someExpression>)+
        \begin{itemize}
            \item Syntactic sugar \verb+(\v_1 ... v_k -> <someExpression>)+
        \end{itemize}
    \item Adoption of Church's $\lambda$-notation
    \ides{$\mathbf{\eta}$-Conversion:} \verb+x -> f x+ and \verb+f+ are equivalent
        \begin{itemize}
            \ides{$\eta$-Contraction:} From left to right
            \ides{$\eta$-Expansion:} From Right to left
            \item Useful to simplify expression
        \end{itemize}
\end{itemize}

\subsection{Function as Values}
\begin{itemize}
    \item Function itself can be returned from function
    \item Returned function cannot be displayed, but only evaluated
\end{itemize}

\subsubsection{Examples}
\begin{itemize}
    \ides{Function Composition}
        \begin{itemize}
            \item Takes two functions as arguments and returned the composite function
            \item Application associates to the left
            \item
\begin{verbatim}
(.) :: (b -> c) -> (a -> b) -> (a -> c)
(f . g) x = f (g x)
\end{verbatim}
        \end{itemize}
    \ides{Iteration}
        \begin{itemize}
            \item Apply a function \verb+a -> a+ $n$ times to a input $x$
            \item
\begin{verbatim}
iter :: Int -> (a -> a) -> a -> a
iter 0 f x = x
iter n f x = f (iter (n - 1) f x)
\end{verbatim}
        \end{itemize}
    \ides{Difference Lists}
        \begin{itemize}
            \ides{Problem:} Appending to list is expensive
            \ides{Idea:} Construct list as a higher order (first) function
            \item
\begin{verbatim}
type DList a = [a] -> [a]

empty :: DLists a
empty = \xs -> xs

sngl :: a -> DList a
sngl x = \xs -> x : xs

app :: DList a -> DList a -> DList a
ys `app` zs = \xs -> ys (zs xs)

fromList :: [a] -> DList a
fromList ys = \xs -> ys ++ xs

toList :: DList a -> [a]
toList ys = ys []
\end{verbatim}
        \end{itemize}
\end{itemize}


\subsection{Function Arguments}
\begin{itemize}
    \ides{Partial Application}
        \begin{itemize}
            \item One applies only some but not all arguments
            \item A new function, still requiring some arguments, is returned
            \item Useful for \verb+map+, \verb+filter+ etc.
            \item If $f :: t_1 \to t_2 \to \dots \to f_n \to t$ and $e_1 :: t_1, \dots , e_k :: t_k$ then the partial application has type $f e_1 \dots e_k :: t_{k + 1} \to \dots \to t_n \to t$
            \item Partial application is consistent with the view that function takes multiple arguments
                \begin{itemize}
                    \item But a function takes exactly one arguments
                \end{itemize}
            \item For infix operator $\oplus$:
                \begin{itemize}
                    \item $(a \oplus) \equiv \lambda x. a \oplus x$
                    \item $(\oplus a) \equiv \lambda x. x \oplus a$
                    \item Importend to consider when operator is not commutative
                        \begin{itemize}
                            \item \verb+(a `func`) != (`func` a)+
                        \end{itemize}
                \end{itemize}
        \end{itemize}
    \ides{Tupling}
        \begin{itemize}
            \item Wrapping multiple arguments into tuple lets us apply them as one argument
            \item Function is one of:
                \begin{itemize}
                    \ides{Curry Func:} Takes multiple arguments
                    \ides{Uncury Func:} Takes a tuple as argument
                \end{itemize}
            \item We want convert one representation to the other using:
                \begin{itemize}
                    \ides{Curry:} Uncurry $\to$ curry
\begin{verbatim}
curry :: ((a,b) -> c) -> a -> b -> c
curry f = f’ where f’ x1 x2 = f (x1,x2)
\end{verbatim}
                    \ides{Uncurry:} Curry $\to$ uncurry
\begin{verbatim}
uncurry :: (a -> b -> c) -> (a,b) -> c
uncurry f’ = f where f (x1,x2) = f’ x1 x2
\end{verbatim}
                \end{itemize}
        \end{itemize}
    \ides{Uncluttering Notation}
        \begin{itemize}
            \item Right associative operator \verb+$+ for arguments
            \item Avoids parentheses
        \end{itemize}
\end{itemize}
