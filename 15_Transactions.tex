% ! TEX root = ./main.tex

\section{Transaction}
\begin{itemize}
    \ides{Motivation}
        \begin{itemize}
            \item Assume:
                \begin{itemize}
                    \item DB is a collection of objects
                        \begin{itemize}
                            \item I.e. one tuple is one object
                        \end{itemize}
                    \item Objects are fixed
                        \begin{itemize}
                            \item We cannot create new ones or delete old ones
                        \end{itemize}
                    \item System has only a single CPU
                        \begin{itemize}
                            \item CPU can only run one instruction at the time
                        \end{itemize}
                \end{itemize}
            \item If not dealt with correctly, simultaneous transactions may get mixed and we get wrong results
            \ides{Concurrent DB Access}
                \begin{itemize}
                    \ides{Schedule:} One way of mixing instructions
                        \begin{itemize}
                            \item Different schedules may result in different results
                        \end{itemize}
                    \item Result of one query may be overwritten partly or completely
                    \ides{Attribute-level Inconsistency:} Concurrent change of a full object
                    \ides{Tuple-level Inconsistency:} Concurrent change of part of an object
                    \ides{Table-level Inconsistency:} Concurrent change of full relation
                    \ides{Multi-statement Inconsistency:} Interleaving of concurrent queries
                    \item When multiple groups of SQL statements are running at the same time, we want the effect as if they are executed sequentially
                \end{itemize}
            \ides{System Failure}
                \begin{itemize}
                    \item Many thing which can break in a real system
                    \item We want that all or none changes apply, but not partial application
                \end{itemize}
        \end{itemize}
    \ides{Transaction:} Collection of instructions which should not mix with other transactions
        \begin{itemize}
            \item Concurrent transactions appear to run in isolation
            \item On a crash, transaction changes appear entirely or not at all
            \item \verb+BEGIN; ..... COMMIT+: Encapsulates a transaction
                \begin{itemize}
                    \item Transaction has finished, database confirms to client whan all changes of the transaction have been made persistent
                    \item Transaction may also fail. Database rollback all changes done by the transaction
                        \begin{itemize}
                            \item Written as \verb+BEGIN; ..... ABORT;+
                            \item Can be initiated by the user or DBMS
                        \end{itemize}
                \end{itemize}
            \ides{Autocommit:} Turns each SQL into a transition
                \begin{itemize}
                    \item Activated by default, can be deactivated
                \end{itemize}
        \end{itemize}
\end{itemize}

\subsection{ACID:}
\begin{itemize}
    \item Desired properties of transaction
    \ides{Atomicity}
        \begin{itemize}
            \item Transaction is executed in its entirety or not at all
        \end{itemize}
    \ides{Consistency}
        \begin{itemize}
            \item A committed transaction goes from one consistent state to another consistent state
                \begin{itemize}
                    \item Before and after a transaction all integrity constraints must hold
                \end{itemize}
            \item Within a transaction constraints may be violated
            \item Transaction leads from consistent state to consistent state
            \item Granularity depends on the integrity constrains
                \begin{itemize}
                    \item I.e. some constrains re checked at commit level some at \todo{???}
                    \item Can be controlled and influenced to some degree
                \end{itemize}
        \end{itemize}
    \ides{Isolation}
        \begin{itemize}
            \ides{Ideally:} Transaction executes as if it were alone in the system
                \begin{itemize}
                    \item I.e. enforce serializability
                    \item Much to hard to enforce
                \end{itemize}
            \item Implies that integrity constraints always hold if each transaction is correct
            \item DMDB picks one execution order at random (if there are multiple)
                \begin{itemize}
                    \item If not desired, application must enforce this
                \end{itemize}
        \end{itemize}
    \ides{Durability}
        \begin{itemize}
            \item If system crashes after a transaction, the changes of the transaction must still remain in the DB
                \begin{itemize}
                    \item Or somehow recoverable
                \end{itemize}
        \end{itemize}
\end{itemize}

\subsection{Isolation}
\begin{itemize}
    \item One of the key properties
    \item DBMS provides different levels of isolation
        \begin{itemize}
            \ides{Anomaly:} Misbehaviour of the DB
                \begin{itemize}
                    \ides{Dirty Read:} If we read from a different transaction which was not yet committed we have a dirty read
                        \begin{itemize}
                            \item May contain values which were/are never in the DB
                                \begin{itemize}
                                    \item When the other transaction aborts
                                \end{itemize}
                        \end{itemize}
                    \ides{Non-repeatable Reads:} Reading the same tuple twice any we may not get the same value both times
                    \ides{Phantoms:} Change to tuple which are only visible during some part of the transaction
                \end{itemize}
            \ides{Isolation Level:} Defines for each transaction what isolation level it should have
                \begin{itemize}
                    \item Each level is defined by the anomalous behaviour that is allowed
                \end{itemize}
            \item $\left.\left.
\begin{tabular}{l | c | c | c}
                     & Dirty Reads  & Non-Repeatable Reads & Phantoms\\\hline
Read Uncommitted     & $\checkmark$ & $\checkmark$        & $\checkmark$\\
Read Committed       & $\times$     & $\checkmark$        & $\checkmark$\\
Repeatable Read      & $\times$     & $\times$            & $\checkmark$\\
Serializable         & $\times$     & $\times$            & $\times$
\end{tabular} \right\downarrow \rotatebox[origin=c]{90}{overhead}\ \right\uparrow\rotatebox[origin=c]{90}{concurrency}$
        \end{itemize}
\end{itemize}

\subsection{More on Serializable}
\begin{itemize}
    \ides{Serializable:} Schedule that leads to the same answer as some serial schedule
        \begin{itemize}
            \item Only depends on final result and not I/O pattern along
            \item Not all sequential orders necessarily lead to the same result
            \icon Hard or impossible to enforce
        \end{itemize}
    \ides{Conflicts}
        \begin{itemize}
            \ides{Def Conflict}
                \begin{itemize}
                    \ides{Same Transaction:}
                        \begin{itemize}
                            \item Two operations are always conflicting
                            \item $\implies$ Reordering within transaction is not allowed
                        \end{itemize}
                    \ides{Different Transaction $O_1$ in $T_1$ and $O_2$ in $T_2$:}
                        \begin{itemize}
                            \item $O_1$ and $O_2$ are conflicting if one of them is a write to the same location
                        \end{itemize}
                \end{itemize}
            \ides{Types}
                \begin{itemize}
                    \ides{Read-Write:}
                        \begin{itemize}
                            \item Leads to unrepeatable reads
                        \end{itemize}
                    \ides{Write-Read:}
                        \begin{itemize}
                            \item Leads to dirty read
                        \end{itemize}
                    \ides{Write-Write:}
                        \begin{itemize}
                            \item Leads to overwriting of uncommitted data
                        \end{itemize}
                \end{itemize}
        \end{itemize}
    \ides{Conflict Equivalent} are two schedules iff:
        \begin{itemize}
            \item They involve the same actions on the same transactions
            \item Every pair of conflicting actions is ordere in the same way
        \end{itemize}
    \ides{Conflict Serializable:} Schedule if it is conflict equivalent to some serial schedule
        \begin{itemize}
            \item I.e. schedule which can be translated into a serial schedule with a sequence of non-conflicting swaps of adjacent actions
            \item Stronger than serializable
            \item Only depends on the read/write pattern
                \begin{itemize}
                    \item And not what we are writing
                \end{itemize}
            \item Easier than serializability for DB to handle as it does require the DB to understand what each operator is doing
            \item Enforced by most DBMSs
            \ides{Decide}
                \begin{itemize}
                    \item Each transaction is a node
                    \item $\exists$ edge $T_i$ to $T_j$ if
                        \begin{itemize}
                            \item Operator $o_i$ in $T_i$ is in conflict with operator $o_j$ in $T_j$
                            \item $o_j$ appears earlier than $o_i$ in same transaction
                        \end{itemize}
                    \item Schedule is conflict serializable iff its dependency graph is acyclic
                \end{itemize}
        \end{itemize}
    \item Serializability and conflict serializability only concern committed transactions
        \begin{itemize}
            \item Operations can be in conflict with \verb+ABORT+
                \begin{itemize}
                    \item E.g. \verb+READ(X)+/ \verb+WRITE(X)+ before or after \verb+ABORT+ of other transaction may lead to different results (only if the other transaction did \verb+WRITE(X)+)
                \end{itemize}
        \end{itemize}
\end{itemize}

\subsection{Enforce Isolation}
\begin{itemize}
    \ides{Goal:} Only allow schedules that are conflict serializable
    \item Two main approaches
        \begin{itemize}
            \ides{Pessimistic:} Assume that conflicts happen all the time
                \begin{itemize}
                    \item Use locks
                \end{itemize}
            \ides{Optimistic:} Assume that most transaction do not conflicts
                \begin{itemize}
                    \item Use snapshot isolation
                \end{itemize}
        \end{itemize}
    \item Need to evaluate which approach is better for our application
\end{itemize}

\subsubsection{Locking}
\begin{itemize}
    \ides{Assume:} Do not know what transactions are going to do in the future
    \ides{Idea:} Before the system access the data object $X$ it locks $X$
        \begin{itemize}
            \item Prevents access of $X$ by other transaction
        \end{itemize}
    \item Lock is released only when it is save to
        \begin{itemize}
            \item I.e. the execution is guaranteed to be conflict serializable
        \end{itemize}
    \item Allows to enforce serializable schedule
    \item Types
        \begin{itemize}
            \ides{Shared Lock (S Lock):}
                \begin{itemize}
                    \item For reading
                \end{itemize}
            \ides{Exclusive Lock (X Lock):}
                \begin{itemize}
                    \item For writing
                \end{itemize}
        \end{itemize}
    \item Does not necessarily enforce conflict serializability
    \ides{Two-Phase Locking (2PL)}
        \begin{itemize}
            \item Consists of two phases
                \begin{itemize}
                    \ides{Phase 1: Growing}
                        \begin{itemize}
                            \item Acquire required locks
                            \item Cannot release any locks
                        \end{itemize}
                    \ides{Phase 2: Shrinking}
                        \begin{itemize}
                            \item Release locks
                            \item Cannot acquire new locks
                        \end{itemize}
                \end{itemize}
            \ipro Guarantees conflict serializability
            \icon \textbf{Cascading Abort:} Abort of one transaction leads to abort of another transaction
                \begin{itemize}
                    \item Happens when we read from a transaction which gets aborted
                    \item Really bad if we have already committed
                        \begin{itemize}
                            \item Commit must be undone
                            \item Conflicts with durability property
                        \end{itemize}
                \end{itemize}
        \end{itemize}
    \ides{Strict Two-Phase Locking (Strict 2PL)}
        \begin{itemize}
            \item Phase 1 is similar to 2PL
            \ides{Phase 2:}
                \begin{itemize}
                    \item All looks are kept until end of transaction
                        \begin{itemize}
                            \item I.e. \verb+COMMIT+ or \verb+ABBORT+ released all locks
                        \end{itemize}
                \end{itemize}
            \icon Deadlocks possible
                \begin{itemize}
                    \ides{Detection}
                        \begin{itemize}
                            \item Each transaction is a node
                            \item $\exists$ edge $T_i$ to $T_j$ if $T_i$ is waiting for a lock currently hold by $T_j$
                        \end{itemize}
                    \item Deadlock if we have a cyclic wait-for graph
                    \item Non-trivial to decide which transaction to kill
                    \item Prevented by locking in some global order
                \end{itemize}
        \end{itemize}
    \ides{Granularity}
        \begin{itemize}
            \item Problems
                \begin{itemize}
                    \item We need to lock every single tuple at its own
                    \item We need to hold locks for whole transaction to prevent phantoms
                \end{itemize}
            \item DB is hierarchical structure and hence needs to support hierarchical locking
            \item New locks
                \begin{itemize}
                    \ides{Intention Share (IS):}
                        \begin{itemize}
                            \item Some lower nodes are in S
                        \end{itemize}
                    \ides{Intention Exclusive (IX):}
                        \begin{itemize}
                            \item Some lower node are in X
                        \end{itemize}
                    \ides{Share and Intention Exclusive (SIX):}
                        \begin{itemize}
                            \item Root is locked in S
                            \item Some lower node are in X
                        \end{itemize}
                \end{itemize}
            \item Old locks
                \begin{itemize}
                    \ides{S:} All lower nodes are in shard
                    \ides{X:} All lower nodes are in exclusive
                \end{itemize}
            \item Full overview\\
                $\rotatebox[origin=c]{90}{Request}
\begin{tabular}{c | c c c c c c}
    & \multicolumn{6}{c}{Current Lock} \\[2ex]
    Mode & NL & IS & IX & S & SIX & X\\\hline
    NL & $\checkmark$ & $\checkmark$ & $\checkmark$ & $\checkmark$ & $\checkmark$ & $\checkmark$\\
    IS & $\checkmark$ & $\checkmark$ & $\checkmark$ & $\checkmark$ & $\checkmark$ & $\times$\\
    IX & $\checkmark$ & $\checkmark$ & $\checkmark$ & $\times$ & $\times$ & $\times$\\
    S & $\checkmark$ & $\checkmark$ & $\times$ & $\checkmark$ & $\times$ & $\times$\\
    SIX & $\checkmark$ & $\checkmark$ & $\times$ & $\times$ & $\times$ & $\times$\\
    X & $\checkmark$ & $\times$ & $\times$ & $\times$ & $\times$ & $\times$\\
\end{tabular}$
            \item Steps to lock a tuple
                \begin{itemize}
                    \item[1)] Acquire $IS$ on database
                    \item[2)] Acquire $IS$ on table
                    \item[3)] Acquire $S$ on tuple
                \end{itemize}
        \end{itemize}
\end{itemize}

\subsubsection{Snapshot Isolation}
\begin{itemize}
    \ides{Idea:} Assume that transaction are serializable and revert if they are not
    \item Working
        \begin{itemize}
            \item Transaction receives timestamp $TS(T)$ when it starts
            \item Reads are carried out as of the DV version of $TS(T)$
            \item Writes are carried out on a separate buffer
            \item When transaction commits, abort $T_1$ if $\exists T_2$ such that:
                \begin{itemize}
                    \item $T_1$ and $T_2$ update the same object and
                    \item $T_2$ committed after $TS(T_1)$ but before $T_1$ commits
                \end{itemize}
            \item Instead of aborting $T_1$, we can also let $T_1$ finish and only then merge $T_2$
        \end{itemize}
    \ides{Timestamps}
        \begin{itemize}
            \item System clock or monotonically increasing for each transaction
        \end{itemize}
    \ipro High concurrency and availability
        \begin{itemize}
            \item Only block when transaction commits
        \end{itemize}
    \ipro No cascading abort
    \ipro No deadlock
    \icon Unnecessary rollbacks
    \icon \textbf{Write Skew:} Interaction of multiple objects
        \begin{itemize}
            \item Checking integrity constrains happens in the snapshots
            \item Two concurrent transaction update different objects
            \item Integrity constrains for each is ok but for the combination not
        \end{itemize}
    \item Looser version
        \begin{itemize}
            \item Idea
                \begin{itemize}
                    \item Object themselves have read (last read) and write (last written) timestamps
                    \item If transaction accesses object from the future (object has higher timestamp than transaction timestamp), transaction is aborted
                    \item Object timestamps are updated on read/writes
                \end{itemize}
            \ipro No unnecessary rollbacks (I guess)
            \icon Long transaction may starve
            \icon Cascading aborts
        \end{itemize}
\end{itemize}
