%! TEX root = ./main.tex

\section{SQL}
\begin{itemize}
    \item Consists of three steps
        \begin{itemize}
            \item Define the schema of the tables
            \item Put information into the tables
            \item Query the tables
        \end{itemize}
    \item SQL is a family of standards
        \begin{itemize}
            \ides{Data Definition Language (DDL):}
            \ides{Data Manipulation Language(DML):}
            \ides{Query Language:}
        \end{itemize}
    \item Released in 1974
    \item Constantly improved and extended
    \item Different DB engines implement slightly different standards
        \begin{itemize}
            \item Important to choose the right engine for a certain project
        \end{itemize}
\end{itemize}

\subsection{DDL}
\begin{itemize}
    \item Defines schema
    \item Relation schema requires
        \begin{itemize}
            \item Name
             \item Set of columns
             \item Type of columns
        \end{itemize}
    \item Data Types
        \begin{itemize}
            \ides{char (n):} String of length $n$
                \begin{itemize}
                    \item Padded with whitespace to match length
                \end{itemize}
            \ides{varchar (n):} String of length $\le n$
            \ides{integer}
            \ides{blob of raw data}
            \ides{date}
            \item etc.
        \end{itemize}

    \ides{Create Relation:}
\begin{verbatim}
CREATE TABLE Professor(
    PersNr integer,
    Name varchar (30),
    Level varchar(2),
    Room integer,
    PRIMARY KEY (PersNr));
\end{verbatim}

    \ides{Delete Relation:}
\begin{verbatim}
Drop TABLE Professor;
\end{verbatim}

    \ides{Add New Column:}
\begin{verbatim}
ALTER TABLE Professors ADD COLUMN (age integer);
\end{verbatim}
        \begin{itemize}
            \item Unclear what to insert into preexisting tuples for that relation
            \item We can set a default value
                \begin{itemize}
                    \item If not provided it keeps the entry empty
                \end{itemize}
        \end{itemize}

    \ides{Drop Column:}
\begin{verbatim}
ALTER TABLE Professor DROP COLUMN age;
\end{verbatim}
\end{itemize}

\subsection{DML}
\begin{itemize}
    \item
        Put and manipulation information

    \ides{Extract, Transform, Load (ETL):} Used to populate DB automatically
        \begin{itemize}
            \ides{E:} Get data from somewhere
            \ides{T:} Bring in right format
            \ides{L:} Insert into DB
        \end{itemize}
    \item Manual population of DB
        \begin{itemize}
            \icon Error-Prone
            \icon Slow
                \begin{itemize}
                    \item Each query has to be parsed one-by-one
                        \begin{itemize}
                            \item Even when all are pretty much equivalent
                        \end{itemize}
                    \item Constraints have to be checked for each query
                \end{itemize}
        \end{itemize}
    \ides{Insert:}
\begin{verbatim}
INSERT INTO Student (PersNr, Name
VALUES (1111, 'Fred')
\end{verbatim}

    \ides{Delete:}
\begin{verbatim}
DELETE Student
WHERE Semester > 13
\end{verbatim}

    \ides{Update:}
\begin{verbatim}
UPDATE Student
SET Semester = Semester + 1
\end{verbatim}

    \ides{From CSV:}
\begin{verbatim}
COPY Professors FROM '/profs.csv' WITH FORAMT csv;
\end{verbatim}
\end{itemize}

\subsection{Query Language}
\begin{itemize}
    \ides{SELECT ... FROM ... WHERE ...}
        \begin{itemize}
            \ides{FROM:} List of relation whose cross product is taken
            \ides{WHERE:} Selection condition
            \ides{SELECT:} Projection
        \end{itemize}
    \item Select uses bag semantic
        \begin{itemize}
            \ides{SELECT DISTINCT:} Set semantic over all fields
        \end{itemize}
    \item Every RA expression can we written in \textit{SQL Subset}
    \item{SQL Subset:}
        \begin{itemize}
            \ides{Base Query:} $\rho_{a_1, \dots , a_n}(\Pi_{A_1, \dots A_n}(\sigma_{P_1 \wedge \dots \wedge P_m}(R_1 \times \dots \times R_k))) \approx$
\begin{verbatim}
SELECT A1 as a1 ... An as an
FROM R1 ... Rk
WHERE P1 AND P2 ... AND Pm
\end{verbatim}
            \ides{Union:} $R_1 \cup R_2$
\begin{verbatim}
(SQL1) UNION (SQL2)
\end{verbatim}
                \begin{itemize}
                    \item Uses set-like semantic
                    \ides{UNION ALL:} used bag semantic
                \end{itemize}
            \ides{Intersection:} $R_1 \cap R_2$
\begin{verbatim}
(SQL1) INTERSECT (SQL2)
\end{verbatim}
            \ides{Difference:} $R_1 - R_2$
\begin{verbatim}
(SQL1) EXCEPT (SQL2)
\end{verbatim}
            \ides{Selection:} $\sigma_c(R)$
\begin{verbatim}
SELECT * FROM (SQL1) WHERE c
\end{verbatim}
            \ides{Projection:} $\Pi_{A_1, \dots , A_n} R$
\begin{verbatim}
SELECT A1, ..., An FrOM (SQL1)
\end{verbatim}
            \ides{Cross Product:} $R_1 \times R_2$
\begin{verbatim}
SELECT * FROM (SQL1), (SQL2)
\end{verbatim}
            \ides{Rename:} $\rho_{a, b, c}R$
\begin{verbatim}
SELECT A as a, B as b, C as c FROM (SQL1)
\end{verbatim}
        \end{itemize}
    \ides{Sorting}
\begin{verbatim}
SELECT A, B FROM (SQL1) ORDER BY A DESC, B ASCD
\end{verbatim}
        \begin{itemize}
            \item ASCD is default
        \end{itemize}
    \ides{Grouping, Aggregation}
\begin{verbatim}
SELECT A, COUNT(*) FROM (SQL1) GROUP BY A
\end{verbatim}
        \begin{itemize}
            \item Can only project to aggregated fields and aggregation function
                \ides{HAVING:} Filter similar to \verb+WHERE+ but for the aggregated field
        \end{itemize}
    \ides{EXISTS, ANY, ALL, SOME:}
        \begin{itemize}
            \item Useful things
        \end{itemize}
    \ides{Snapshot Semantics:}
        \begin{itemize}
            \item Problematic when deleting (updating) from a relation which we need in a subquery of that relation
            \item Would lead to non-determinism
            \item First all tuples which would get modified are marked
            \item Then the updates are implemented
        \end{itemize}
\end{itemize}

\begin{itemize}
    \item \verb+substring(<string>, <base1startindex>, <length>+
\end{itemize}
