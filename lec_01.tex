\lecture{1}{Thu 15 Sep 2020 10:15}{Introduction}
\subsubsection{System Definition}
\textit{System} encompasses operating systems, database systems networking protocols and couting, compiler design and implementation, distributes systems, cloud computing and online services, bid data and machine learning frameworks. System programming is on and above the HW/SW boundary.

\subsubsection{C Introduction}
There are many C standards and C-like variants. In this course we use the C99 standard. Many languages have borrowed the syntax of C. C is very fast and because it is close to the metal one knows that the code is doing on to the hardware.\\
In contrast to e.g. Java, C does not have objects, classes, methods, interfaces, built-in types, exceptions etc. We get what the hardware gives us. Furthermore, the programmer is responsible for the memory management, garbage collection etc. In C one is also able to access the memory directly via pointers.
\begin{itemize}
    \item \textbf{Comments}: \code{/*...*/} or \code{//}
    \item \textbf{Identifiers}: Same as in Java
    \item \textbf{Block} structure using \code{\{...\}}
\end{itemize}
