%! TEX root = ./main.tex

\section{Integrity Constraints}
\begin{itemize}
    \item Additional constraint to the key and domain constraint
    \item Makes sure changes are consistent
    \item Control the content of the date and its consistency
    \item Are enforced by the schema
    \item Can be defined when:
        \begin{itemize}
            \item Creating the table (\verb+CREATE table+)
            \item Later (\verb+ALTER table+)
        \end{itemize}
    \item Checked at \verb+INSERT+ as well as \verb+UPDATE+
        \begin{itemize}
            \item For foreign key also on \verb+DELETE+
        \end{itemize}
    \item Check happen at tuple level and not at the semantic of the command
        \begin{itemize}
            \item I.e. try to run it and see what happens instead of analysing the query
        \end{itemize}
    \item Some check may fail or succeed depending on the order of the tuples
        \begin{itemize}
            \item We have no influence on this
        \end{itemize}
\end{itemize}

\subsection{Types}
\begin{itemize}
    \ides{NOT NULL}
        \begin{itemize}
            \item Prevents attribute from being \verb+NULL+
            \ides{Syntax:} \verb+some_field any_type not null+
        \end{itemize}
    \ides{PRIMARY KEY}
        \begin{itemize}
            \item Mark attribute as primary key
            \item Must not be \verb+NULL+ and not empty
            \ides{Syntax:} \verb+some_field any_type PRIMARY KEY+
            \item If applied to a tuple, all field must not be \verb+NULL+
            \ides{Syntax:} \verb+PRIMARY KEY (field1, fiels2)+
        \end{itemize}
    \ides{UNIQUE}
        \begin{itemize}
            \item Value must be unique or \verb+NULL+
                \begin{itemize}
                    \item In contrast to \verb+PRIMARY KEY+ which cannot be \verb+NULL+
                \end{itemize}
            \item Multiple entries may be \verb+NULL+ in the same column
            \item Multiple fields can be marked as unique
            \ides{Syntax:} \verb+some_field any_type UNIQUE+
            \item Tuples of fields can be marked as unique
            \ides{Syntax:} \verb+UNIQUE (field1, fiels2)+
        \end{itemize}
    \ides{CHECK}
        \begin{itemize}
            \item Boolean check based on values of a single tuple
            \item Reject if \verb+False+
            \item Accept if \verb+True+ or \verb+Unknown+
            \item Some engines treat check somewhat weirdly
            \item Some engines allow subqueries to be part of a check
            \ides{Syntax:} \verb+CHECK(some_expression_evaluation_to_bool)+
        \end{itemize}
    \ides{FOREIGN KEY}
        \begin{itemize}
            \item Involve two relations
            \item Field must be \verb+NULL+ or a valid reference to another table
            \item The reference field is often a \verb+PRIMARY KEY+ or at least \verb+UNIQUE+
            \ides{Referencing Table:} Table which references a tuple form another table
            \ides{Referenced Table:} Table being referenced by another table
            \ides{Syntax:} \verb+FOREIGN KEY some_field any_type REFERENCES some_table(some_field)+
            \ides{Maintenance}
                \begin{itemize}
                    \item Changes to the referenced table influences the referencing table
                        \begin{itemize}
                            \item And not the other way around!
                            \item On \verb+UPDATE+ or \verb+DELETE+
                        \end{itemize}
                    \item Different ways of handling changes
                    \ides{Cascade}
                        \begin{itemize}
                            \item Propagate modification or delete
                        \end{itemize}
                    \ides{Restrict}
                        \begin{itemize}
                            \item Prevent modification or deletion if if violates constraint
                                \begin{itemize}
                                    \item By throwing an error
                                \end{itemize}
                            \item Check right after each command
                        \end{itemize}
                    \ides{No Action}
                        \begin{itemize}
                            \item Prevent modification or deletion if if violates constraint
                                \begin{itemize}
                                    \item By throwing an error
                                \end{itemize}
                            \item Check after a transaction
                            \item Is the default of PostgreSQL
                            \item Is equivalent to restrict in mySQL
                        \end{itemize}
                    \ides{Set Default/Set Null:}
                        \begin{itemize}
                            \item Set reference to default value or \verb+NULL+
                        \end{itemize}
                    \ides{Syntax:} \verb+ON UPDATE method+, \verb+ON DELETE method+
                \end{itemize}
        \end{itemize}
    \item Using \verb+CONSTRAINT some_name+ we can give a name to constraints
        \begin{itemize}
            \item Useful in practice since it allows easy modification later on
        \end{itemize}
\end{itemize}

