%! TEX root = ./main.tex

\section{Relational Model}
\begin{itemize}
    \item Knowledge is represented as a \textit{collection of facts}
    \item Inference is done using \textit{mathematical logic}
\end{itemize}

\subsection{Schema}
\begin{itemize}
    \ides{Database Schema:}
        \begin{itemize}
            \item Set of relation schema
        \end{itemize}
    \ides{Relation Schema:}
        \begin{itemize}
            \item ``Represented'' as a table
            \item Has a name
            \item Contains a set of fields/attributes
            \item Sometimes referred to as \textit{Relation}
            \item Described as $\text{R}(\text{f}_1: \text{D}_1, \dots, \text{f}_n:\text{D}_n)$
                \begin{itemize}
                    \ides{$\text{R}$:} relation name
                    \ides{$\text{f}_i$:} name of field $i$
                    \ides{$\text{D}_i$:} domain of field $i$
                \end{itemize}
        \end{itemize}
    \ides{Field/Attribute:}
        \begin{itemize}
            \item ``Represented'' as a single columns of the table
            \item Has a name
            \item Described by a domain (/type)
        \end{itemize}
    \item Describes only the header (does not contain any content)
    \item Is not unique
        \begin{itemize}
            \item Different schema have different advantages/disadvantages
        \end{itemize}
\end{itemize}

\subsubsection{Instance}
\begin{itemize}
    \item ``Represented'' a set of rows in the table
    \item Set of tuples $\text{I}_R \subseteq \text{D}_1 \times \dots \times \text{D}_n$ for $\text{R}(\text{f}_1: \text{D}_1, \dots, \text{f}_n:\text{D}_n)$
    \ides{Domain Constraint:} For each filed the domain and the schema domain must match
    \item In practice a DB is a bag and not a set (allows duplicate entries)
        \begin{itemize}
            \item But in theory we assume it is a set
        \end{itemize}
    \item Attributes have no ordering principle, but ordering of attributes and tuples has to match
\end{itemize}

\subsection{Key}
\begin{itemize}
    \ides{Candidate Key:} minimal set of fields that uniquely identify a tuple
    \ides{Primary Key:} one candidate key
        \begin{itemize}
            \item Indicated by underlining the field: $\text{R}(\underline{\text{f}_1: \text{D}_1}, \dots, \text{f}_n:\text{D}_n)$
            \item Every relation must have one (but in a DB this is not required)
        \end{itemize}
    \ides{Key Constraint:}
        \begin{itemize}
            \item The primary key must be unique in each an instance
            \item All valid instances $\text{I} \subseteq \text{D}_k \times \text{D}_a \times \text{D}_b \wedge \forall (k, a, b), (k', a', b') \in I, k = k' \implies (a, b) = (a', b')$
        \end{itemize}
\end{itemize}
