%! TEX root = ./main.tex

\section{Lazy Evaluation}
\begin{itemize}
    \item Only evaluate arguments when needed
    \item Substitute arguments without argument evaluation
    \item Some expressions are never evaluated
        \begin{itemize}
            \item Can save arbitrary amount of work
        \end{itemize}
    \ides{Duplicate Evaluation:}
        \begin{itemize}
            \item One argument may be used multiple times
            \item Haskell avoids duplicate evaluation of the same arguments
            \ides{Sharing:} Pointer graph of arguments indicated if an argument was already executed
                \begin{itemize}
                    \item If it was, we can directly take the result
                \end{itemize}
        \end{itemize}
    \item Arguments are evaluated only when needed and at most once
    \ides{Pattern Matching}
        \begin{itemize}
            \item Arguments evaluate as far as needed to determine pattern match
            \item Start matching the top most pattern and on failure go to the next
        \end{itemize}
    \ides{Guards}
        \begin{itemize}
            \item Evaluate only what is required to check if guard is true
            \item Start matching the top most guard and on failure go to the next
        \end{itemize}
    \ides{Local Definitions}
        \begin{itemize}
            \item \verb+where+ and \verb+let+ are lazily evaluated
        \end{itemize}
    \ides{Functions}
        \begin{itemize}
            \item Outermost operator is first evaluated
                \begin{itemize}
                    \item Top-down evaluation in a syntax tree
                \end{itemize}
            \item If on same level, evaluate from left to right or according to operator precedence
        \end{itemize}
\end{itemize}

\subsection{Application}
\begin{itemize}
    \ides{Data-Driven Programming}
        \begin{itemize}
            \item Data can be generate on demand
                \begin{itemize}
                    \item Improved runtime complexity
                \end{itemize}
            \item Due to lazy evaluation, only required data is constructed
        \end{itemize}
    \ides{Infinite Data}
        \begin{itemize}
            \item Finite representation of infinite data
                \begin{itemize}
                    \item E.g. \verb|from n = n : ones (n+1)| generates an infinite list
                \end{itemize}
            \item We can calculate with infinite data in finite time
                \begin{itemize}
                    \item E.g. \verb+head from 1+
                \end{itemize}
            \item I.e. we describe an infinite stream and compute with arbitrarily large finite prefixes of it
        \end{itemize}
\end{itemize}

\subsection{Correctness}
\begin{itemize}
    \item Complicated analysis of correctness and complexity
    \item Type like \verb+[Int]+ include finite and infinite lists
    \item Proof by induction is sound only for finite lists
        \begin{itemize}
            \item We always assume finite lists for this course
        \end{itemize}
\end{itemize}
