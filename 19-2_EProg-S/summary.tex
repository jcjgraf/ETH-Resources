% Created 2019-11-12 Di 07:43
% Intended LaTeX compiler: pdflatex
\documentclass[11pt]{article}
\usepackage[utf8]{inputenc}
\usepackage[T1]{fontenc}
\usepackage{graphicx}
\usepackage{grffile}
\usepackage{longtable}
\usepackage{wrapfig}
\usepackage{rotating}
\usepackage[normalem]{ulem}
\usepackage{amsmath}
\usepackage{textcomp}
\usepackage{amssymb}
\usepackage{capt-of}
\usepackage{hyperref}
\usepackage{amsmath}
\usepackage[paper=a4paper,left=30mm,right=20mm,top=25mm, bottom=25mm]{geometry}
\usepackage{tikz}
\usepackage{pgfplots}
\usepackage{aligned-overset}
\usepackage[ruled,vlined]{algorithm2e}
\usepackage{xcolor}
\newcommand\mycommfont[1]{\footnotesize\ttfamily\textcolor{blue}{#1}}
\SetCommentSty{mycommfont}
\newcommand\com[1]{\textcolor{red}{#1}}
\newcommand*{\QED}{\hfill\ensuremath{\square}}%
\author{Jean-Claude Graf}
\date{\textit{[2019-09-17 Di]}}
\title{Einführung in die Programmierung}
\hypersetup{
 pdfauthor={Jean-Claude Graf},
 pdftitle={Einführung in die Programmierung},
 pdfkeywords={},
 pdfsubject={},
 pdfcreator={Emacs 26.3 (Org mode 9.1.9)}, 
 pdflang={English}}
\begin{document}

\maketitle
\tableofcontents

\setcounter{secnumdepth}{0}

\section{Header}
\label{sec:org1483c50}
\begin{itemize}
\item Einführung in die Programmierung\\
\item \href{https://www.bi.id.ethz.ch/personensuche/personenDetail.view?page=1\&lang=de\&schnellSuche=Thomas+Gross\&order=NAME\&descending=false\&paging=true\&pid=134a9}{Prof. Dr. Thomas Gross}\\
\item Department Informatik\\
\item \href{https://www.lst.inf.ethz.ch}{Laboratory for Software Technology}\\
\item \href{http://www.vvz.ethz.ch/Vorlesungsverzeichnis/lerneinheitPre.do?lerneinheitId=132733\&semkez=2019W\&lang=de}{Vorlesungsverzeichnis}\\
\item \href{https://www.lst.inf.ethz.ch/education/einfuehrung-in-die-programmierung-i--252-0027-.html}{Vorlesungswebseite}\\
\end{itemize}

\subsection{Ziel der Vorlesung}
\label{sec:orga27870d}
\begin{itemize}
\item Kompetenz\\
\begin{itemize}
\item Korrekte Programme systematisch erstellen\\
\end{itemize}
\item Wichtig sind:\\
\begin{itemize}
\item Aufmerksamkeit\\
\item Imagination, Phantasie\\
\item Übung\\
\end{itemize}
\item Java\\
\begin{enumerate}
\item Programm lesen\\
\item Programm verstehen\\
\item Programm erstellen\\
\end{enumerate}
\end{itemize}

\subsection{Media}
\label{sec:orgc839a6e}
\subsubsection{Slides}
\label{sec:org1c21afc}
\begin{itemize}
\item unbeschrieben\\
\begin{itemize}
\item ev. vor der Vorlesung\\
\end{itemize}
\item beschrieben (nur mit wichtigen Notizen)\\
\begin{itemize}
\item 24h-48h nach Vorlesung\\
\end{itemize}
\end{itemize}
\subsubsection{Video}
\label{sec:org46d9731}
\begin{itemize}
\item übertragenes Video\\
\item Hauptprojektor\\
\end{itemize}

\subsection{Übungen}
\label{sec:org6ca0275}
\begin{itemize}
\item Aufgabenblätter\\
\begin{itemize}
\item auf Website publiziert\\
\item in Übungsgruppe vor-besprochen\\
\end{itemize}
\item Praxisübung\\
\begin{itemize}
\item Lösung im Internet\\
\end{itemize}
\item Bonusübung\\
\begin{itemize}
\item ab 4. od 5. Aufgabenblatt\\
\item Max 0.25 Note Bonus für Basisprüfung\\
\item selber lösen\\
\begin{itemize}
\item Mit anderen reden, aber keine Notizen davon mitnehmen und innert mind 1h danach nichts aufschreiben\\
\end{itemize}
\end{itemize}
\end{itemize}

\noindent\rule{\textwidth}{0.5pt}
\section{0.0 Einleitung}
\label{sec:org4723959}
\subsection{Programm, Programmiereun und co.}
\label{sec:orgf81f35c}
\subsubsection{Programm}
\label{sec:org6757c47}
\begin{itemize}
\item \emph{Programmieren} - \emph{Programm}: griechisch \emph{programma} = schriftliche Banntmachung, Aufruf\\
\item Folge von Anweisungen\\
\item Von Computer ausgeführt\\
\begin{itemize}
\item Wir müssen wissen was er alles versteht. Mögliche Anweisungen -> Programmiersprache\\
\end{itemize}
\item Realisiert einen Algorithmus\\
\end{itemize}

\subsubsection{Programmieren}
\label{sec:orga9a09fd}
\begin{itemize}
\item Erstellen eines Programms/Software Entwicklung\\
\item Beinhaltet alle Aspekte von Entwurf bis Installation\\
\item Bedenken was nicht offensichtlich ist macht es fordernd\\
\item \emph{Programming as universal activity} by Vinton Cerf, CACM March 2016, vol 59(3) p7\\
\begin{itemize}
\item analyzing problems\\
\item breaking them into manageable parts\\
\item finding solutions\\
\item integrating the results\\
\end{itemize}
\item Lösungen finden\\
\begin{itemize}
\item Für Mensch (Lösung beschreiben)\\
\item Für Maschiene (Anweisung)\\
\end{itemize}
\item Wichtiges Informatik Konzept\\
\end{itemize}

\subsubsection{Probleme}
\label{sec:org69cc7d5}
\begin{itemize}
\item Möglichkeiten und Grenzen der maschinellen Informationsverarbeitung\\
\begin{itemize}
\item es exists unmögliche Probleme\\
\item Kosten der Berechnung sind sehr wichtig\\
\end{itemize}
\end{itemize}

\subsubsection{Programmiersprache}
\label{sec:org88d716e}
\begin{itemize}
\item Sprache für Computer (führt aus)\\
\item Folge von mögliche Anweisungen\\
\item Sprache für Menschen (schreibt, liest)\\
\item Geben vor wie Löaungen für ein Problem beschrieben werden können\\
\end{itemize}

\subsection{Java}
\label{sec:orgb2bb9fc}
\begin{itemize}
\item \emph{Industrial strength} Sprache\\
\begin{itemize}
\item sehr verbreitet\\
\end{itemize}
\item Viele Konzepte\\
\begin{itemize}
\item nicht alle in dieser Vorlesung behandelt\\
\end{itemize}
\item JSHell\\
\begin{itemize}
\item nur für simple Snippets\\
\end{itemize}
\item Beispiel:\\
\begin{itemize}
\item Grosse Zahlen können zu unerwarteten Lösungen führen\\
\item Daher, was sind legale Eingabewerte für Zahlen?\\
\end{itemize}
\end{itemize}
\begin{figure}[htbp]
\centering
\includegraphics[width=.9\linewidth]{./img/190917eclipse.jpg}
\caption{Funktionsweise von Eclipse (jedem IDE)}
\end{figure}


\section{1.0 EBNF Notation}
\label{sec:org7b75009}
E - Extended\\
B - Backus\\
N - Naur oder Normal\\
F - Form\\

\begin{itemize}
\item Beschreibt die Syntax einer Sprache\\
\item EBNF ist eine Erweiterung von BNF\\

\item Vier elementaren Ausdrucksmöglichkeiten ("control forms") die in Java vorkommen\\
\item Symbol kann Name, Keyword, Anweisung, Zahl etc. sein\\
\end{itemize}

\subsection{EBNF Beschreibung}
\label{sec:org0558326}
\begin{itemize}
\item haben einen Namen (für kompliziertere Beschreibungen zu erstellen)\\
\begin{itemize}
\item gewählter Name ist irrelevant\\
\end{itemize}
\item erstellen ist ähnlich wie Programmieren in Java\\
\item sind sehr \emph{formal} \\
\begin{itemize}
\item sehr präzise und verständlich\\
\end{itemize}
\item Sind eine Menge EBNF Regeln\\
\begin{itemize}
\item Menge: -> Reihenfolge unwichtig\\
\end{itemize}
\item geben an welche Symbole erlaubt sind\\
\begin{itemize}
\item Alphabet muss jedoch nicht explizit definiert werden. Es besteht einfach aus den Verwendeten Symbolen\\
\end{itemize}
\item Reihenfolge der Regeln ist irrelevant\\
\begin{itemize}
\item Konvention:\\
\begin{itemize}
\item von einfachen nach komplexen Regeln\\
\item Name der letzten Regel ist der Name der relevanten Beschreibung\\
\end{itemize}
\end{itemize}
\end{itemize}

\subsection{EBNF Regel}
\label{sec:orgb32c77e}
\begin{itemize}
\item Haben 3 Bestandteile\\
\item Gibt an welche Symbole erlaubt sind\\
\end{itemize}

\subsubsection{Linke-Seite (LHS)}
\label{sec:orgf105570}
\begin{itemize}
\item Wort, welches der Name der Regel ist (kursiv- (oder in '<>' als Ersatzdarstellung) und  kleingeschrieben)\\
\end{itemize}
\subsubsection{Rechte-Seite (RHS)}
\label{sec:org0acaa0f}
\begin{itemize}
\item Beschreibt den Namen\\
\item Kann enthalten\\
\begin{itemize}
\item Namen anderer Beschreibungen\\
\item Buchstaben (Definiertes Set möglicher Werten)\\
\item Kombinationen der vier Kontrollelementen\\
\end{itemize}
\end{itemize}
\subsubsection{Verbindung}
\label{sec:org67d80ef}
\begin{itemize}
\item "ist definiert als" (\(\Leftarrow\)) Spezielles zeichen\\
\item trennt LHS und RHS\\
LHS \(\Leftarrow\) RHS\\
\end{itemize}

\subsubsection{Kontrollelemente (\emph{Control Forms})}
\label{sec:orgb3186a5}
\begin{enumerate}
\item Aufreihung (\emph{sequence})
\label{sec:orgba00afa}
\begin{itemize}
\item von links nach rechts gelesen\\
\item Reihenfolge ist wichtig\\
\begin{itemize}
\item 'abc' != 'cba'\\
\end{itemize}
\item Zwischenraum zwischen Aufreihung ist nicht relevant\\
\begin{itemize}
\item Leerzeichen muss explizit gezeigt werden wenn eines vorhanden sein sollte\\
\end{itemize}
\item Andere EBNF Regeln können als Baustein verwendet werden\\
\end{itemize}
digit-d \Leftarrow d\\
digit-2 \Leftarrow 2\\
digit-8 \Leftarrow 8\\
raum \Leftarrow digit-d digit-2 digit-8\\

\item Entscheidung (\emph{decision})
\label{sec:org0eb593a}
\begin{enumerate}
\item Auswahl
\label{sec:orgccb8870}
\begin{itemize}
\item Menge von Alternativen\\
\item Reihenfolge ist nicht wichtig\\
\item Durch \(|\) (\emph{stroke}) getrennt\\
\item Um Umklarheiten zu vermeiden können wir Klammern \(()\)  verwenden um die Reihenfolge zu verdeutlichen\\
\begin{itemize}
\item wie in der Mathe\\
\end{itemize}
\end{itemize}

<digit> \Leftarrow 1 | 2 | 3 | 4 | 5 | 6 | 7 | 8 | 9 | 0

\item Option
\label{sec:org805dda4}
\begin{itemize}
\item wie die Alternative jedoch können wir auch "nichts" wählen\\
\item Element(e) in \([]\) (\emph{Square Brackets}) und durch \(|\) getrennt\\
\item \(\Epsilon\) (Epsilon) steht für "nichts"\\
\end{itemize}

<initials> \Leftarrow T [R] G  //TRG oder TG\\
---\\
<Vorzeichen> \Leftarrow [+|-]\\
---\\
<pm> \Leftarrow +|-\\
<vorzeichen> \Leftarrow [pm]\\
---\\
<choice> \Leftarrow [a|\Epsilon] // Wähle a oder "nichts"\\be
\end{enumerate}

\item Wiederholung (\emph{repetition})
\label{sec:org89d57dd}
\begin{itemize}
\item Zu Wiederholende Ausdruck steht in \({}\) (\emph{Curly Brackets})\\
\item kann 0, 1, oder beliebig mal wiederholt werden\\
\begin{itemize}
\item 0 Wiederholungen heisst Element fehlt (Epsilon)\\
\item können nicht sagen das etwas genau n mal wiederholt werden soll\\
\end{itemize}
\end{itemize}

<folge> \Leftarrow {a} // kann eine Anzahl von a, oder auch \Epsilon sein\\
--- \\
<digit> \Leftarrow 1 | 2 | 3 | 4 | 5 | 6 | 7 | 8 | 9 | 0\\
<integer> \Leftarrow [+|-] digit{digit} // Integral Darstellung\\

\item Rekursion (\emph{recursion})
\label{sec:orgcb225cd}
\begin{itemize}
\item Manchaml nötig für komplexe Beschreibungen\\
\item Direkt Rekursiv\\
\begin{itemize}
\item Der Name der Linken Seite wird \textbf{direkt} (LHS in RHS) auf der Rechten Seite der Regel verwendet\\
<r> \Leftarrow A | A <r>

\begin{itemize}
\item Ist AAB legal?\\
\begin{table}[htbp]
\caption{Tabellen überprüfung}
\centering
\begin{tabular}{ll}
 & Regel\\
\hline
\(<r>\) & \\
\(B \vert Ar\) & \#1\\
\(Ar\) & \#2\\
\(A(b \vert Ar)\) & \#1\\
\(AAr\) & \#2\\
\(AA(B\vert Ar)\) & \#1\\
\(AAB\) & \#2\\
\end{tabular}
\end{table}

\begin{figure}[htbp]
\centering
\includegraphics[width=.9\linewidth]{img/190927_1.jpg}
\caption{Ableitungbaum überprüfung}
\end{figure}
\end{itemize}
\end{itemize}
\item Indirekt Rekusiv\\
\begin{itemize}
\item Der Name einer Regel kommt in der RHS einer anderen Regel der selben Beschreibung vor\\
name1 \Leftarrow A name2\\
name2 \Leftarrow B name1 | C\\
\end{itemize}

\item Nicht alle Rekursionen können durch Wiederholungen ausgedrückt werden\\
\begin{itemize}
\item keine  Beschreibung ohne Rekursion für \(A^n B^n \forall n \geq 0\)\\
balance \Leftarrow | A balance B

\begin{center}
\includegraphics[width=.9\linewidth]{img/190927_10.jpg}
\end{center}
\end{itemize}

\item BNF erhielt anfänglich nur Rekursion und Auswahl\\
\item Nicklaus Wirth fügte das E hinzu\\
\end{itemize}
\end{enumerate}

\subsection{Überprüfung für Legalität}
\label{sec:orgea8b907}
Eine gebegenes Symbol (Folge von Buchstaben) kann für eine EBNF beschreibung legal sein, wenn \textbf{alle} Buchstaben des Symbols mit den Elementen der EBNF Regel übereinstimmen, oder ansonnsten illegal, sein. Ein legales symbol \emph{entspricht} einer EBNF Beschreibung.\\

\subsubsection{Informeller Bestimmung:}
\label{sec:org3642b21}
Wir gehen das Symbol von links nach rechts durch und überprüfen ob es allen Regeln der Beschreibung entspricht\\

\subsubsection{Tabellen}
\label{sec:orgbb7b0a3}
Sind \textbf{formaler} und \textbf{kompakter}\\

\begin{itemize}
\item Aufbau:\\
\begin{itemize}
\item 1. Zeile: Namen der zu testenden EBNF Beschreibung\\
\item folgende Zeilen verden aus der vorgehenden abgeleitet\\
\item letzte Zeile: Symbol (gegebenes Symbol wenn es der Beschreibung entspricht)\\
\end{itemize}

\item Durchführung:\\
\begin{enumerate}
\item Ersetzen eines Namen (LHS) durch die entsprechende Definition (RHS)\\
\item Wahle einer Alternative\\
\item Entscheidung ob ein optionales Element dabei ist oder nicht\\
\item Bestimmung der Zahl der Wiederholungen\\

\item 1. \& 2. Regeln dürfen als einzige zusammengefasst werden\\
\end{enumerate}

\item Bsp. Überprüfung von \(+142\) für \(<digit>\)\\
\begin{center}
\begin{tabular}{ll}
Status & Reason (rule \#)\\
\hline
\(<interger>\) & Given\\
\([+ \vert -]<digit>{<digit>}\) & Replace \(<integer>\) by RHS (\#1)\\
\([+]<digit>{<digit>}\) & Choose \(+\) alternative (\#2\#\\
\(+<digit>{<digit>}\) & Include option (3)\\
\(+1{<digit>}\) & Replace the first \(<digit>\) by \(1\) alternative (\#1 \& \#2)\\
\(+1<digit><digit>\) & Use two repetitions (\#4)\\
\(+14<digit>\) & Replace the first <digit> by \(4\) alternative (\#1 \& \#2)\\
\(+142\) & Replace the first <digit> by \(2\) alternative (\#1 \& \#2)\\
\end{tabular}
\end{center}
\end{itemize}

\subsubsection{Ableitungsbäume}
\label{sec:orgef61d57}
Grafische Darstellung eines Beweises\\

\begin{itemize}
\item Aufbau:\\
\begin{itemize}
\item Oben: Name der zu testenden EBNF Beschreibung\\
\item Unten: Symbol (gegebenes Symbol wenn es der Beschreibung entspricht)\\
\end{itemize}

\item Durchführung:\\
\begin{itemize}
\item Kanten Zeigen welche Regel verwendet wird um von einer Zeile zur nächsten zu kommen\\
\begin{figure}[htbp]
\centering
\includegraphics[width=.9\linewidth]{img/190927_3.jpg}
\caption{Ableitungbaum}
\end{figure}
\end{itemize}
\end{itemize}

\subsection{Sonderzeichen}
\label{sec:org1ed7352}
\begin{itemize}
\item Viele Zeichen habe eine spezielle Bedeutung in EBNF\\
\begin{itemize}
\item \{, \}, [, ], |, (, ), \(\Rightarrow\),\\
\item weiter ev. auch <, > und  "\\
\end{itemize}
\item Um ein solches Zeichen als Symbol zu verwenden wir es in einem Quadrat (oder als Ersatzdarstellung in Anzührungszueichen) geschrieben\\
\begin{itemize}
\item \(\boxed{\{}\)\\
\end{itemize}
\end{itemize}

\subsection{Äquivalenz von Beschreibungen}
\label{sec:org9151ca3}
\begin{itemize}
\item EBNF Beschreibungen sind äquivalent wenn sie die selben Symbolen legal resp. illegal sind\\
\item die Sprache der Beschreibungen sind identisch\\
\item Zwei Beschreibungen \(B_1\) und \(B_2\) definieren die selbe Sprache:\\
\begin{itemize}
\item Symbol legal für \(B_1\): dann auch legal für \(B_2\)\\
\item Symbol illegal für \(B_1\): dann auch illegal für \(B_2\) \\
\item Symbol legal für \(B_2\): dann auch legal für \(B_1\)\\
\item Symbol illegal für \(B_2\): dann auch illegal für \(B_1\)\\
\end{itemize}
\end{itemize}

-> äquivalenz zwischen folgenden Beschreibungen liegen vor:\\
<singn> \Leftarrow +|-\\
<digit> \Leftarrow 0|1|2|3|4|5|6|7|8|9\\
<integer> \Leftarrow [<sign>]{<digit>}\\
---\\
<integer> \Leftarrow [+|-]{0|1|2|3|4|5|6|7|8|9}\\

\subsection{Syntax und Semantik}
\label{sec:orgb6965ad}
\begin{itemize}
\item \emph{Syntax}: Form\\
\item \emph{Semantik}: Bedeutung\\

\item Syntak ist viel einfach festzulegen als die Semantik\\
\item Eine Beschreibung kann zwar syntaxisch richtig sein, jedoch keinen Sinn machen\\
\begin{itemize}
\item Bsp. Alle lesenden Schiffe riechen gelb\\
\end{itemize}

\item zwei wichtige Semantik Fragen:\\
\begin{enumerate}
\item Können unterschiedliche Symbole die selbe Bedeutung haben?\\
\item Kann ein Symbol verschiedene Bedeutungen haben?\\
\end{enumerate}
\item Der Kontext spielt eine wichtige Rolle\\
\begin{itemize}
\item z.B. haben 0012 und 12 die selbe Bedeutung? Kommt auf den Kontext drauf an\\
\begin{itemize}
\item Mathematik: ja\\
\item Pin: nein\\
\end{itemize}
\end{itemize}
\end{itemize}

\subsection{Menge von Zahlen}
\label{sec:org098f235}
\begin{itemize}
\item Klammern mit eine Box oder Anführungzeichen escapen\\
\end{itemize}
<intergerlist> \Leftarrow <integer>{, <integer>}\\
<integerset> \Leftarrow \boxed{\{}[<integerlist>]\boxed{\}}

\begin{itemize}
\item Entsprechen einer Beschreibung (sind legal) wenn sie legal für alle Elemente sind\\
\begin{itemize}
\item müssen den Beweis nicht bis zum Ende führen, sondern können beim Erreichen eines Lemmas stoppen\\
\end{itemize}
\end{itemize}

\subsubsection{Äquivalenz von Mengen}
\label{sec:org240436e}
\begin{itemize}
\item Mehrfachnennungen und Reihenfolge sind irrelevant\\
\item Kanonische Darstellung (geordnet von klein (links) zu gross (rechts))\\
\begin{itemize}
\item Kann nicht durch EBNF Regeln erzwungen werden\\
\end{itemize}
\item Langsam kommen EBNF Regeln an ihre Grenzen\\
\end{itemize}

<nonzero> \Leftarrow 1 | 2 | 3 | 4 | 5 | 6 | 7 | 8 | 9\\
<sign> \Leftarrow [+|-]\\
<integer> \Leftarrow <sign> <nonzero> {<nonzero> | 0}\\

\subsection{Graphische Darstellung von EBNF Regeln}
\label{sec:org799f42f}
\begin{itemize}
\item Syntax Graph: graphische Darstellung\\
\item Machen es leichter zu erkennen, welche Zeichen in einem Symbol (in welcher Reihenfolge) auftreten können\\
\item Aufreigung: \(A B C D\)\\
\begin{itemize}
\item Durch jedes element der Reihe\\
\begin{center}
\includegraphics[width=.9\linewidth]{img/190927_4.jpg}
\end{center}
\end{itemize}
\item Option: \([A]\)\\
\begin{itemize}
\item Ein element der Leiter\\
\begin{center}
\includegraphics[width=.9\linewidth]{img/190927_5.jpg}
\end{center}
\end{itemize}
\item Wiederholung: \({A}\)\\
\begin{itemize}
\item entweder Kante ohne oder Kante mit Element\\
\begin{center}
\includegraphics[width=.9\linewidth]{img/190927_6.jpg}
\end{center}
\end{itemize}
\item Auswahl: \(A | B | C | D\)\\
\begin{itemize}
\item Pfeil von rechts zurück nach links\\
\begin{center}
\includegraphics[width=.9\linewidth]{img/190927_7.jpg}
\end{center}
\end{itemize}
\item Klammersetzung: \(A B | C\)\\
\begin{itemize}
\item Klammern sind sehr wichtig\\
\begin{center}
\includegraphics[width=.9\linewidth]{img/190927_8.jpg}
\end{center}
\end{itemize}
\item Können einen Graphen in einen anderen einsetzen/substitutionieren\\
\begin{itemize}
\item interne Namen von Beschreibungen verschwinden\\
\begin{center}
\includegraphics[width=.9\linewidth]{img/190927_9.jpg}
\end{center}
\end{itemize}
\item Können von einem Graphen auf legale Symbole zurückführen oder sogar die Beschreibung mit den Regeln rekonstruieren\\
\end{itemize}


\section{2.0 Einfach Java Programme}
\label{sec:org7475015}
\subsection{EBNF}
\label{sec:org36aece4}
\begin{itemize}
\item Hält die Synatx der Regeln von Java fest\\
\item Namen / identifier:\\
\begin{itemize}
\item mind. 1 Zeichen lang\\
\item mit Buchstaben anfangen [a-zA-Z]\\
\item kann Buchstaben und Ziffern enthalten\\
\end{itemize}
\end{itemize}
\begin{figure}[htbp]
\centering
\includegraphics[width=.9\linewidth]{img/190924_1.jpg}
\caption{Java Regeln}
\end{figure}

\subsection{Einleitung}
\label{sec:org5bce39f}
\begin{itemize}
\item Programm besteht aus mind einer Klasse, mind einer Methode \(main\) und einer Reihe von Answeisungen\\
\item Wir müssen uns strikt an die Anweisungen halten, sonst gibt es Probleme beim kompolieren \(\rightarrow\) Error Message\\
\begin{itemize}
\item nicht alle Fehlermeldungen sind intuitiv\\
\end{itemize}
\item jshell erlaubt das ausführen von Code wie in einer Hauptclass und Main Methode\\

\item Bestimmte Namen sind durch Java reserviert und dürfen nicht vorkommen\\
\end{itemize}

\subsubsection{Classe}
\label{sec:org8560afb}
\begin{itemize}
\item nur eine \(class\) pro Datei (fürs Erste)\\
\item Name/Bezeichner der \(class\) gleich Name der Datei\\
\item Konvention: UpperCamelCase\\
\begin{center}
\includegraphics[width=.9\linewidth]{img/190924_2.jpg}
\end{center}
\end{itemize}
\subsubsection{Methode}
\label{sec:orgbb155a6}
\begin{itemize}
\item \(public static void main\) muss zwingen in einer Klasse vorkommen\\
\item \(main\) muss denn auszufürenden Code beinhalten\\
\item Konvention lowerCamelCase\\
\begin{center}
\includegraphics[width=.9\linewidth]{img/190924_3.jpg}
\end{center}
\item \(println\) Java Medhode/Funktion\\
\begin{itemize}
\item gibt Strings aus mit neuer Zeile\\
\end{itemize}
\item \(print\)\\
\begin{itemize}
\item gibt String aus ohne neue Zeile\\
\end{itemize}
\end{itemize}
\subsubsection{\(String\)}
\label{sec:orgc8e68bf}
\begin{itemize}
\item Text zwischen Anführungszeichen "bla"\\
\item Folge von Buchstaben\\
\item Darf nur eine Zeile lang sein\\
\item Darf keine Anführungzeichen beinhalten\\
\item Sonderzeichen\\
\begin{itemize}
\item Ersatzdarstellungen (escape sequences)\\
Startet mit Backslash\\
\begin{itemize}
\item \(\n\) new line character\\
\item \(\t\) Tab character\\
\item \(\"\) Quotation mark characeter\\
\item \(\\\) Backslash character\\
\end{itemize}
\end{itemize}
\end{itemize}

\subsubsection{Kommentare}
\label{sec:org586061e}
\begin{itemize}
\item Notizen im Programmcode\\
\begin{verbatim}
// Text bis zum Ende der Zeile
/* Text bis zum 
naechsten
*/
\end{verbatim}
\begin{itemize}
\item Verwendung:\\
\begin{itemize}
\item Anfang des Programms\\
\begin{itemize}
\item Zweck\\
\item Algorithmus\\
\item Author\\
\end{itemize}
\item Anfang jeder Methode\\
\item Wenn Code nicht direkt verständliche\\
\begin{verbatim}
/* 
 * Author: Ein Student; Herbst 2016, Uebung 1
 * Entwurf uebernommen von einer Frau XXXX (Assistentin) 
 */
\end{verbatim}
\end{itemize}
\end{itemize}
\begin{center}
\includegraphics[width=.9\linewidth]{img/190924_4.jpg}
\end{center}
\end{itemize}

\subsection{2.1 Methoden}
\label{sec:orgb24fbaa}
\begin{itemize}
\item Folge von strukturierten Anweisungen mit Namen\\
\item Wiederholungen können vermieden werden\\
\item \(static\)\\
\begin{itemize}
\item gibt der Methode weitere Eigenschaften\\
\item \(main\) ist \(static\)\\
\item \(main\) wird automatisch aufgerufen\\
\end{itemize}
\item Methoden finden, die Teilprobleme lösen\\
\begin{itemize}
\item Teilprobleme einzeln lösen\\
\end{itemize}
\item Müssen \emph{deklariert} werden, befor sie aufgerufen wrden können\\
\item Mehrfachaufruf möglich\\
\item Folge der Ausführungen wird \emph{control flow} genannt\\
\begin{itemize}
\item Zuerst wird die Methode komplet ausgeführt, erst dann wird zurück zur Aufrufermethode gegangen\\
\(\Rightarrow\) geradliniger Kontrollflow\\
\item Anweisungsreihenfolge ist explizit\\
\begin{itemize}
\item Control flow kann aktiv geändert werden\\
\end{itemize}
\begin{center}
\includegraphics[width=.9\linewidth]{img/190927_11.jpg}
\end{center}
\end{itemize}
\item Anordnung der Methoden in der Class ist irrelevant\\
\end{itemize}

\subsection{2.2 Typen und Variable}
\label{sec:org5770744}
\subsubsection{Typen}
\label{sec:orgc4dfdd5}
\begin{itemize}
\item Beschreiben Eigenschaften von Daten\\
\item Haben Auswirkung auf:\\
\begin{itemize}
\item Darstellung der Werte\\
\item mögliche Operationen\\
\end{itemize}
\item Sagen wie Daten im Computer gespeichert sind\\
\begin{itemize}
\item Intern wird alles in Binär gespeichert\\
\item Definiert durch eine Tabelle (Ähnlich wie ASCII)\\
\end{itemize}
\item ist ein Binärstring \(01100001\) nun eine Zahl \(97\) oder ein Zeichen \(a\)?\\
\begin{itemize}
\item Datentyp gibt nötige Zusatzinformation\\
\end{itemize}
\item Tatentypen sollten immer explizit angegeben werden\\
\begin{itemize}
\item Compiler kann auch selbstständig den Typ herausfinden, ist jedoch nicht so empfolen\\
\end{itemize}
\item Vrhinden Fehler\\
\item Erlauben Optimierungen\\
\item Von wo kommen Typen\\
\begin{itemize}
\item in der Sprache Definierte Type (\emph{primitive types})\\
\begin{table}[htbp]
\caption{Primative Types}
\centering
\begin{tabular}{ll}
Name & Beschreibung\\
\hline
int & ganze Zahlen\\
long & ganze lange Zahlen\\
double & reelle Zahlen\\
char & einzelne Buchstaben\\
characterboolean & logische Werte\\
\end{tabular}
\end{table}
\item Typen aus Bibliotheken\\
\item Benutzer-definierte Typen\\
\end{itemize}

\item Werte haben einen (festen) Typen\\
\begin{itemize}
\item \(TypeA \bigdot TypeA \rightarrow TypeA\)\\
\begin{itemize}
\item z.b. \(14 / 3 = 4\)\\
\end{itemize}
\end{itemize}
\end{itemize}

\begin{enumerate}
\item Ausdrücke (Expressions)
\label{sec:orgd675276}
\begin{itemize}
\item Ein Wert (\emph{literal value}), Operanden, Operatoren oder eine kombination von mehreren sind Ausdrücke für einen Typen\\
\item Werden während der Ausführung ausgewertet\\
\begin{center}
\includegraphics[width=.9\linewidth]{img/190927_12.jpg}
\end{center}
\end{itemize}

\item Operatoren
\label{sec:org9f35136}
\begin{itemize}
\item Verknüpfen Werte oder Ausdrücke\\
\begin{itemize}
\item + Addition\\
\item - Subtraktion\\
\item * Multiplikation\\
\item / Division\\
\item \% Modulo\\
\begin{itemize}
\item gibt den Rest der int division zurück\\
\end{itemize}
\end{itemize}
\end{itemize}

\item Assoziativität
\label{sec:orgfdc9a1d}
\begin{itemize}
\item Bestimmt welche Elemente ein Operand verbindet\\
\item \(a \bigodot b \bigodot c\)\\
\begin{itemize}
\item Rechtsassoziativ: \(a \bigodot (b \bigodot c)\)\\
\item Linksassoziativ: \((a \bigodot b) \bigodot c\)\\
\end{itemize}

\item die meisten bekannten Operatoren sind Linksassoziativ\\
\end{itemize}

\item Rangordnung (\emph{Precedence})
\label{sec:org620938c}
\begin{itemize}
\item Bestimmt welche Operatoren in welcher Reihenfolge ausgeführt werden\\
\item Rangordnung von \(\bigodot\) und \(\bigotimes\) entscheidet wenn ein Ausdruck mehrere Operanden \(X, Y, Z\) hat\\
\end{itemize}

\item Operanden und Operatoren
\label{sec:org45949fc}
\begin{itemize}
\item Ausführ Reihenfolge bei vorkommen von mehreren Operatoren\\
\begin{itemize}
\item Rangordnung\\
\item Assoziativität\\
\item von links nach rechts\\
\end{itemize}

\item mit Klammern kann eine Abweichung von diesen Regeln erzwungen werden.\\
\begin{center}
\includegraphics[width=.9\linewidth]{img/190926_1.jpg}
\end{center}
\end{itemize}

\item \(Long\)
\label{sec:org4dfda6d}
\begin{itemize}
\item löst das Problem von \(int\) (Grosse Zahle verhalten sich "komisch")\\
\begin{itemize}
\item kann aber auch überfordert werden\\
\end{itemize}
\item Durch anfügen von \(L\) ans Ende der Zahl wird die Zahl als \(Long\) definiert\\
\end{itemize}

\item \(Double\)
\label{sec:orgb3f8088}
\begin{itemize}
\item Reelle Zahlen\\
\item durch hinzuführen von \$.\$ oder \$.0\$ zu einer ganzen Zahl,  macht sie zu \(Double\)\\
\end{itemize}

\item Kombinieren von Ausdrücken und Typen
\label{sec:org715a9d1}
\begin{itemize}
\item \(Int\) oder \(Long\) und \(Double\) werden kombiniert zu \(Double\)\\
\begin{itemize}
\item \(double \bigodot intOrLong) \rightarow  double\)\\
\end{itemize}
\item Wird in einem Audruck für jede Operation einzeln gemacht\\
\begin{itemize}
\item dabei wird die Rangordnung und Assoziativität beachtet\\
\end{itemize}
\end{itemize}

\item Umwandlung von Typen
\label{sec:org33e5bc8}
\begin{itemize}
\item implizit: Durch kompination von Typen\\
\item explizip: durch type cast\\
\begin{itemize}
\item cast bezieht sich nur auf den Ausdruck direkt dahinter (Rechtsassoziativ)\\
\begin{itemize}
\item \((type) expression\)\\
\end{itemize}
\item höhere Rangordnung als Arithmetische Operatoren\\
\end{itemize}
\end{itemize}

\item String Operatoren
\label{sec:orge6e6c52}
\begin{itemize}
\item \(+\) Operator verkettet (\emph{concatenation}) \(Strings\)\\
\item Zahlen werden automatich zu \(String\) konvertiert wenn man den Operator für \(String\) und Zahl verwendet\\
\end{itemize}
\end{enumerate}

\subsubsection{Variabeln}
\label{sec:org433a780}
\begin{itemize}
\item Name der es erlauft, auf einen gespeicherten Wert zuzugreifen\\
\item verhindert Wiederholungen und doppeltes Berechnen\\
\item Ablauf\\
\begin{itemize}
\item Deklaration: gibt Name und Typ an\\
\begin{itemize}
\item Reserviert Speicher für den Wert\\
\(int myNumber;\)\\
\begin{center}
\includegraphics[width=.9\linewidth]{img/190927_13.jpg}
\end{center}
\end{itemize}
\item Zuweisung(\emph{Assignment}): speichert einen Wert in der Variabel\\
\begin{itemize}
\item mittels \(=\) Zeichen\\
\(myNumber = 5;\)\\
\begin{center}
\includegraphics[width=.9\linewidth]{img/190927_14.jpg}
\end{center}
\item Wiederholte Zuweisungen sind erlaubt\\
\end{itemize}
\item Gebrauch: in einem Ausdruck durch aufrufen des Variabelnamen\\
\end{itemize}

\item Deklaration und Zuwesung separat oder auch kombiniert in einer Zeile erfolgen\\
\(int x = 5;\)\\
\begin{center}
\includegraphics[width=.9\linewidth]{img/190927_15.jpg}
\end{center}

\item Java ist passed by Value\\
\item Können nur Werte ihres Types speichern\\
\end{itemize}

\subsection{2.3 Schleifen}
\label{sec:orgb9618c5}
\begin{itemize}
\item Wiederholtes Ausführen von Anweisungen\\
\end{itemize}

\subsubsection{\(for\) Schleife}
\label{sec:orgcab001c}
\begin{itemize}
\item wird einmal Inizialisiert\\
\begin{itemize}
\item legt die Variabel (Zählervariabel) der Schleife fest (\emph{loop counter})\\
\end{itemize}
\item solange \(Test\) true wird folgendes Wiederholt\\
\begin{itemize}
\item \(Statements\)\\
\item \(Update\)\\
\end{itemize}
\end{itemize}
\begin{center}
\includegraphics[width=.9\linewidth]{img/190927_16.jpg}
\end{center}

\begin{itemize}
\item Loop Counter muss sich ändern sonst terminiert der Loop nie\\
\end{itemize}
\begin{center}
\includegraphics[width=.9\linewidth]{img/190927_17.jpg}
\end{center}

\begin{itemize}
\item Können verschachtelt sein\\
\end{itemize}

\subsection{2.4 Methoden mit Parameter}
\label{sec:orgff2dd8e}
\begin{itemize}
\item Parameter: Ein Wert den eine aufgerufene Methode von der aufrufenden Methode erhält\\
\item Parametrifizierung: mit Parametern versehen\\
\item Parameter werden mit Datentyp bei der Methoden Deklaration angegeben\\
\begin{itemize}
\item \textbf{Alle} Parameter müssen im richtigen Datentyp gegeben werden\\
\end{itemize}
\item Parameter werden beim Aufrufen der Methode übergeben\\
\item Parameter in der Deklaration heisst Formaler Parameter (\emph{Formal Parameter})\\
\item Der tatsächlich übergebene Wert heisst tatäschlicher Parameter (\emph{Actual Parameter}) oder Argument\\
\item Argument wird in der Methode in einer Parameter Variable gespeichert\\
\item Die Methode erhält lediglich den Wert und hat z.B. keine Referenz zur Variabel von wo der Wert kommt\\
\begin{itemize}
\item Die Variabel die übergeben wurde kann sich nicht ändern, da die Methode keine Referenz zu ihr hat\\
\item Parameter der Basistypen werden bei der Übergabe kopiert (\emph{Value Semantics})\\
\end{itemize}
\end{itemize}


\subsection{2.5 if Anweisungen}
\label{sec:org58bc863}
\begin{itemize}
\item erlauben Verzweigungen\\
\item if Block nur ausgeführt, wenn ein bestimmter Test wahr ist\\
\begin{itemize}
\item else Block wird ansonnsten ausgeführt\\
\end{itemize}
\item Vergleichsketten sind in Java nicht erlaubt \sout{\(2 <= x <= 10\)}\\
\item Operatoren \&\&, || und ! um Aussagen zu verbinden\\
\item Pfade:\\
\begin{itemize}
\item Genau ein Pfad: if, else if, else\\
\item o oder 1 Pfad: if, else if, else if\\
\item 0, 1, oder viele: if, if, if\\
\end{itemize}
\end{itemize}

\subsection{Boolean}
\label{sec:org36b6a60}
\begin{itemize}
\item können nur wahr oder falsch sein\\
\item ein Vergleich ist ein Boolischer Ausdruck\\
\item werden mit boolischen Operatoren kombiniert\\
\item werden in Tests nicht mit true oder false verglichen\\
\begin{itemize}
\item \sout{isTrue == true}\\
\end{itemize}
\item werden von links nach rechts folgend ihrer Preszendenz ausgewertet\\
\item Bedingte Auswertung\\
\begin{itemize}
\item Java beendet die evaluation sobald ein teil des Tests falsch ist\\
\item (false \&\& egalWasHierSteht) -> egalWasHierSteht wird nicht ausgewertet\\
\end{itemize}
\end{itemize}


\subsection{{\bfseries\sffamily TODO} 2.X Einschub Input}
\label{sec:orgae41e05}
\begin{itemize}
\item Liest Input von der Konsole\\
\item Standard Input: Vordefiniertes Inputfenster\\
\begin{itemize}
\item \emph{System.in}\\
\end{itemize}
\item Standard Output: Vordefiniertes Outputfenster\\
\begin{itemize}
\item \emph{System.out}\\
\end{itemize}
\item Normalerweise int Input- und Outputfenster das selber\\
\item Output ist einfacher als Input\\
\begin{itemize}
\item Da man nicht weis was der Benutzer alles eingibt?\\
\end{itemize}
\item \emph{Scanner} erlaubt Services um von der Konsole zu lesen\\
\begin{itemize}
\item Erlaubt Input von unterschiedlichen Quellen (Datei, Konsole, Webseite\ldots{})\\
\item Kommt von Bibliotheke \emph{java.util}\\
\item Import mit /import java.util.*;\\
\item Scanner constuction: \emph{Scanner name = new Scanner(System.in);}\\
\item Wartet bis der User "ENTER" oder "RETURN" clickt\\
\end{itemize}
\end{itemize}


\begin{center}
\begin{tabular}{ll}
Methode & Description\\
\hline
nextInt() & reads an Int from the user and returnss it\\
nextDouble() & reads a Double from the user\\
next() & reads a one-word String from the user\\
nextLine() & reads a one-line String from the user\\
\end{tabular}
\end{center}

\begin{itemize}
\item Fordern (\emph{prompt}) the User auf eine Eingabe zu machen\\
\item \emph{nextInt()} liest eine Folge von Ziffern (Token) und wandelt diese in einen Int um\\
\begin{itemize}
\item Tokens werden durch \emph{whitespaces} getrennt (\emph{space}, \emph{tab}, \emph{new line})\\
\end{itemize}
\end{itemize}


\begin{verbatim}
  System.out.print("Wie alt sind Sie? "); // prompt
int alter = console.nextInt();
System.out.println("Ihre Eingabe " + alter);
\end{verbatim}


\subsection{{\bfseries\sffamily TODO} Strings}
\label{sec:org166cb94}
\begin{itemize}
\item Zugriff auf die Buchstaben mittels Index\\
\begin{itemize}
\item Startet mit Index 0\\
\item Buchstaben sind vom Basistyp \emph{char}\\
\end{itemize}
\item Ein String ist kein Array\\
\item Sind sehr wichtig und werden vom Compiler daher anders behandelt\\
\end{itemize}


\begin{center}
\begin{tabular}{ll}
Methode & Description\\
\hline
charAt(index) & Character at index\\
indexOf(str) & index where the start of the given string appears. If not found -1\\
length() & nunber of characters in this string\\
substring(index1, index2) & the characters in this string from index1 (inclusive) to index2 (exclusive)\\
substring(index1) & grabs till end of string\\
toLowerCase() & a new String with all lowercase lettes\\
toUpperCase() & a new String with all uppercase letters\\
\end{tabular}
\end{center}



\subsection{2.6 Nochmals Schleifen}
\label{sec:orgb7bb37c}
\subsubsection{Kurzform zur Aktualisierung des Loop Counters}
\label{sec:orgcbc01b2}
Increment oder decrement um 1 ist sehr häufig\\
\begin{itemize}
\item \(i = i + 1 \Rightarrow i ++\)\\
\begin{itemize}
\item \(i++\) wird zuerst verwendet, danach erhöht.\\
\(y = x++; \Rightarrow temp = x; x = x + 1; y = temp;\)\\
\item Diese Operatioen sind nicht so effizient wie man denken könnte und führen zu verwirrung und Fehler\\
\end{itemize}

\item \(i = i + 1 \Rightarrow i += 1\)\\

\item Increments in einer \(||\) und \(&&\) expression sind gefährlich, denn diese werden nicht zwingend ausgeführt\\
\begin{itemize}
\item \((ex1 || ex2) \neq (ex2 || ex1)\)\\
\end{itemize}
\end{itemize}

\subsubsection{Schleifen Probleme}
\label{sec:org09a1abb}
Wir wollen alle Zahlen, kommagetrennt bis \(n\) ausgeben.\\
\begin{itemize}
\item Verhindern das ein Komma am Schluss kommt\\
\item Lösung:\\
Wir Printen den erste Fahl ausserhalb des Loops und dann immer eine Komma, gefolgt von der Zahl. Oder können auch den letzten Fall am Ende nach dem Loop Printen\\
\item \emph{off-by-one} Error ist sehr läufig\\
Loop wurde einmal zuviel ode zu wenig ausgeführt\\
\end{itemize}

\subsubsection{Terminierung von Loops}
\label{sec:orgdca50be}
\begin{enumerate}
\item while Schleife
\label{sec:org6f86046}
\begin{itemize}
\item unkestimmte Schleife (\emph{idefinite loop}): Anzahl der Ausführungen ist im voraus nicht bekannt\\
\begin{center}
\includegraphics[width=.9\linewidth]{img/191004_1.jpg}
\end{center}

\item Sentinel: Ein Wert der das Ende einer Reihe anzeigt\\
\begin{itemize}
\item While loop wird terminiert sobald der sentinel eingegeben wird\\
\end{itemize}

\item do-wile löst das Problem das etwas bei eingabe eines sientinels nochmals eingegeben wird\\
\end{itemize}
\end{enumerate}

\subsection{2.7 Rückgabe für Methoden}
\label{sec:orgfd3f221}
\begin{itemize}
\item erlaubt kommunikation zwischen Aufrufer und Methode\\
\item \emph{Return value} wird an Aufrufer gegeben\\
\item \emph{return} gibt die Expression zurück und beendet die Methode\\
\begin{itemize}
\item kann auch keinen Wert zurückgeben\\
\end{itemize}
\item Typ muss stimmen\\
\item Bei verschachtelten If/Else macht der Kompiler manchmal blöd wenn ein If nur true sein kann.\\
\end{itemize}

\subsection{2.8 Sichtbarkeit von Vaiablennamen}
\label{sec:org0dfdd60}
\begin{itemize}
\item Sichtbarkeit (\emph{Scope})\\
\item Variabeln müssen deklariert sein bevor sie sichtbar sind\\
\item Sichtbar von Deklaration bis zum Ende des Blocks für den die Variable deklariert ist\\
\begin{itemize}
\item Block ist durch \{ und \} begrenzt\\
\end{itemize}
\end{itemize}


\section{3.0 Arrays}
\label{sec:org232b73c}
\begin{itemize}
\item Sind Objekte\\
\item Mehrere Werte des selben Typs speichern\\
\item Element: Wert eines Array\\
\item Index: Zahl un ein Element des Arrays auzuwählen\\
\item Base: Erste Element hat Index 0\\
\begin{verbatim}
type[] name = new type[length];
\end{verbatim}
\item Länge kann eine beliebiger int sein\\
\item Länge kann auch implizit gegeben sein durch die Anzahl elemente bei der Zuweisung\\
\item Anhängig von Typ wird der Array mit unterschiedlichen defaultwerten initiaisiert\\
\item Zugriff via Index:\\
\begin{verbatim}
name[index]; // Access
name[index] = value; // assign
\end{verbatim}
\item Legaler index zwischen 0 und lenght -1\\
\item \(name.length\) liefert die Anzahl Elemente des Arrays\\
\item werden häufig zusammen mit Schleifen verwendet\\
\item Methoden\\
\begin{center}
\begin{tabular}{ll}
binarySearch(array, value) & returns the index of the given value in a sorted array (or < 0 if not found)\\
copyOf(array, length) & returns a new copy of an array\\
equals(array1, array2) & returns true if the two arrays contain same elements in the same order\\
fill(array, value) & sets every element to the given value\\
sort(array) & arranges the elements into sorted order\\
toString(array) & returns a string representing the array, such as "[10, 30, - 25, 17]"\\
\end{tabular}
\end{center}
\item Bei der Übergabe von Arrays al Paameter oder als Return wert müssen wir [] Klammern verwenden\\
\end{itemize}

\section{4.0 Klassen und Objekte}
\label{sec:orgaf1ffc5}
\subsection{4.1 Klassen}
\label{sec:orgbeccfa2}
\begin{itemize}
\item wird mit Keyword \emph{class} definiert\\
\item Bieten einen \emph{Service} an\\
\item Typen von Klassen\\
\begin{itemize}
\item Namenloser Dienst\\
\begin{itemize}
\item wird geladen und ausgeführt\\
\end{itemize}
\item Mit Namen ausgewählter Dienste\\
\item Eigene Klassen\\
\end{itemize}
\item Beschreibt einen Typ\\
\item Referenzvariabel verweist auf ein Objekt\\
\item Objekt ist der Sammelbegriff aller Datenwerte die durch eine Klasse beschrieben werden\\
\begin{itemize}
\item wird erschaffen\\
\begin{itemize}
\item wird durch \emph{new} operator gemacht, oder durch eine Initialisierung\\
\begin{verbatim}
String s; int counter;
s = "hello " + counter;
\end{verbatim}
\end{itemize}
\end{itemize}
\item Vergleichsoperatoren funktionieren nicht für Objekte\\
\begin{itemize}
\item \("Hello" == "Hello" \Rightarrow false\)\\
\item \emph{equals} wird stattdessen verwend\\
\begin{center}
\begin{tabular}{ll}
equals(str) & whether two strings contain the same characters\\
equalsIgnoreCase(str) & whether two strings contain the same characters, ignoring upper vs. lower case\\
startsWith(str) & whether one contains other's characters at start\\
endsWith(str) & whether one contains other's characters at end\\
contains(str) & whether the given string is found within this one\\
\end{tabular}
\end{center}
\end{itemize}
\end{itemize}

\subsection{Referenzvariabel}
\label{sec:orgd2ba881}
\begin{itemize}
\item Referenzvariabel (\emph{reference type variable})\\
\item verweisen immer auf das selbe Objekt\\
\item bei der Zuweisung muss jedoch der Typ stimmen\\
\item wenn einer Referenzvariabel eine neue Referenzvariabel assigned wird, geht der initiale Wert verlohren(er existiert jedoch weiter im Speicher)\\
\item Diese "verlohreren" Werte werden automatisch gelöscht\\
\item Änderung der Referenzvariabel ändert die originale Variabel\\
\end{itemize}

\subsection{Value Semantics}
\label{sec:org4eca990}
\begin{itemize}
\item Passed by Value\\
\item trifft für die Primitive Types zu (int, boolean etc.)\\
\end{itemize}

\subsection{Reference Semantics}
\label{sec:org5ff5cdb}
\begin{itemize}
\item Passed by Reference (gibt es eigentlich nicht in Java, aber irgendwie trotzdem)\\
\item Erlauben einer Methode den Parameter direkt zu bearbeiten ohne diesen zu kopieren\\
\begin{itemize}
\item Updates in place\\
\end{itemize}
\item Sparen Zeite und Platz\\
\item Funktioniert mit Objekten\\
\end{itemize}

\subsection{4.4 Attribute}
\label{sec:org6dcd374}
\begin{itemize}
\item Variabel (\emph{field}) innerhalb eines Objects\\
\item Zugriff via Referenzvariabel\\
\item "null" kann verwendet werden um Referenzvariabeln zurückzusetzen\\
\item Dereferenzieren (\emph{dereference}) wia Dotnotation durch Referenzvariabel auf Wert zugreifen\\
\end{itemize}

\subsection{4.5 Methoden}
\label{sec:org8f64edc}
\begin{itemize}
\item Instanzmethode existiert innerhalb jedes Objekts einer Klasse und beschreibt das Verhalten eines Objects\\
\item Impliziter Parameter: Das Objekt für das die Methode aufgerufen wird\\
\begin{itemize}
\item wird immer übergeben\\
\item System weiss nicht wie ein Objekt gedruckt werden soll. Eine custom printmethode löst das Problem\\
\begin{itemize}
\item "toString()" wird immer verwendet wenn ein nicht String resp nicht primitive Type gedruckt wird\\
\end{itemize}
\end{itemize}
\item Accessor Methode wird verwendet um auf den Zustand eines Objekts zuzugreifen\\
\begin{itemize}
\item Read-only method\\
\end{itemize}
\item Mutator Method wird verwendet um den Zusatand eines Objekts zu verändern\\
\end{itemize}

\subsection{4.6 Konstruktor}
\label{sec:orga586227}
\begin{itemize}
\item Wird durch "new" verwendet um ein neues Objekt zu erstellen\\
\item Wir geben die Werte für die Attribute direkt mit bei der Initialisierung\\
\begin{itemize}
\item Der Kustruktor verarbeitet diese Parameter\\
\end{itemize}
\item Der Konstruktor hat keinen Typen, resp der Typ ist der Name der Klasse\\
\item Der Konstruktor ist keine Methode\\
\item Attribute die nicht durch Konstruktor gesetzt werden bekommen den Wert null\\
\item Wir können mehrere Konstruktoren haben, jedoch dürfen sie nicht die slebe Anzahl Parameter haben\\

\item in line if/Else: booleanExpression ? expression1 : expression2\\
\end{itemize}

\subsection{4.7 Sichtbarkeit für Attribute}
\label{sec:org8ec9ef0}
\begin{itemize}
\item Encapsulation\\
\begin{itemize}
\item Objekte haben einen Zustand, nicht nur Klasse\\
\item Können angeben von wo man auf welches Attribut zugreifen kann\\
\begin{itemize}
\item private: nicht von ausserhalb der Klasse zugreiffbar\\
\begin{itemize}
\item Zugriff von Accessor und Mutator Methoden geht\\
\end{itemize}
\end{itemize}
\end{itemize}
\item Bei Namenskonflikten "gewinnt" die innerte deklaration.\\
\item \emph{this} verweis innerhalb einer Methode auf das Objekt selbst\\
\begin{itemize}
\item kann auch innerhalb eines Konstruktors auf einen anderen Konstruktor zugreifen \emph{this(parameter1, parameter2)}\\
\end{itemize}

\item Shaddowing:\\
\begin{itemize}
\item zwei Veriabeln den selben Namen\\
\item Illegar in Java, ausser für Attribute\\
\end{itemize}

\item public macht etwas von überall her sichtbar\\
\end{itemize}

\subsection{4.8 Static Methoden und Variabeln}
\label{sec:org6c94181}
\begin{itemize}
\item Modul\\
\begin{itemize}
\item Kein eigenständiges Programm, sondern wird von einem anderen Programm verwendet.\\
\item Kein Mail Methode\\
\item Wird via die Klassenreferenz verwendet\\
\end{itemize}
\item static ist eine Ausnahme, normal ist mit Attributen\\
\item Leicht die übersicht zu verlieren\\
\begin{itemize}
\item wenig Gründe für static\\
\begin{itemize}
\item z.b. bei final (Wert kann nach einmal setzten nicht mehr verändert werden)\\
\end{itemize}
\end{itemize}
\item static sollte private oder final sein für attribute\\
\item statische Methoden dürfen nicht this verwenden!\\
\begin{itemize}
\item Objektattribute können nicht gelesen/geschrieben werden\\
\end{itemize}
\end{itemize}


\section{Einschub Random und Math}
\label{sec:orgd1e74f4}
\subsection{Random}
\label{sec:orgfd5e942}
\begin{itemize}
\item liefert Pseudozufallszahlen\\
\item teil von java.util\\
\begin{center}
\begin{tabular}{ll}
Method & Description\\
\hline
nextInt() & returns a random integer\\
nextInt(max) & returns a random integer in the range [0, max)\\
nextDouble() & returns a random real number in range [0.0, 1.0)\\
\end{tabular}
\end{center}
\end{itemize}

\subsection{Math}
\label{sec:orgaa74cab}
\begin{itemize}
\item Sammlung von Mathe Operanden\\
\item Teil von java.util\\
\item auch wenn ein Returnvalue ein theoretischer Int ist, wird ein Double returned\\
\end{itemize}

\section{4.3 Klassen (selber entwickeln)}
\label{sec:org42494f2}
\begin{itemize}
\item Objekte fassen zusammengehörende Daten zusammen\\
\item Programm wird übersichtlicher und robuster\\
\end{itemize}

\subsection{4.3.1 Einleitung}
\label{sec:orgb767301}
\begin{itemize}
\item Eine Art neue Typen zu beschreiben\\
\item Objekt (object): Gebilde das Zustand (state) und Verhalten (behaviour) verbindet\\
\begin{itemize}
\item stellt services zur Verfügung (Methoden)\\
\end{itemize}
\item OOP: Program als Menge von aufeinanderwirkenden Objekte\\
\item Klassen beschreiben Objekte\\
\item Objekte sind Exemplate (Instances) einer Klasse\\
\item OOP geht auch ohne Klassen, aber dies ist nicht verbreitet\\
\end{itemize}

\subsection{4.3.2 OOP in der Praxis}
\label{sec:orge999e2d}
\begin{itemize}
\item grosse Softwaresysteme\\
\item Modellierungen realter Simulationen\\
\item Abstraktion: lassen irrelevante Sachen weg in der Klasse\\
\item \emph{Client} verwendet eine bestimmte Klasse\\
\item Klassen schreibt man gross\\
\item Attribute werden wie bei Methoden übergeben\\
\item Zugriff auf Atribute via dot dotation\\
\end{itemize}

\section{5.1 Graphische Benutzeroberfläche}
\label{sec:org700a1f2}
\begin{itemize}
\item GUI: Graphical User Interface\\
\item Thema: Input / Output I/O\\

\item Vorteile GUI\\
\begin{itemize}
\item 2 oder 3 Dimensional\\
\item Eingabe durch Maus/Geste\\
\item Oft intuitiver und ansprechender\\
\end{itemize}

\item Nachteil GUI\\
\begin{itemize}
\item Höhere Komplexität\\
\begin{itemize}
\item Kontrollfluss: Programm muss immer auf den Benutzer reagieren\\
\item Verschiedene Steuerelemente\\
\item An Fenstergrösse anpassen\\
\end{itemize}
\end{itemize}
\end{itemize}

\subsection{Window}
\label{sec:org200aff4}
\begin{itemize}
\item new Window("name", b, h)\\

\item Window Methods\\
\begin{center}
\begin{tabular}{ll}
open() & öffne Fenster\\
close() & schliesse Fenster\\
waitUntilClosed() & warte bis manuell geschlossen\\
isOpen() & überprüfe ob geöffnet\\
fillRect(x, y, h, b) & Zeichne Rechteck\\
setColor(r, g, b) & ändere Farbe für fenster\\
fillRect() & \\
fillCircle() & \\
fillOval() & \\
fillRect(x, y, 1, 1) & Zeichnen eines pixels -> frei Zeichnen\\
refresh() & änderungen werden angezeigt\\
refresh(int waitTime) & wartet eine best Anzahl milisekunden\\
refreshAndClear() & macht alles weiss und bemalt es dann wieder\\
isKeyPressed() & \\
isLeftMouseButtonPressed() & \\
isRightMouseButtonPressed() & \\
wasKeyTyped() & \\
wasMouseButtonClicked() & \\
getMouseX() & \\
getMouseY() & \\
drawRect() & \\
drawCircle() & \\
drawLine() & \\
drawString() & \\
drawImage() & \\
drawImageCentered() & \\
setColor() & \\
setStrokeWidth() & \\
setFontSize() & \\
\end{tabular}
\end{center}

\item Input:\\
\begin{itemize}
\item Tasten drücken\\
\item Mausclick\\
\end{itemize}
\end{itemize}


\section{5.2 Input/Output mit Dateien}
\label{sec:orga5f7371}
\begin{itemize}
\item import java.io.*;\\

\item new File(fileName);\\
\begin{itemize}
\item Kann auch Ordner handeln\\
\end{itemize}

\item wird mit Scanner kombiniert um Files zu lesen. Dafür übergeben wir dem Scanner das File Object\\
\begin{itemize}
\item jedoch throwt der Scanner eine Exception welcher gehandelt werden muss\\
\begin{itemize}
\item falls man aus einer Dateil liest, welche nicht existiet\\
\item mit throws kann man sagen das eine Exception nicht gefangen werden soll, resp. weitergeleitet wird -> crash\\
\begin{itemize}
\item public static void foo() throws type\\
\end{itemize}
\end{itemize}
\end{itemize}
\end{itemize}


\begin{center}
\begin{tabular}{ll}
exists() & \\
canRead() & \\
getName() & \\
length() & \\
rename() & \\
\end{tabular}
\end{center}


\begin{itemize}
\item print(), println() kann auch zu File schreiben\\
\begin{itemize}
\item übergeben File Object als Parameter\\
\item alte Datei wird überschrieben\\
\end{itemize}

\item Alle System.out können über einen Alternativen Stream ausgegeben werden\\
\end{itemize}

\section{5.3 Scanner im Einsatz}
\label{sec:org885742c}
\begin{itemize}
\item jenachdem was wir für einen Typen werwarten benutzen wir eine andere lese Methode\\
\item input cursor scannt denn input -> identifiziert ein Token (wo fängt z.B. der nächste Int an)\\
\item Der Scanenr \emph{konsumiert} das Token (liest es ein)\\
\item mit \$.hasNextDouble()\$ kann mit einer Schleife eine unbekannte Anzahl Daten eingelesen werden\\
\begin{center}
\begin{tabular}{ll}
hasNext() & if there is a next token\\
hasNextInt() & if next token int exists\\
hasNextDouble() & if next token double exists\\
\end{tabular}
\end{center}
\item \$.next()\$ geht zum nächsten Token ohne es zu konsumieren\\
\item Zeilenbasierter Scanner kann verwendet werden wenn auf einer Zeile verschiedene Anzahle von verschienenen Typen existieren\\
\begin{center}
\begin{tabular}{ll}
nextLine() & Returns the entire line till \n\\
hasNextLine() & Checks if there is another line\\
\end{tabular}
\end{center}
\begin{itemize}
\item \n wird konsumiert, jedoch nicht an die Methode übergeben\\
\end{itemize}
\item Der Scanner auch auf einen String angewant werden\\
\begin{itemize}
\item Dies ist hilfreich wenn wird z.b. mit einem Scanner die Zeilen aus einem File einlesen, und ein 2. Scanner verarbeitet die Zeilen (Strings) dann\\
\end{itemize}
\item Wenn ein File bereits existsiert sollten wir e.v. ein neue File erstellen, da sonnst die File überschriben wird wenn der Scanner für den Output zuständig ist\\
\end{itemize}


\section{6.0) Arbeiten mit Objekten und Klassen}
\label{sec:orgb1d4c91}
\subsection{6.1) Einleigung-Datenstruktur mit Verknüpfungen}
\label{sec:org7fb463e}
\begin{itemize}
\item Arrays haben eine feste Strukture, jedoch wollen wir manchmal mehr "dynamisch" arbeiten können\\
\item Listen lösen diese Problem\\
\item Listen mit Construktoren sind mühsam\\
\end{itemize}


\subsection{6.2) Entwurf von abgekapselten Klassen}
\label{sec:org03717fd}
\begin{itemize}
\item System.exit(-1) wird verwendet um ein Programm zu beenden\\
\begin{itemize}
\item In einer return methode mussen wird ein "dummy" return einfügen damit der Compile nicht reklamiert\\
\end{itemize}
\item Methoden können den selben Namen haben wenn sie sich in der Parametern unterscheiden\\

\item Array Vorteile:\\
\begin{itemize}
\item Konstante Zugriffszeit\\
\end{itemize}

\item Vorteile List\\
\begin{itemize}
\item Länge variabel\\
\end{itemize}
\end{itemize}

\subsection{6.3) Hinweise und Regel für verständliche Programme}
\label{sec:org87d5228}
\subsubsection{Code Struktur}
\label{sec:org5d4a82a}
\begin{itemize}
\item gut lesbar und kompakt\\
\item Nicht mehr als 100 Zeichen Pro Zeile\\
\begin{itemize}
\item neue Zeile für jede Anweisung\\
\item Keine unnötigen neuen Zeilen\\
\end{itemize}
\item Aufeinandervolgende Anweisungen untereinander\\
\item Blöcke einrücken\\
\item Geschweifte Klammern verwenden\\
\item Klammern setzen\\
\item Lange if Statements mit oder oder und auf mehrere Zeilen\\
\item main Methode entweder zuoberst oder zuunterst\\
\end{itemize}

\subsubsection{Namen}
\label{sec:orgf919dca}
\begin{itemize}
\item Nur Buchstaben und Ziffern ohne Spezialzeichen\\
\item Klassennamen\\
\begin{itemize}
\item gross\\
\end{itemize}
\item Methodennamen\\
\begin{itemize}
\item klein\\
\item Verb\\
\end{itemize}
\item Variabelnnamen\\
\begin{itemize}
\item beschreibend\\
\end{itemize}
\item Kurze Namen für Loopcounter\\
\item Keine Typ/Metainformationen im Namen\\
\end{itemize}

\subsubsection{Weiteres}
\label{sec:orgb6a0a69}
\begin{itemize}
\item Masseinheiten soll im Namen vorhanden sein\\
\item Deklaration zusammen mit Initialisierung\\
\item Bei Arrays Eckigen Klammern hiter Typ und nicht hinter Namen\\
\item Durch Faktorisierung gemeinsame Codeelemente herausarbeiten -> Kompakterer Code\\
\item Keine unnötigen mehrfachverechnungen in if Statements da es den Code länger macht\\
\item Fragezeicheoperator: (test) ? value1 : value2\\
\begin{itemize}
\item kicht so effizient aber kompakter\\
\end{itemize}
\end{itemize}


\subsubsection{Speicher und Addressen}
\label{sec:org583dbc5}
\begin{itemize}
\item In 4 Digit Hex angegeben\\
\item Addressen sind in Java "unsichtbar"\\
\item 4 Bytes werden zu einem Wort zusammengefasst und daher sind die Addressen in 4er Schritten\\
\item Speicher ist in verschiedene Bereich aufgeteilt:\\
\begin{itemize}
\item Variabeln die immer existierene: in "static data"\\
\item durch new Operator erschaffene daten: "heap"\\
\item aufgerufene Methoden: "Stack Frame"\\
\begin{itemize}
\item In einem Stack organisiert\\
\end{itemize}
\end{itemize}
\item Müssen stets genügend Platz frei haben damit sich der Stack und Heap nicht überlappen\\
\item Reference Variabel enthält Addressen des Objekt Exemplars\\
\item Stack zeigt auf Exemplar im Heap mittels eines Pointers\\
\begin{itemize}
\item Gibt es Bereiche im Heap die keine Referenzen haben zählt es als garbage und wird von der garbage collection empfernt.\\
\end{itemize}
\item Valuesemantics und Referencesemantics wird im Kern gleich behandelt\\
\end{itemize}




\subsection{6.5) Mehr Optionen für Kontrolle der Sichbarkeit}
\label{sec:orgb42e39c}
\begin{itemize}
\item package ist eine Ansammlung von zusammengehörenden Klassen\\
\begin{itemize}
\item Datei kann nur in einem Package sein\\
\item Klassen legen fest in welcher Package sie ist\\
\item Schaffen einen namespace um namenskonflikte zu vermeinden\\
\end{itemize}
\item Package -> Verzeichnis\\
\item Klasse -> Datei\\
\item Root des Package wird durc den Class Path, oder Ausruferverzeichnis von java gegeben\\
\item Class Path:\\
\begin{itemize}
\item Path an welchem Java class Files sucht\\
\item Konfiguriert in Eclipse\\
\end{itemize}

\item Package\\
\begin{itemize}
\item Deklaration\\
\begin{itemize}
\item Anfang der Datei\\
\item package packageName;\\
\end{itemize}
\item Import\\
\begin{itemize}
\item import PackageName.*;\\
\item Namenskonflikte werden unterschiedlich behandelt\\
\end{itemize}
\item Können auch ohne import auf Packages zugreifen\\
\begin{itemize}
\item Durch den vollen Namen\\
\item packagename.className\\
\item Praktisch um namenskonflikte zu vermeiden\\
\end{itemize}
\item Gliedern das Projekt\\
\item Default Package\\
\begin{itemize}
\item alle Daten die nicht explizit in einem Package sind sind automatisch da\\
\item Können nicht importiert werden\\
\item Können nicht von anderen Klassen verwendet werden\\
\end{itemize}
\end{itemize}
\item Access Modifiers:\\
\begin{itemize}
\item Public: in allen Klassen nach Import\\
\item Private: nur in der Klasse selbst\\
\item default (package). In der Klasse und allen anderen Klasse des packages\\
\end{itemize}
\end{itemize}
\end{document}
