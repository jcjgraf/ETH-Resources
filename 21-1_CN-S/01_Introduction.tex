%! TEX root = ./main.tex

\section{Introduction}
\subsection{Components}
\begin{itemize}
    \item Are composed of three basic components
        \begin{itemize}
            \ides{End-Systems:}
                \begin{itemize}
                    \item Send and receive data
                    \item Represent the users of a networks
                    \item Huge diversity of such devices
                \end{itemize}
            \ides{Switches/Routers:} Forward data to the destination
            \ides{Links:} Interconnect switches/routers and end-systems
        \end{itemize}
    \ides{Last Mile:} Link which connects user to their ISP
        \begin{itemize}
            \item There are different types of last mile
                \ides{Digital Subscriber Line (DSL)}
                    \begin{itemize}
                        \item Used telephone lines
                            \begin{itemize}
                                \ipro Cheap because lines were already there
                            \end{itemize}
                        \item Composed of:
                            \begin{itemize}
                                \item Upstream data
                                \item Downstream data
                                \item $2$-way phone channel
                            \end{itemize}
                        \item Asynchronous
                    \end{itemize}
                \ides{Cable Access Technology (CATV)}
                    \begin{itemize}
                        \item Uses TV cable
                        \item Coaxial copper cable for house connection
                        \item Fiber cable for household junction to ISP
                        \item Composed of:
                            \begin{itemize}
                                \item Downstream data
                                \item Upstream data
                            \end{itemize}
                        \item Asynchronous
                        \item Many households share the same connection
                            \begin{itemize}
                                \icon Vulnerable to hacks
                            \end{itemize}
                    \end{itemize}
                \ides{Enterprise Access}
                    \begin{itemize}
                        \item Relatively directly connected to ISP
                        \item Local switches and aggregated switches
                    \end{itemize}
                \ides{Ethernet}
                    \begin{itemize}
                        \item Most widely used local area network technology
                        \item Twisted pair copper
                        \item Symmetric (same speed in both direction)
                    \end{itemize}
                \ides{Undersea Cable}
                    \begin{itemize}
                        \item Different hardening depending on the environment
                    \end{itemize}
                \ides{Other}
                    \begin{itemize}
                        \item Many other technologies
                        \item Satellites, balloons etc.
                    \end{itemize}
        \end{itemize}
    \ides{Internet:} Network of networks
        \begin{itemize}
            \item Enormously complex
            \item We consider a abstraction
            \item A network is graph with nodes and edged:
                \begin{itemize}
                    \ides{Node:}
                        \begin{itemize}
                            \ides{Capacity:} How much data can they transfer at a given time?
                            \ides{Degree:} How are they connected to other nodes?
                        \end{itemize}
                    \ides{Edges:}
                        \begin{itemize}
                            \ides{Directed:} Can data travel in both directions?
                            \ides{Capacity:} How much data can they transfer at a given time?
                            \ides{Latency:} How much time does it take to transfer data?
                        \end{itemize}
                \end{itemize}
        \end{itemize}
\end{itemize}

\subsection{Resources Sharing}
\begin{itemize}
    \item Links are shared among users
    \ides{Oversubscribe:} Assumed that not all user will use it at the same time
    \item Terminology
        \begin{itemize}
            \ides{P:} Peak rate
            \ides{A:} Average rate
            \ides{A/P:} Level of utilisation in $\%$
        \end{itemize}
    \item Two main principles
    \ides{Circuit Switching/Reservation}
        \begin{itemize}
            \item Reserve the bandwidth $P$ for all switches on the path in advance
            \item Only use $A/P\%$
            \item Good when $P/A$ small
            \item Multiplexed at flow-level
            \item Two different methods
            \ides{Time-base:} Periodic timeslots are reserved
            \ides{Frequency-base:} Specific frequencies are reserved
            \item Both methods can be combined
            \ipro Predictable performance
            \icon Does not routes around failed switches
                \begin{itemize}
                    \item Setup of new path required
                \end{itemize}
            \icon Long setup and take down of the link
                \begin{itemize}
                    \item Not suited for short communication
                \end{itemize}
            \item Used in phone networks
        \end{itemize}
    \ides{Packet Switching/On-Demand}
        \begin{itemize}
            \item Send data when needed
            \item Good when $P/A$ large
            \item Multiplexed at packet-level
            \ipro Efficient use of resource
            \ipro Routes around failed switches
            \icon Unpredictable performance
            \icon Packets can clash
                \begin{itemize}
                    \item I.e. arrive at a switch/router at the same time
                    \item Requires buffers
                \end{itemize}
            \icon Network can ``overflow'' (congestion)
            \item Used by the internet
        \end{itemize}
\end{itemize}

\subsection{Network Layer}
\begin{itemize}
    \item Network is structured in $5$ layer
    \item Each layer has its protocols
    \item Each layer builds on a service provided by the lower layer
    \item Each layer provides a service to the upper layer
    \item OSI model has $7$ layers
    \ides{Physical Layer:}
        \begin{itemize}
            \item Provides physical transfer of bits
            \item Moves \textit{bits}
        \end{itemize}
    \ides{Link Layer:}
        \begin{itemize}
            \item Provides local best-effort delivery
            \item Moves \textit{frames}
        \end{itemize}
    \ides{Network Layer:}
        \begin{itemize}
            \item Provides global best-effort delivery
            \item Moves \textit{packet}
        \end{itemize}
    \ides{Transport Layer:}
        \begin{itemize}
            \item Provides end-to-end delivery (reliable or not)
            \item Transports \textit{segments}
        \end{itemize}
    \ides{Application Layer:}
        \begin{itemize}
            \item Provides network access
            \item Exchanges \textit{messages}
        \end{itemize}
    \item Each layer adds its own header to the message before transmitting it to the lower level
    \item Different devices implement different layers
        \begin{itemize}
            \item Host implements all layers
            \item Router implements up to L3
            \item Switch implements up to L2
        \end{itemize}
\end{itemize}

\subsection{End-To-End Principle}
\begin{itemize}
    \item Network is unreliable, but try to make it as reliable as possible
        \begin{itemize}
            \item Else we have too many end-to-end retries
        \end{itemize}
    \item End-to-end check \textbf{always} necessary
    \ides{Fate-Sharing Principle:} Acceptable to loose state but only if the entity is lost too
        \begin{itemize}
            \item Store state always co-local to entities
        \end{itemize}
\end{itemize}

\subsection{Characteristics and Performance}
\begin{itemize}
    \item Network is characterized by delay, loss and throughput
    \ides{Delay}
        \begin{itemize}
            \item Total delay is sum of:
                \begin{itemize}
                    \ides{Transmission Delay}
                        \begin{itemize}
                            \item Time to push all bits onto the link
                            \item $\text{Transmission delay [sec]} = \frac{\text{packet size [\# bits]}}{\text{line bandwith [\# bits/sec]}}$
                            \item Due to link properties
                        \end{itemize}
                    \ides{Propagation Delay}
                        \begin{itemize}
                            \item Time for a bit to travel through the link
                            \item $\text{Propagation delay [sec]} = \frac{\text{link length [m]}}{\text{propagation speed [m/sec]}}$
                            \item Propagation speed is a fraction of $c$ (speed of light)
                            \item Due to link properties
                        \end{itemize}
                    \ides{Processing Delay}
                        \begin{itemize}
                            \item Time to process a request
                            \item Due to traffic and switch internals
                            \item Tiny
                                \begin{itemize}
                                    \item Can be neglected
                                \end{itemize}
                        \end{itemize}
                    \ides{Queuing Delay}
                        \begin{itemize}
                            \item Time a packet waits in the buffer for transmission
                            \item Varies from packet to packet
                                \begin{itemize}
                                    \item Hard to predict
                                    \item Analysis requires statistical measures
                                \end{itemize}
                            \item Depends on:
                                \begin{itemize}
                                    \item Arrival rate of new pack
                                    \item Transmission rate of outgoing link
                                    \item Traffic burstiness
                                \end{itemize}
                            \ides{Analysis}
                                \begin{itemize}
                                    \ides{a [packets/sec]:} average packet arrival rate
                                    \ides{R [bit/sec]:} transmission rate on outgoing link
                                    \ides{L [bit]:} fixed packet length
                                    \ides{La [bit/sec]:} average bits arrival rate
                                    \ides{La/R:} traffic intensity
                                    \item Queuing delay is exponential with increasing traffic intensity
                                    \item Diverges at $1$
                                    \item Increases fast if traffic is bursty
                                    \item Goal: operate far from critical point
                                \end{itemize}
                            \item Due to traffic and switch internals
                        \end{itemize}
                \end{itemize}
            \item Cannot predict which delay dominates
        \end{itemize}
    \ides{Loss}
        \begin{itemize}
            \item Queue can hold at most $N$ packets in their buffer
                \begin{itemize}
                    \item $N + 1$ packet gets dropped
                \end{itemize}
            \item Upper bound $= N \cdot L / R$
            \item Loss may be acceptable or not
                \begin{itemize}
                    \item Can resend redundant information as $A, B, A \oplus B$
                \end{itemize}
        \end{itemize}
    \ides{Throughput}
        \begin{itemize}
            \item Rate at which a host receives data
            \item $\text{Average throughput [\# bits/sec]} = \frac{\text{data size [\# bits]}}{\text{transfer time [sec]}}$
            \item End-to-end throughput is equal to the throughput of the bottleneck
                \begin{itemize}
                    \item I.e. the min of the throughputs of all links
                \end{itemize}
        \end{itemize}
\end{itemize}
