% ! TEX root = ./main.tex

\section{Grundideen}
\begin{itemize}
    \item Man fasst die Daten $x_1, \dots, x_n$ auf als Realisierungen $X_1(\omega), \dots, X_n(\omega)$ von ZV $X_1, \dots, X_n$, und sucht dann Aussagen über die Verteilung von $X_1, \dots, X_n$.
        \begin{itemize}
            \item $x_1, \dots, x_n$ sind Daten (in der Regel Zahlen)
            \item $X_1, \dots, X_n$ sind der generierende Mechanismus (ZV)
            \ides{Stichprobe:} ZV $X_1, \dots, X_n$
            \ides{Stichprobenumfang:} Anzahl Stichproben $n$
        \end{itemize}
    \item Wir haben einen Datensatz aus einer Stichprobe und suchen dafür ein Modell welches durch einen ggf. hochdimensionalen parameter $\vartheta \in \Theta$ beschreibbar ist. $P_\vartheta$ beschreibt das Wahrscheinlichkeitsmass abhängig von $\vartheta$. Wir versuchen nun über die Daten Rückschlüsse auf $\Theta$ zu ziehen.
    \item Parametrische Statistische Analyse:
        \begin{itemize}
            \item[1.] Aufgrund von Daten/Graphen eine Idee für die Wahl einer geeigneten Modelierung zu finden
            \item[2.] \textbf{Wahl eines Modells:} Paremetermenge $\Theta$ und Familie $(P_\vartheta)_{\vartheta \in \Theta}$ von Modellen spezifizieren
            \item[3.] Schätzung der Parameter durch Benutzung eines Schätzers
            \item[4.] Überprüfen ob gewähltes $\vartheta$ bzw. Modell $P_\vartheta$ gut passt durch geeignete statistische Tests
            \item[5.] Aussagen über Zuverlässigkeit der Schätzungen machen. Ev. Bereich in $\Theta$ wählen statt einzelner parameter $\vartheta$ (Konfidenzbereich)
        \end{itemize}
\end{itemize}
