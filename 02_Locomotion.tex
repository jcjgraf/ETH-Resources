%! TEX root = ./main.tex

\section{Locomotion}
\begin{itemize}
    \item Physical interaction between vehicle and environment
    \ides{Stability:} characterized by
        \begin{itemize*}
            \item Number of contact pts
            \item CoG
            \item Static/dynamic stabilization
            \item Inclination of terrain
        \end{itemize*}
    \ides{Contact:} characterized by
        \begin{itemize*}
            \item Contact pt size and shape
            \item Angle of contact
            \item Friction
        \end{itemize*}
    \ides{Environment:} characterized by
        \begin{itemize*}
            \item Structure
            \item Medium
        \end{itemize*}
    \ides{Implementation Aspects:}
        \begin{itemize*}
            \item Number of actuators
            \item Structural complexity
            \item control complexity
            \item Energy consumption
        \end{itemize*}
\end{itemize}

\subsection{Legged Locomotion}
\begin{itemize*}
    \ipro Mobility
    \ipro Adaptability
    \ipro Ability to manipulate environment
    \icon Mechanical complexity
    \icon Control complexity
    \icon Energy Consumption
\end{itemize*}
\begin{itemize}
    \ides{Static Gaits:}
        \begin{itemize*}
            \item System is statically sable
            \item Requires $\ge 4$ legs
            \ipro Safe
            \icon Slow
            \icon Energetically inefficient
        \end{itemize*}
    \ides{Dynamic Gaits:}
        \begin{itemize*}
            \item System is stabilized on a limited cycle
            \item Falls over if stopped
            \ipro Fast
            \ipro Energetically efficient
            \icon Demanding for actuators
            \icon Demanding for control
        \end{itemize*}
    \ides{Locomotion Control}
        \begin{itemize}
            \item[1)] Stepping sequence defined by gait pattern
            \item[2)] Stepping location
            \item[3)] Contact forces
        \end{itemize}
\end{itemize}

\subsection{Wheeled Locomotion}
\begin{itemize*}
    \ipro Highly efficient on flat surface
\end{itemize*}

\subsubsection{Wheel Types}
\begin{itemize}
    \ides{Standard Wheel:}
        \begin{itemize*}
            \item $2$ DoF
                (\begin{itemize*}
                    \item wheel axle
                    \item wheel contact point
                \end{itemize*})
            \item Can be steered or fixed
            \item Steering without side effects on the body
        \end{itemize*}
    \ides{Castor Wheel:}
        \begin{itemize*}
            \item $3$ DoF
                (\begin{itemize*}
                    \item wheel axle
                    \item wheel contact point
                    \item offset steering joint
                \end{itemize*})
            \item Steering must be compensated by the body
        \end{itemize*}
    \ides{Swedish Wheel:}
        \begin{itemize*}
            \item $3$ DoF
                (\begin{itemize*}
                    \item Rotation around the wheel axle
                    \item Rotation around the contact point
                    \item Rotation around the rollers
                \end{itemize*})
            \item Similar to standard wheel but provides low resistance in another direction
        \end{itemize*}
    \ides{Spherical Wheel:}
        \begin{itemize*}
            \item $3$ DoF
            \icon Difficult to realize
        \end{itemize*}
\end{itemize}

\subsubsection{Wheel Arrangements}
\begin{itemize}
    \item Influences manoeuvrability, controllability and stability
    \item No optimal arrangement
    \ides{Stability:}
        \begin{itemize*}
            \item $\ge 3$ wheels required for static stability
            \item Improved by adding more wheels
        \end{itemize*}
    \ides{Manoeuvrability:}
        \begin{itemize*}
            \item Number of DoF
        \end{itemize*}
    \ides{Controllability:}
        \begin{itemize*}
            \item Inverse correlation between controllability and manoeuvrability
            \item Less manoeuvrability robots can be controlled more accurately
        \end{itemize*}
\end{itemize}
