%! TEX root  = ./main.tex

\section{Ordinary Differential Equations (ODE)}
Equation where the unknown is a function $y = f(x)$ and the equation relates values of $y$ and its derivatives $y', y'' \dots$ at a single point $x$.

\subsection{Properties}
\begin{compactdesc}
    \item[Order:] Highest derivative appearing in the DE.
\end{compactdesc}

\subsection{Linear Differential Equations}
A linear ODE of order $k$ is a DE of the form $y^{(k)} + a_{k - 1}y^{(k - 1)} + \dots + a_1y' + a_0y = b$ where $a_i, b, y$ are functions on $I \subset \mathbb{K}$. Meaning, all terms $y, y' \dots$ appear linearly.

\subsubsection{Types}
\begin{compactdesc}
    \item[Homogenous Linear ODE:] RHS $b = 0$
    \item[Inhomogenous Linear ODE:] RHS $b \neq 0$
\end{compactdesc}

\subsubsection{Solutions}
The general solution to a inhomogeneous linear ODE is $y = y_h + y_p$. The solutions set is $S_b = \{y_h + y_p\}$.
\begin{compactdesc}
    \item[$\mathbf{y_h}$:] General solution to the homogeneous linear DE. $S_0$ is the solution set.
        \begin{compactitem}
            \item $S_0$ is a vector space with dimension equal to the order of the DE
        \end{compactitem}
    \item[$\mathbf{y_p}$:] A particular solution the inhomogeneous linear DE.
\end{compactdesc}

\paragraph{Construction of Solutions}
The following follows from the linearity of the DE:
\begin{compactitem}
    \item If $y_1$ and $y_2$ are two different solutions to an inhomogeneous linear DE, then $y = c_1 \cdot y_1 + c_2 \cdot c_2$ is also a solution.
    \item Superposition: If $y_1$ solves the linear DE for RHS $b_1$ and $y_2$ solves the linear DE for RHS $b_2$, then $y_1 + y_2$ solves the linear DE for RHS $b_1 + b_2$.
\end{compactitem}

\subsubsection{Initial Value Problem}
For any DE of order $k$ the initial condition is a set of $k$ equations $y^{(i)} (x_0) = y_i, 1 \le i \le k$ at some initial point $x_0$.  For any $k$ initial conditions there exists a unique solution $y$ (valid for homogeneous and inhomogeneous).

\recipe{Solve $y' + ay = b$}
\begin{compactenum}
    \item Compute $\int a \mathrm{d}x = A(x)$
    \item Formulate $y_h = z e^{-A(x)}$ for $z \in C$.
    \item Calculate $y_p = (\int b(x) e^{A(x)} \mathrm{d}x) e^{-A(x)}$.
    \item Form general solution $y = y_h + y_p$.
    \item If initial condition is given, solve for $z$.
\end{compactenum}

\subsection{Linear ODE With Constant Coefficients}
Equivalent to regular linear ODE with the difference that $a_i \in \C$.

\subsubsection{Characteristic Polynomial}
For lin. ODE with constant coefficients $y^{(k)} + a_{k - 1}y^{(k - 1)} + \dots + a_1y' + a_0y = 0$ the characteristic polynomial is $P(\lambda) = \lambda^k + a_{k - 1} \lambda^{k - 1} + \dots + a_0$.

The $k$ zeros of $P(\lambda)$ are the eigenvalues.

\subsubsection{Solution}
Any general solution is again of the form $y = y_h + y_p$.

\paragraph{Homogeneous $y_h$}
\begin{compactitem}
    \item For district eigenvalues $\lambda_1, \dots, \lambda_r$ with multiplicity $m_1, \dots, m_r$ the function $f_{j, l}, x \mapsto x^l e^{\lambda_j x}, \quad 1 \le j \le r, 0 \le l <m_j$ gives a system of solutions. I.e. all terms are multiplied with a constant $z_i$ and summed up.
    \item If $\lambda_i = \beta + i \gamma$ is a EV then so is $\bar{\lambda_i} = \beta - i \gamma$.
    \item If a root $\lambda_i$ is complex, i.e. $\lambda_i = \beta_i \pm i\gamma_i$, with multiplicity $m_i$ then they contribute the system $x^l e^{\beta_i x} (\cos(\gamma_ix) + i \sin(\gamma_ix))$ to the solution.
    \item For complex roots $\beta + i \gamma$, which are part of a solution system, the following transformation is useful: $z_1 e^{(\beta + i \gamma) x} + z_2 e^{(\beta - i \gamma) x} = \tilde{z_1} e^{\beta x} \cos(\gamma x) + \tilde{z_2} e^{\beta x} \sin(\gamma x)$.
\end{compactitem}

\paragraph{Inhomogeneous $y_p$}

\subparagraph{Method of Undetermined Coefficients/Ansatz}
Solution will be similar to disturbance function $b(x)$.

\begin{tabular}{l | l}
    $b(x)$                                  & Ansatz\\\hline
    $a$                                     & $b$\\
    $P_n(x)$                                & $Q_n(x)$\\
    $P_n e^{\alpha x}$                      & $Q_n e^{\alpha x}$\\
    $P_n \sin(\beta x)$                     & $Q_n \sin(\beta x) + R_n \cos(\beta x)$\\
    $P_n \cos(\beta x)$                     & $Q_n \sin(\beta x) + R_n \cos(\beta x)$\\
    $P_n \cos(\beta x)$                     & $Q_n \sin(\beta x) + R_n \cos(\beta x)$\\
    $P_n \sin(\beta x) + G_n \cos(\beta x)$ & $Q_n \sin(\beta x) + R_n \cos(\beta x)$\\
    $a e^{\alpha x} \sin(\beta x)$          & $e^{\alpha x} (c \sin(\beta x) + d \cos(\beta x))$\\
    $b e^{\alpha x} \cos(\beta x)$          & \\
    $P_n(x) e^{\alpha x}$                   & $R_n(x) e^{\alpha x}$\\
    $P_n(x) e^{\alpha x} \sin(\beta x)$     & $e^{\alpha x}(R_n(x) \sin(\beta x) + S_n(x)\cos(\beta x))$\\
    $Q_n(x) e^{\alpha x} \cos(\beta x)$     & \\
\end{tabular}

\begin{compactitem}
    \item If $b(x)$ is a lin. combination of functions we try
        \begin{inparaitem}
            \item Lin. comb. of Ansatz functions
            \item Superposition
        \end{inparaitem}
    \item If $\lambda_i$ is a EW with multiplicity $m_i$ then we multiply the Ansatz by $x^{m_i}$.
    \item If a Ansatz is equal to solution of the homogeneous equation we have to multiply the Ansatz by $x$.
\end{compactitem}

\recipe{Solve using Ansatz}
\begin{compactenum}
\item Select a suitable Ansatz and set $y_p =$ chosen Ansatz (check special cases listed above).
    \item Calculate required derivatives $y_p^{(i)}$.
    \item Insert $y_p$ and derivatives into the ODE.
    \item Solve for the constant in the Ansatz.
    \item $y_p$ is equal the Ansatz with replaces constant.
\end{compactenum}

\subparagraph{Variation of Constants}
Assume $k = 2$, i.e. $y'' + a_1y' + a_0y = b$ and a homogeneous solution $y_h = z_1 f_1 + z_2 + f_2$, where $f_1, f_2$ are independent solutions.

For $y_p$ we try $y_p = z_1(x) f_1 + z_2(x) f_2$. To determine the unknown functions $z_1(x), z_2(x)$ we try:
\begin{align*}
    z_1'f_1 + z_2'(x)f_2 =& 0\\ 
    z_1'f_1 + z_2'(x)f_2 =& b
\end{align*}

$
\underbrace{
\begin{pmatrix}
    f_1 & f_2\\
    f_1' & f_2'
\end{pmatrix}
}_{=A}
\begin{pmatrix}
    z_1'(x)\\
    z_2'(x)
\end{pmatrix}=
\begin{pmatrix}
    0\\
    b
\end{pmatrix}\implies
\begin{pmatrix}
    z_1'(x)\\
    z_2'(x)
\end{pmatrix}= 
A^{-1}
\begin{pmatrix}
    0\\
    b
\end{pmatrix}
$

Which gives the particular solution $y_p = z_1(x) f_1 + z_2(x) f_2$.

\subsection{Separation of Variables}
A ODE is separable if it is of the form $y' = b(x)g(y)$.

We can separate the variables $x, y$.

$\frac{dy}{dx} = b(x)g(y) \implies \frac{dy}{g(y)} = b(x)dx \implies \int \frac{1}{g(y)} \mathrm{d}y = \int b(x)\mathrm{d}x$
